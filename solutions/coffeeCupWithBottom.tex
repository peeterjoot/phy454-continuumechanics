%
% Copyright � 2012 Peeter Joot.  All Rights Reserved.
% Licenced as described in the file LICENSE under the root directory of this GIT repository.
%
\label{chap:coffeeCupWithBottom}
\section{Motivation.}
%
Having tackled the problem of the spin down of a bottomless cup of coffee, lets try setting up the harder problem with a non-infinite cup.  It turns out that this is a lot harder.  Even the steady state solution requires a superposition of Bessel functions (whereas we only had that in the bottomless problem when we tried to find the time evolution after ceasing the stirring).

I can find the form of the solution, but do not know how to actually apply the boundary value constraints.  I also find the no-slip constraints themselves become problematic, because they lead to an inconsistency at the point of contact of a moving and a static interface.  Before continuing, I need to find out how to deal with that inconsistency.
%
\section{Navier-Stokes for the problem.}
%
Working in cylindrical coordinates is the only sensible option.  Let us look first at the steady state problem, with the cup being stirred at at constant rate with the unrealistic rotating cylinder  stir stick used previously.  We will assume an azimuthal velocity profile
%
\begin{equation}\label{eqn:coffeeCupWithBottom:10}
\Bu = u(r, z, t) \phicap.
\end{equation}
%
Let us first verify that this satisfies our incompressible condition
%
%
\begin{dmath}\label{eqn:coffeeCupWithBottom:30}
\spacegrad \cdot \Bu
=
\left( \rcap \partial_r + \frac{\phicap}{r} \partial_\phi + \zcap \partial_z \right) \cdot ( u \phicap )
=
\cancel{\rcap \cdot \phicap} \partial_r u
+ \frac{\phicap}{r} \cdot (u \partial_\phi \phicap + \phicap \cancel{\partial_\phi u})
+ \cancel{\zcap \cdot \phicap} \partial_z u.
\end{dmath}
%
The only term that survives is the \(\partial_\phi \phicap = -\rcap\) but that is perpendicular to \(\phicap\) so we have \(0 = \spacegrad \cdot \Bu\) as desired.  This leaves us with
%
%
\begin{dmath}\label{eqn:coffeeCupWithBottom:50}
\frac{u}{r} \partial_\phi ( u \phicap )
= -\inv{\rho}
\left( \rcap \partial_r + \frac{\phicap}{r} \partial_\phi + \zcap \partial_z \right) p
+ \nu
\left( \inv{r} \partial_r ( r \partial_r ) + \frac{1}{r^2} \partial_{\phi \phi} + \partial_{zz} \right) u \phicap + \Bg
\end{dmath}
%
Splitting into \(\phicap, \rcap, \zcap\) coordinates respectively we have
%
\begin{subequations}
\begin{equation}\label{eqn:coffeeCupWithBottom:70}
0 = -\inv{r \rho} \partial_\phi p + \nu \left( \inv{r} \partial_r ( r \partial_r u ) - \frac{u}{r^2} + \partial_{zz} u \right)
\end{equation}
\begin{equation}\label{eqn:coffeeCupWithBottom:90}
- \frac{u^2}{r} = -\inv{\rho} \partial_r p
\end{equation}
\begin{equation}\label{eqn:coffeeCupWithBottom:110}
0 = -\inv{\rho} \partial_z p - g.
\end{equation}
\end{subequations}
%
Demanding a symmetrical solution kills off the pressure term in \eqnref{eqn:coffeeCupWithBottom:70}, and we can proceed with separation of variables to find the allowable solutions for the velocity before imposing our boundary value conditions.  With \(u = R(r) Z(z)\) we have
%
\begin{equation}\label{eqn:coffeeCupWithBottom:130}
\inv{r R} ( R' + r R'' ) - \frac{1}{r^2} = -\frac{Z''}{Z} = -\inv{a^2} = \inv{b^2}
\end{equation}
%
Should we pick a negative or a positive constant here (ie: trig or hyperbolic functions for \(Z\))?  Allowing for both temporarily, we find for \(R\)
%
\begin{subequations}
\begin{equation}\label{eqn:coffeeCupWithBottom:150}
0 = r R' + r^2 R'' + R \left( -1 + \frac{r^2}{a^2} \right)
\end{equation}
\begin{equation}\label{eqn:coffeeCupWithBottom:170}
0 = r R' + r^2 R'' + R \left( -1 - \frac{r^2}{b^2} \right).
\end{equation}
\end{subequations}
%
Using Mathematica (\nbref{coffeeCupWithBottom.cdf}) to look up the solutions, we find that this was a Bessel equation with solutions
%
\begin{subequations}
\begin{equation}\label{eqn:coffeeCupWithBottom:190}
R(r) \in \Span \{ J_1(r/a), Y_1(r/a) \}
\end{equation}
\begin{equation}\label{eqn:coffeeCupWithBottom:210}
R(r) \in \Span \{ J_1(i r/b), Y_1(i r/b) \}.
\end{equation}
\end{subequations}
%
However, the second set are not real valued for \(r > 0\).  This means we want the hyperbolic solutions for \(Z\).  Because we also have a boundary value constrain of \(u = 0\) on the bottom of the cup, we can pick only hyperbolic sine solutions.  As before, we will ``stir'' the coffee with a cylinder at radius \(s\) (with cup radius \(R\)), our solution has the form
%
\begin{equation}\label{eqn:coffeeCupWithBottom:230}
u(r, z, 0) =
\left\{
\begin{array}{l l}
s \Omega
\sum
C_a J_1(r/a) \sinh(z/a)
& \quad \mbox{\(r < s\)} \\
s \Omega
\sum
\left( D_a J_1(r/a) + E_a Y_1(r/a) \right) \sinh(z/a)
 & \quad \mbox{\(r \in [s, R]\)} \\
\end{array}
\right.
\end{equation}
%
Observe that, regardless of \(a\) we have \(u(0, z, 0) = 0\) because \(J_1(0) = 0\).  We also can not scale \(a\) as \(\lambda_i/s\) (where \(\lambda_i\) are the zeros of \(J_1\)) when the stirring is not at the edge since we need \(u(s, z, 0) = \Omega s\).  That suggests that we probably want to write our system as
%
\begin{equation}\label{eqn:coffeeCupWithBottom:230b}
u(r, z, 0) =
\left\{
\begin{array}{l l}
s \Omega
\sum
C_i J_1(\lambda_i r/R) \sinh(\lambda_i z/R)
& \quad \mbox{\(r < s\)} \\
s \Omega
\sum
\left( D_i J_1(\lambda_i r/R) + E_i Y_1(\lambda_i r/R) \right) \sinh(\lambda_i z/R)
 & \quad \mbox{\(r \in [s, R]\)} \\
\end{array}
\right.
\end{equation}
%
The boundary value constraints of \(u(r = s) = \Omega s\) and \(u(r = R) = 0\) require the solution of
%
\begin{subequations}
\begin{equation}\label{eqn:coffeeCupWithBottom:310}
1 = \sum C_i J_1(\lambda_i s/R) \sinh(\lambda_i z/R)
\end{equation}
\begin{equation}\label{eqn:coffeeCupWithBottom:330}
1 = \sum \left( D_i J_1(\lambda_i s/R) + E_i Y_1(\lambda_i s/R) \right) \sinh(\lambda_i z/R)
\end{equation}
\begin{equation}\label{eqn:coffeeCupWithBottom:350}
0 = \sum E_i Y_1(\lambda_i) \sinh(\lambda_i z/R).
\end{equation}
\end{subequations}
We know how to solve a system like \eqnref{eqn:coffeeCupWithBottom:310} if we did not have the \(\sinh\) term in the mix.  Can we look for a numerical (least squares?) solution to this more general problem.  Will it be possible to find something that is a good fit regardless of the value of \(z\)?

We have other troubles too.  We can not simultaneously satisfy the boundary value condition of \(u(s, z) = \Omega s\) and also \(u(s, 0) = 0\).  Should we attempt something like a least squares solution and start going near \(z = 0\) the small \(\sinh\) values near there will start forcing the \(C_i\)'s arbitrarily high.

We have to somehow modify the no-slip condition so that when we have a moving interface in contact with a static interface we somehow deal with the fact that the no-slip constraints cannot simultaneously require the velocity to match both the moving and static interfaces at that point of contact.
%
\section{Base of cup no-slip troubles.}
\index{no slip}
%
We have just found the functional form for an azimuthal flow that has \(z\)-axis dependence.  Attempting to apply the no-slip boundary value condition for our mixing device got us into trouble, since this simultaneously require zero and non-zero velocity at the point of contact of the mixing interface and the bottom of the vessel (i.e. where the stir stick contacts the bottom of the cup).  We can avoid this issue by constraining the mixing to only occur above the bottom of the cup, and then look at the flow that this induces below the mixing point.

Let us attempt to solve this new simpler boundary value problem.  We will position the mixing cylinder at a height \(d\) above the bottom of the cup, and set the radius of that inner cylinder to \(s\) as before, with the mixing occurring at an angular velocity of \(\Omega\).  We now want apply these boundary value constrains to the velocity function we have found for the steady state problem.  Provided \(z < d\), this will have the form
%
\begin{equation}\label{eqn:coffeeCupWithBottom:500}
u(r, z, 0) = s \Omega \sum C_i J_1(\lambda_i r/R) \sinh(\lambda_i z/R)
\end{equation}
%
where \(\lambda_i\) are the zeros of the order one Bessel function \(J_1\).  Observe that our boundary value conditions of \(u(R, z, 0) = 0\) and \(u(r, 0, 0) = 0\) are automatically satisfied.  Note that because of the \(u(R, z, 0) = 0\) equality here, this form of solution is no good if we are mixing at the edge of the cup, so we require \(s \ne R\).

Our remaining boundary value condition \(u(s, d, 0) = s \Omega\), means that we have to solve
%
\begin{equation}\label{eqn:coffeeCupWithBottom:520}
1 = \sum C_i J_1(\lambda_i s/R) \sinh(\lambda_i d/R)
\end{equation}
%
Having had the same sort of problem in our steady state bottomless coffee problem, we know how to solve this
%
\begin{equation}\label{eqn:coffeeCupWithBottom:540}
C_i \sinh(\lambda_i d/R) =
\frac{
\int_0^1 w J_1 (\lambda_i w) dw
}{
\int_0^1 w J_1^2 (\lambda_i w) dw
}.
\end{equation}
%
So we find that below the point of stirring (\(z = d\)), our steady state solution for the velocity should be
%
\begin{equation}\label{eqn:coffeeCupWithBottom:560}
\Bu(r, z, 0) = s \Omega \phicap \sum_{i=1}^\infty
\frac{
\int_0^1 w J_1 (\lambda_i w) dw
}{
\int_0^1 w J_1^2 (\lambda_i w) dw
}
J_1(\lambda_i r/R) \frac{\sinh(\lambda_i z/R)}{\sinh(\lambda_i d/R) }
\end{equation}
%
For a cup size of \(R = 5 \text{cm}\), stir radius of \(s = 3 \text{cm}\), stir depth \(d = 2 \text{cm}\), and angular velocity \(\Omega = 2 \pi \text{rad}/\text{s}\), we can compute (\nbref{coffeeCupWithBottom.cdf}) some terms of this series
%
%
\begin{dmath}\label{eqn:coffeeCupWithBottom:580}
u(r, z, 0) =
27.1119 J_{1}(7.66341 r) \sinh (7.66341 z)-3.42819 J_{1}(14.0312 r) \sinh (14.0312 z)
+4.97807 J_{1}(20.3469 r) \sinh (20.3469 z)-1.53542 J_{1}(26.6474 r) \sinh (26.6474 z) + \cdots
\end{dmath}
%
We can plot this velocity field as in \cref{fig:coffeeCupWithBottom:coffeeCupWithBottomFig2}, but it is hard to see the radial dependence.
%
\imageFigure{../figures/phy454-continuumechanics/coffeeCupWithBottomVector_field_plot_of_the_velocity_field_below_the_stir_depthFig2}{Vector field plot of the velocity field below the stir depth.}{fig:coffeeCupWithBottom:coffeeCupWithBottomFig2}{0.3}
%
The radial dependence of the magnitude of the velocity can be seen in \cref{fig:coffeeCupWithBottom:coffeeCupWithBottomFig1}, which plots \(u(r, d, 0)\).  We see the zero velocity at the edges of the cup as expected, and once we hit the stir height, matching of stir velocity.  An animation showing the variation of the radial velocity profile at various depths up to the stir height is available at \youtubehref{BS8XQdXljSk}.  Observing this we see that the velocity is dominated by the first term in the Bessel series, essentially just scaled by the hyperbolic sine that multiplies it.
%
\imageFigure{../figures/phy454-continuumechanics/coffeeCupWithBottomRadial_velocity_dependence_at_the_stir_heightFig1}{Radial velocity dependence at the stir height.}{fig:coffeeCupWithBottom:coffeeCupWithBottomFig1}{0.2}
%
It is not clear to me how to compute the velocity profile above the point of lowest stir depth.  Let us ignore that for now, and think a bit about the spin down problem.
%
\section{Spin down below the stir point.}
%
Let us now consider the time evolution problem when we stop the stirring abruptly at \(t = 0\).  The \(\phicap\) component of Navier-Stokes becomes
%
\begin{equation}\label{eqn:coffeeCupWithBottom:800}
\PD{t}{u} =
\nu
\left( \inv{r} \partial_r ( r \partial_r u ) -\frac{u}{r^2} + \partial_{zz} u \right)
\end{equation}
%
with \(u = R(r) Z(z) T(t)\) we find
%
\begin{equation}\label{eqn:coffeeCupWithBottom:620b}
\frac{T'}{T} = -\nu \frac{\alpha^2}{R^2} =
\nu
\left( \inv{r R} ( r R' )' -\frac{1}{r^2} + \frac{Z''}{Z} \right).
\end{equation}
%
We have immediately
%
\begin{equation}\label{eqn:coffeeCupWithBottom:640b}
T \propto e^{-\alpha^2 \nu t/R^2},
\end{equation}
%
and can insist that we also have
%
\begin{subequations}
\begin{equation}\label{eqn:coffeeCupWithBottom:660b}
\inv{r R} ( r R' )' -\frac{1}{r^2} = -\frac{\lambda_i^2}{R^2}
\end{equation}
\begin{equation}\label{eqn:coffeeCupWithBottom:680b}
\frac{Z''}{Z} = \frac{\lambda_i^2}{R^2} - \frac{\alpha^2}{R^2}
\end{equation}
\end{subequations}
%
Our time dependent solution below the point of stirring is therefore of the form
%
%
\begin{dmath}\label{eqn:coffeeCupWithBottom:700b}
u(r, z, t)
=
s \Omega \sum_{\lambda_i^2 > \alpha^2} c_{i \alpha} J_1(\lambda_i r/R) \sinh\left( \sqrt{\lambda_i^2 - \alpha^2} z/R \right) e^{-\alpha^2 \nu t/R^2}
+s \Omega \sum_{\lambda_i^2 < \alpha^2} d_{i \alpha} J_1(\lambda_i r/R) \sin\left( \sqrt{\alpha^2 -\lambda_i^2} z/R \right) e^{-\alpha^2 \nu t/R^2}
\end{dmath}
%
with a boundary value condition
%
%
%\begin{dmath}\label{eqn:coffeeCupWithBottom:720f}
%\sum_{i=1}^\infty
%\frac{
%\int_0^1 w J_1 (\lambda_i w) dw
%}{
%\int_0^1 w J_1^2 (\lambda_i w) dw
%}
%J_1(\lambda_i r/R) \frac{\sinh(\lambda_i z/R)}{\sinh(\lambda_i d/R) }
%=
%\sum_{\lambda_j^2 > \alpha^2} c_{j \alpha} J_1(\lambda_j r/R) \sinh\left( \sqrt{\lambda_j^2 - \alpha^2} z/R \right)
%+\sum_{\lambda_j^2 < \alpha^2} d_{j \alpha} J_1(\lambda_j r/R) \sin\left( \sqrt{\alpha^2 -\lambda_j^2 } z/R \right)
%\end{dmath}
\begin{equation}\label{eqn:coffeeCupWithBottom:720f}
\begin{aligned}
&\sum_{i=1}^\infty
\frac{
\int_0^1 w J_1 (\lambda_i w) dw
}{
\int_0^1 w J_1^2 (\lambda_i w) dw
}
J_1(\lambda_i r/R) \frac{\sinh(\lambda_i z/R)}{\sinh(\lambda_i d/R) } \\
&\quad =
\sum_{\lambda_j^2 > \alpha^2} c_{j \alpha} J_1(\lambda_j r/R) \sinh\left( \sqrt{\lambda_j^2 - \alpha^2} z/R \right)  \\
&\qquad+\sum_{\lambda_j^2 < \alpha^2} d_{j \alpha} J_1(\lambda_j r/R) \sin\left( \sqrt{\alpha^2 -\lambda_j^2 } z/R \right).
\end{aligned}
\end{equation}
I had like to go back and understand the Sturm Liouville theory in \citep{sagan1989boundary} that I used the result from to solve the steady state problem.  Thinking about what I have been using, we basically been using a weighted inner product, so that if we are looking for a fit for \(x \in [0,1]\) of
\begin{equation}\label{eqn:coffeeCupWithBottom:820}
\phi(x) = \sum_i c_i J_1(\lambda_i x),
\end{equation}
we have been utilizing a weighted inner product relation of the form
\begin{equation}\label{eqn:coffeeCupWithBottom:840}
\int_0^1 x J_1(\lambda_i x) J_1(\lambda_j x) dx = \delta_{ij} \int_0^1 x J_1^2(\lambda_i x) dx.
\end{equation}
Suppose that we drop the hyperbolic terms above, and insist that
\begin{equation}\label{eqn:coffeeCupWithBottom:860}
\sqrt{\alpha^2 - \lambda_j^2} \frac{d}{R} = 2 \pi j,
\end{equation}
or
\begin{equation}\label{eqn:coffeeCupWithBottom:880}
\frac{\alpha^2}{R^2} = \frac{4 \pi^2 j^2}{d^2} + \frac{\lambda_j^2}{d^2},
\end{equation}
leaving us with the hope that we can find \(d_j = d_{j\alpha}\) so that for \(x,y \in [0,1]\) we have
\begin{equation}\label{eqn:coffeeCupWithBottom:900}
\sum_{i=1}^\infty
\frac{
\int_0^1 w J_1 (\lambda_i w) dw
}{
\int_0^1 w J_1^2 (\lambda_i w) dw
}
J_1(\lambda_i x) \frac{\sinh(\lambda_i y d/R)}{\sinh(\lambda_i d/R) }
=
\sum_{j = 1}^\infty d_j J_1(\lambda_j x) \sin\left( 2 \pi j y \right).
\end{equation}
%
Observing that
\begin{equation}\label{eqn:coffeeCupWithBottom:920}
\int_0^1 \sin(2 \pi j y) \sin(2 \pi k y) dy = \inv{2} \delta_{j k}.
\end{equation}
We can multiply both sides by \(x J_1(\lambda_k x) \sin ( 2 \pi m y )\) and integrating over the unit square we have
%
%\begin{dmath}\label{eqn:coffeeCupWithBottom:940}
%\sum_{i=1}^\infty
%\frac{
%\int_0^1 w J_1 (\lambda_i w) dw
%}{
%\int_0^1 w J_1^2 (\lambda_i w) dw
%}
%\int_0^1 x J_1(\lambda_i x) J_1( \lambda_k x) dx \int_0^1 \sin( 2 \pi m y ) \frac{\sinh(\lambda_i y d/R)}{\sinh(\lambda_i d/R) } dy
%=
%\sum_{j = 1}^\infty d_j \int_0^1 x J_1(\lambda_j x) J_1( \lambda_k x) dx \int_0^1 \sin( 2 \pi m y ) \sin\left( 2 \pi j y \right) dy.
%\end{dmath}
\begin{equation}\label{eqn:coffeeCupWithBottom:940}
\begin{aligned}
\sum_{i=1}^\infty
&\frac{
\int_0^1 w J_1 (\lambda_i w) dw
}{
\int_0^1 w J_1^2 (\lambda_i w) dw
} \\
&=
\sum_{j = 1}^\infty d_j \int_0^1 x J_1(\lambda_j x) J_1( \lambda_k x) dx \int_0^1 \sin( 2 \pi m y ) \sin\left( 2 \pi j y \right) dy.
\end{aligned}
\end{equation}
%
Applying the orthogonality relations we have
%
%\begin{dmath}\label{eqn:coffeeCupWithBottom:960}
%\sum_{i=1}^\infty
%\frac{
%\int_0^1 w J_1 (\lambda_i w) dw
%}{
%\int_0^1 w J_1^2 (\lambda_i w) dw
%}
%\int_0^1 x J_1^2(\lambda_i x) \delta_{i k} \int_0^1 \sin( 2 \pi m y ) \frac{\sinh(\lambda_i y d/R)}{\sinh(\lambda_i d/R) }
%=
%\sum_{j = 1}^\infty d_j \int_0^1 x J_1^2(\lambda_j x) dx \delta_{j k} \inv{2} \delta_{m j},
%\end{dmath}
\begin{equation}\label{eqn:coffeeCupWithBottom:960}
\begin{aligned}
\sum_{i=1}^\infty
&\frac{
\int_0^1 w J_1 (\lambda_i w) dw
}{
\int_0^1 w J_1^2 (\lambda_i w) dw
}
\int_0^1 x J_1^2(\lambda_i x) \delta_{i k} \int_0^1 \sin( 2 \pi m y ) \frac{\sinh(\lambda_i y d/R)}{\sinh(\lambda_i d/R) } \\
&=
\sum_{j = 1}^\infty d_j \int_0^1 x J_1^2(\lambda_j x) dx \delta_{j k} \inv{2} \delta_{m j},
\end{aligned}
\end{equation}
or
\begin{equation}\label{eqn:coffeeCupWithBottom:980}
\int_0^1 w J_1 (\lambda_k w) dw
\int_0^1 \sin( 2 \pi k y ) \frac{\sinh(\lambda_k y d/R)}{\sinh(\lambda_k d/R) }
=
\inv{2} d_k \int_0^1 x J_1^2(\lambda_k x) dx.
\end{equation}
%
We have got the same Bessel integral on both sides leaving us with
\begin{equation}\label{eqn:coffeeCupWithBottom:1000}
d_k = 2 \int_0^1 \sin( 2 \pi k y ) \frac{\sinh(\lambda_k y d/R)}{\sinh(\lambda_k d/R) } dy
=
\frac{4 \pi k (-1)^{k+1} \sinh \left(\frac{d \lambda_k}{R}\right)}{\frac{d^2 \lambda_k^2}{R^2} + 4 k^2 \pi^2}
\end{equation}
%
Putting all the pieces together we have for the spin down of the fluid below the stir height for \(t \ge 0\).
\begin{dmath}\label{eqn:coffeeCupWithBottom:1020}
\Bu(r, z, t)
=
\frac{\Omega s}{\pi} \phicap \sum_{j = 1}^\infty
(-1)^{j+1}
\frac{j }{\frac{d^2 \lambda_j^2 }{4 R^2 \pi^2}+ j^2}
\sinh \left(\frac{d \lambda_j}{R}\right)
\exp\left( -\left( \frac{\lambda_j^2}{R^2} + \frac{4 \pi^2 j^2}{d^2} \right) \nu t\right)
J_1 \left(\frac{\lambda_j r}{R}\right)
\sin\left( \frac{2 \pi j z}{d} \right)
.
\end{dmath}
%
It is interesting that we end up with products of the orthonormal Bessel and trig functions when we started with a requirement for Bessel times hyperbolic trig functions.  We can likely utilize a similar Fourier decomposition of the hyperbolic trig functions to solve the steady state problem above the stir point, then solve for all our Fourier coefficients using this technique.  That will obviously be a harder problem, but one that looks at least feasible.  Regardless, we have to start at the bottom of the cup and work our boundary value conditions up from there.

We have a result that appears to validate the claim in \citep{acheson1990elementary} about the importance of the bottom of the cup in this problem.  Let us look at the magnitude of the viscous boundary layer term and see if we can estimate approximately when the spin down is mostly complete.

Let us also plot this function and see that it matches up with what is we have previously computed for the steady state problem.

Plotting this gave me something that looked completely bogus.  Going back and looking where things went wrong, I see that even my steady state ``solution'' is wrong.  The problems start at \eqnref{eqn:coffeeCupWithBottom:520} where we have matched values at a single point \(r = s\), and then integrated over \(s\) as if it was a variable.  Also later in my spin down work, I think I am too loose with my deltas, and that is probably wrong too.  Back to the drawing board.

NOTE: constructing the Fourier fit for a \(\sinh\) curve I see that sine, cosine and a constant term is required.  Have to rework with this in mind.  Can not just pick the sine function so that we match \(u = 0\) at \(z = 0\), but need to match at all \(z\).  Is that constant term going to be trouble?
