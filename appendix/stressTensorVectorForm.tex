%
% Copyright � 2012 Peeter Joot.  All Rights Reserved.
% Licenced as described in the file LICENSE under the root directory of this GIT repository.
%
%
%
\label{chap:continuumstressTensorVectorForm}
\section{Motivation.}
%
Exercise 6.1 from \citep{acheson1990elementary} is to show that the traction vector can be written in vector form (a rather curious thing to have to say) as
%
\begin{equation}\label{eqn:stressTensorVectorForm:10}
\Bt = -p \ncap + \mu ( 2 (\ncap \cdot \spacegrad)\Bu + \ncap \cross (\spacegrad \cross \Bu)).
\end{equation}
%
Note that the text uses a wedge symbol for the cross product, and I have switched to standard notation.  I have done so because the use of a Geometric-Algebra wedge product also can be used to express this relationship, in which case we would write
%
\begin{equation}\label{eqn:stressTensorVectorForm:30}
\Bt = -p \ncap + \mu ( 2 (\ncap \cdot \spacegrad) \Bu + (\spacegrad \wedge \Bu) \cdot \ncap).
\end{equation}
%
In either case we have
%
\begin{equation}\label{eqn:stressTensorVectorForm:50}
(\spacegrad \wedge \Bu) \cdot \ncap
=
\ncap \cross (\spacegrad \cross \Bu)
=
\spacegrad' (\ncap \cdot \Bu') - (\ncap \cdot \spacegrad) \Bu,
\end{equation}
%
(where the primes indicate the scope of the gradient, showing here that we are operating only on \(\Bu\), and not \(\ncap\)).
%
After computing this, lets also compute the stress tensor in cylindrical and spherical coordinates (a portion of that is also problem 6.10), something that this allows us to do fairly easily without having to deal with the second order terms that we encountered doing this by computing the difference of squared displacements.
%
We will work primarily with just the strain tensor portion of the traction vector expressions above, calculating
%
\begin{equation}\label{eqn:stressTensorVectorForm:250}
2 {\Be}_{\ncap}
=
2 (\ncap \cdot \spacegrad)\Bu + \ncap \cross (\spacegrad \cross \Bu)
=
2 (\ncap \cdot \spacegrad)\Bu + (\spacegrad \wedge \Bu) \cdot \ncap.
\end{equation}
%
We will see that this gives us a nice way to interpret these tensor relationships.  The interpretation was less clear when we computed this from the second order difference method, but here we see that we are just looking at the components of the force in each of the respective directions, dependent on which way our normal is specified.
\section{Verifying the relationship.}
Let us start with the plain old cross product version
%
\begin{equation}\label{eqn:stressTensorVectorForm:970}
\begin{aligned}
(\ncap \cross (\spacegrad \cross \Bu) + 2 (\ncap \cdot \spacegrad) \Bu)_i
&=
n_a (\spacegrad \cross \Bu)_b \epsilon_{a b i}  + 2 n_a \partial_a u_i \\
&=
n_a \partial_r u_s \epsilon_{r s b} \epsilon_{a b i}  + 2 n_a \partial_a u_i \\
&=
n_a \partial_r u_s \delta_{ia}^{[rs]} + 2 n_a \partial_a u_i \\
&=
n_a ( \partial_i u_a -\partial_a u_i ) + 2 n_a \partial_a u_i \\
&=
n_a \partial_i u_a + n_a \partial_a u_i \\
&=
n_a (\partial_i u_a + \partial_a u_i) \\
&=
\sigma_{i a } n_a.
\end{aligned}
\end{equation}
We can also put the double cross product in wedge product \index{wedge product}form
%
\begin{equation}\label{eqn:stressTensorVectorForm:990}
\begin{aligned}
\ncap \cross (\spacegrad \cross \Bu)
&=
-I \ncap \wedge (\spacegrad \cross \Bu) \\
&=
-\frac{I}{2}
\left(
\ncap (\spacegrad \cross \Bu)
- (\spacegrad \cross \Bu) \ncap
\right) \\
&=
-\frac{I}{2}
\left(
-I \ncap (\spacegrad \wedge \Bu)
+ I (\spacegrad \wedge \Bu) \ncap
\right) \\
&=
-\frac{I^2}{2}
\left(
- \ncap (\spacegrad \wedge \Bu)
+ (\spacegrad \wedge \Bu) \ncap
\right) \\
&=
(\spacegrad \wedge \Bu) \cdot \ncap.
\end{aligned}
\end{equation}
Equivalently (and easier) we can just expand the dot product of the wedge and the vector using the relationship
%
\begin{equation}\label{eqn:stressTensorVectorForm:70}
\Ba \cdot (\Bc \wedge \Bd \wedge \Be \wedge \cdots )
=
(\Ba \cdot \Bc) (\Bd \wedge \Be \wedge \cdots ) - (\Ba \cdot \Bd) (\Bc \wedge \Be \wedge \cdots ),
\end{equation}
so we find
%
\begin{equation}\label{eqn:stressTensorVectorForm:1010}
\begin{aligned}
((\spacegrad \wedge \Bu) \cdot \ncap + 2 (\ncap \cdot \spacegrad) \Bu
)_i
&=
(
\spacegrad' (\Bu' \cdot \ncap)
-
(\ncap \cdot \spacegrad) \Bu
+ 2 (\ncap \cdot \spacegrad) \Bu
)_i \\
&=
\partial_i u_a n_a
+
n_a \partial_a u_i \\
&=
\sigma_{ia} n_a.
\end{aligned}
\end{equation}
\section{Cylindrical strain tensor.}
\index{strain tensor}
Let us now compute the strain tensor (and implicitly the traction vector) in cylindrical coordinates.
%
Our gradient in cylindrical coordinates is the familiar
%
\begin{equation}\label{eqn:stressTensorVectorForm:110b}
\spacegrad = \rcap \PD{r}{} + \phicap \inv{r }\PD{\phi}{} + \zcap \PD{z}{},
\end{equation}
and our cylindrical velocity is
%
\begin{equation}\label{eqn:stressTensorVectorForm:111}
\Bu = \rcap u_r + \phicap u_\phi + \zcap u_z.
\end{equation}
Our curl is then
%
\begin{equation}\label{eqn:stressTensorVectorForm:1030}
\begin{aligned}
\spacegrad \wedge \Bu
&=
\left(
\rcap \PD{r}{} + \phicap \inv{r }\PD{\phi}{} + \zcap \PD{z}{}
\right)
\wedge
\left(
\rcap u_r + \phicap u_\phi + \zcap u_z
\right) \\
&=
\rcap \wedge \phicap
\left(
\partial_r u_\phi
-\inv{r} \partial_\phi u_r
\right) \\
&+
\phicap \wedge \zcap
\left(
\inv{r} \partial_\phi u_z
- \partial_z u_\phi
\right) \\
&+
\zcap \wedge \rcap
\left(
\partial_z u_r - \partial_r u_z
\right) \\
&+
\inv{r} \phicap \wedge \left(
(\partial_\phi \rcap) u_r
+
(\partial_\phi \phicap) u_\phi
\right).
\end{aligned}
\end{equation}
%
Since \(\partial_\phi \rcap = \thetacap\) and \(\partial_\phi \phicap = -\rcap\), we have only one cross term and our curl is
\begin{equation}\label{eqn:stressTensorVectorForm:610}
\begin{aligned}
\spacegrad \wedge \Bu
&=
\rcap \wedge \phicap
\left(
\partial_r u_\phi
-\inv{r} \partial_\phi u_r
+ \frac{u_\phi}{r}
\right) \\
&\quad +
\phicap \wedge \zcap
\left(
\inv{r} \partial_\phi u_z
- \partial_z u_\phi
\right) \\
&\quad +
\zcap \wedge \rcap
\left(
\partial_z u_r - \partial_r u_z
\right).
\end{aligned}
\end{equation}
We can now move on to compute the directional derivatives and complete the strain calculation in cylindrical coordinates.  Let us consider this computation of the stress for normals in each direction in term.
\subsection{Outwards radial normal \texorpdfstring{\(\ncap = \rcap\)}{ncap equals rcap}.}
Our directional derivative component for a \(\rcap\) normal direction does not have any cross terms
\begin{equation}\label{eqn:stressTensorVectorForm:1050}
\begin{aligned}
2 (\rcap \cdot \spacegrad) \Bu
&=
2 \partial_r
\left(
\rcap u_r + \phicap u_\phi + \zcap u_z
\right) \\
&=
2
\left(
\rcap \partial_r u_r + \phicap \partial_r u_\phi + \zcap \partial_r u_z
\right).
\end{aligned}
\end{equation}
Projecting our curl bivector onto the \(\rcap\) direction we have
%
\begin{equation}\label{eqn:stressTensorVectorForm:1070}
\begin{aligned}
(\spacegrad \wedge \Bu) \cdot \rcap
&=
(\rcap \wedge \phicap) \cdot \rcap
\left(
\partial_r u_\phi
-\inv{r} \partial_\phi u_r
+ \frac{u_\phi}{r}
\right) \\
&+
(\phicap \wedge \zcap) \cdot \rcap
\left(
\inv{r} \partial_\phi u_z
- \partial_z u_\phi
\right) \\
&+
(\zcap \wedge \rcap) \cdot \rcap
\left(
\partial_z u_r - \partial_r u_z
\right) \\
&=
-\phicap
\left(
\partial_r u_\phi
-\inv{r} \partial_\phi u_r
+ \frac{u_\phi}{r}
\right)
+
\zcap
\left(
\partial_z u_r - \partial_r u_z
\right).
\end{aligned}
\end{equation}
Putting things together we have
%
\begin{equation}\label{eqn:stressTensorVectorForm:1090}
\begin{aligned}
2 \Be_{\rcap}
&=
2
\left(
   \rcap \partial_r u_r + \phicap \partial_r u_\phi + \zcap \partial_r u_z
\right)
-\phicap
\left(
   \partial_r u_\phi
   -\inv{r} \partial_\phi u_r
   + \frac{u_\phi}{r}
\right) \\
&\quad+
\zcap
\left(
   \partial_z u_r - \partial_r u_z
\right) \\
&=
\rcap
\left(
   2 \partial_r u_r
\right)
+
\phicap
\left(
   2 \partial_r u_\phi
   -\partial_r u_\phi
   +\inv{r} \partial_\phi u_r
   - \frac{u_\phi}{r}
\right) \\
&\quad +
\zcap
\left(
   2 \partial_r u_z
   +\partial_z u_r - \partial_r u_z
\right).
\end{aligned}
\end{equation}
%
For our stress tensor
\begin{equation}\label{eqn:stressTensorVectorForm:290c}
\Bsigma_{\rcap} = - p \rcap + 2 \mu e_{\rcap},
\end{equation}
we can now read off our components by taking dot products to yield
\begin{subequations}
\begin{equation}\label{eqn:stressTensorVectorForm:630}
\sigma_{rr}
=
-p + 2 \mu \PD{r}{u_r},
\end{equation}
\begin{equation}\label{eqn:stressTensorVectorForm:650}
\sigma_{r \phi}
=
\mu \left(
 \PD{r}{u_\phi}
+\inv{r} \PD{\phi}{u_r}
- \frac{u_\phi}{r}
\right),
\end{equation}
\begin{equation}\label{eqn:stressTensorVectorForm:670}
\sigma_{r z}
=
\mu \left(
 \PD{r}{u_z}
+\PD{z}{u_r}
\right).
\end{equation}
\end{subequations}
\subsection{Azimuthal normal \texorpdfstring{\(\ncap = \phicap\)}{ncap equals phicap}.}
%
Our directional derivative component for a \(\phicap\) normal direction will have some cross terms since both \(\rcap\) and \(\phicap\) are functions of \(\phi\)
\begin{equation}\label{eqn:stressTensorVectorForm:1110}
\begin{aligned}
2 (\phicap \cdot \spacegrad) \Bu
&=
\frac{2}{r}
\partial_\phi
\left(
\rcap u_r + \phicap u_\phi + \zcap u_z
\right) \\
&=
\frac{2}{r}
\left(
\rcap \partial_\phi u_r + \phicap \partial_\phi u_\phi + \zcap \partial_\phi u_z
+(\partial_\phi \rcap) u_r + (\partial_\phi \phicap) u_\phi
\right) \\
&=
\frac{2}{r}
\left(
\rcap (\partial_\phi u_r - u_\phi) + \phicap (\partial_\phi u_\phi + u_r )+ \zcap \partial_\phi u_z
\right) \\
\end{aligned}
\end{equation}
Projecting our curl bivector onto the \(\phicap\) direction we have
%
\begin{equation}\label{eqn:stressTensorVectorForm:1130}
\begin{aligned}
(\spacegrad \wedge \Bu) \cdot \phicap
&=
(\rcap \wedge \phicap) \cdot \phicap
\left(
\partial_r u_\phi
-\inv{r} \partial_\phi u_r
+ \frac{u_\phi}{r}
\right) \\
&+
(\phicap \wedge \zcap) \cdot \phicap
\left(
\inv{r} \partial_\phi u_z
- \partial_z u_\phi
\right) \\
&+
(\zcap \wedge \rcap) \cdot \phicap
\left(
\partial_z u_r - \partial_r u_z
\right) \\
&=
\rcap
\left(
\partial_r u_\phi
-\inv{r} \partial_\phi u_r
+ \frac{u_\phi}{r}
\right)
-
\zcap
\left(
\inv{r} \partial_\phi u_z
- \partial_z u_\phi
\right)
\end{aligned}
\end{equation}
Putting things together we have
%
\begin{equation}\label{eqn:stressTensorVectorForm:1150}
\begin{aligned}
2 \Be_{\phicap}
&=
\frac{2}{r}
\left(
   \rcap (\partial_\phi u_r - u_\phi) + \phicap (\partial_\phi u_\phi + u_r )+ \zcap \partial_\phi u_z
\right) \\
&\quad +\rcap
\left(
   \partial_r u_\phi
   -\inv{r} \partial_\phi u_r
   + \frac{u_\phi}{r}
\right) \\
&\quad-
\zcap
\left(
   \inv{r} \partial_\phi u_z
   - \partial_z u_\phi
\right) \\
&=
\rcap
\left(
\frac{1}{r}\partial_\phi u_r
-\frac{u_\phi}{r}
+\partial_r u_\phi
\right)
+
\frac{2}{r} \phicap
\left(
\partial_\phi u_\phi + u_r
\right)
+
\zcap
\left(
\frac{1}{r} \partial_\phi u_z
    + \partial_z u_\phi
\right).
\end{aligned}
\end{equation}
%
For our stress tensor
\begin{equation}\label{eqn:stressTensorVectorForm:690}
\Bsigma_{\phicap} = - p \phicap + 2 \mu e_{\phicap},
\end{equation}
so we can now read off our components by taking dot products to yield
%
\begin{subequations}
\begin{equation}\label{eqn:stressTensorVectorForm:710}
\sigma_{\phi \phi}
=
-p + 2 \mu
\left(
\inv{r}
\PD{\phi}{u_\phi} + \frac{u_r}{r}
\right),
\end{equation}
\begin{equation}\label{eqn:stressTensorVectorForm:730}
\sigma_{\phi z}
=
\mu \left(
\frac{1}{r} \PD{\phi}{u_z}
    + \PD{z}{u_\phi}
\right),
\end{equation}
\begin{equation}\label{eqn:stressTensorVectorForm:750}
\sigma_{\phi r}
=
\mu \left(
\frac{1}{r}\PD{\phi}{u_r}
-\frac{u_\phi}{r}
+\PD{r}{u_\phi}
\right).
\end{equation}
\end{subequations}
\subsection{Longitudinal normal \texorpdfstring{\(\ncap = \zcap\)}{ncap equals zcap}.}
Like the \(\rcap\) normal direction, our directional derivative component for a \(\zcap\) normal direction will not have any cross terms
%
\begin{equation}\label{eqn:stressTensorVectorForm:1170}
\begin{aligned}
2 (\zcap \cdot \spacegrad) \Bu
&=
\partial_z
\left(
\rcap u_r + \phicap u_\phi + \zcap u_z
\right) \\
&=
\rcap \partial_z u_r + \phicap \partial_z u_\phi + \zcap \partial_z u_z.
\end{aligned}
\end{equation}
Projecting our curl bivector onto the \(\zcap\) direction we have
%
\begin{equation}\label{eqn:stressTensorVectorForm:1190}
\begin{aligned}
(\spacegrad \wedge \Bu) \cdot \phicap
&=
(\rcap \wedge \phicap) \cdot \zcap
\left(
\partial_r u_\phi
-\inv{r} \partial_\phi u_r
+ \frac{u_\phi}{r}
\right) \\
&+
(\phicap \wedge \zcap) \cdot \zcap
\left(
\inv{r} \partial_\phi u_z
- \partial_z u_\phi
\right) \\
&+
(\zcap \wedge \rcap) \cdot \zcap
\left(
\partial_z u_r - \partial_r u_z
\right) \\
&=
\phicap
\left(
\inv{r} \partial_\phi u_z
- \partial_z u_\phi
\right)
-
\rcap
\left(
\partial_z u_r - \partial_r u_z
\right).
\end{aligned}
\end{equation}
Putting things together we have
%
\begin{equation}\label{eqn:stressTensorVectorForm:1210}
\begin{aligned}
2 \Be_{\zcap}
&=
2 \rcap \partial_z u_r + 2 \phicap \partial_z u_\phi + 2 \zcap \partial_z u_z
+
\phicap
\left(
\inv{r} \partial_\phi u_z
- \partial_z u_\phi
\right)
-
\rcap
\left(
\partial_z u_r - \partial_r u_z
\right) \\
&=
\rcap
\left(
2 \partial_z u_r
-\partial_z u_r + \partial_r u_z
\right)
+
\phicap
\left(
2 \partial_z u_\phi
+
\inv{r} \partial_\phi u_z
- \partial_z u_\phi
\right)
+
\zcap
\left(
2 \partial_z u_z
\right) \\
&=
\rcap
\left(
\partial_z u_r
+ \partial_r u_z
\right)
+
\phicap
\left(
\partial_z u_\phi
+
\inv{r} \partial_\phi u_z
\right)
+
\zcap
\left(
2 \partial_z u_z
\right).
\end{aligned}
\end{equation}
For our stress tensor
%
\begin{equation}\label{eqn:stressTensorVectorForm:770}
\Bsigma_{\zcap} = - p \zcap + 2 \mu e_{\zcap},
\end{equation}
we can now read off our components by taking dot products to yield
%
\begin{subequations}
\begin{equation}\label{eqn:stressTensorVectorForm:790}
\sigma_{z z}
=
-p + 2 \mu
\PD{z}{u_z},
\end{equation}
\begin{equation}\label{eqn:stressTensorVectorForm:810}
\sigma_{z r}
=
\mu \left(
\PD{z}{u_r}
+ \PD{r}{u_z}
\right),
\end{equation}
\begin{equation}\label{eqn:stressTensorVectorForm:830}
\sigma_{z \phi}
=
\mu \left(
\PD{z}{u_\phi}
+
\inv{r} \PD{\phi}{u_z}
\right).
\end{equation}
\end{subequations}
\subsection{Summary.}
%
\begin{subequations}
\begin{equation}\label{eqn:stressTensorVectorForm:850}
\sigma_{rr}
=
-p + 2 \mu \PD{r}{u_r},
\end{equation}
\begin{equation}\label{eqn:stressTensorVectorForm:870}
\sigma_{\phi \phi}
=
-p + 2 \mu
\left(
\inv{r}
\PD{\phi}{u_\phi} + \frac{u_r}{r}
\right),
\end{equation}
\begin{equation}\label{eqn:stressTensorVectorForm:890}
\sigma_{z z}
=
-p + 2 \mu
\PD{z}{u_z},
\end{equation}
\begin{equation}\label{eqn:stressTensorVectorForm:910}
\sigma_{r \phi}
=
\mu \left(
 \PD{r}{u_\phi}
+\inv{r} \PD{\phi}{u_r}
- \frac{u_\phi}{r}
\right),
\end{equation}
\begin{equation}\label{eqn:stressTensorVectorForm:930}
\sigma_{\phi z}
=
\mu \left(
\frac{1}{r} \PD{\phi}{u_z}
    + \PD{z}{u_\phi}
\right),
\end{equation}
\begin{equation}\label{eqn:stressTensorVectorForm:950}
\sigma_{z r}
=
\mu \left(
\PD{z}{u_r}
+ \PD{r}{u_z}
\right).
\end{equation}
\end{subequations}
\section{Spherical strain tensor.}
Having done a first order cylindrical derivation of the strain tensor, let us also do the spherical case for completeness.  Would this have much utility in fluids?  Perhaps for flow over a spherical barrier?
We need the gradient in spherical coordinates.  Recall that our spherical coordinate velocity was
%
\begin{equation}\label{eqn:stressTensorVectorForm:90}
\frac{d\Br}{dt} = \rcap \rdot + \thetacap (r \thetadot) + \phicap ( r \sin\theta \phidot ),
\end{equation}
and our gradient mirrors this structure
%
\begin{equation}\label{eqn:stressTensorVectorForm:110}
\spacegrad = \rcap \PD{r}{} + \thetacap \inv{r }\PD{\theta}{} + \phicap \inv{r \sin\theta} \PD{\phi}{}.
\end{equation}
%We also previously calculated \inbookref{phy454:continuumL2}{eqn:continuumL2:1010} the unit vector differentials
Referring back to \cref{eqn:continuumL2:1010} where we noted that the unit vector differentials were
%
\begin{subequations}
\begin{equation}\label{eqn:stressTensorVectorForm:130}
d\rcap = \phicap \sin\theta d\phi + \thetacap d\theta,
\end{equation}
\begin{equation}\label{eqn:stressTensorVectorForm:150}
d\thetacap = \phicap \cos\theta d\phi - \rcap d\theta,
\end{equation}
\begin{equation}\label{eqn:stressTensorVectorForm:170}
d\phicap = -(\rcap \sin\theta + \thetacap \cos\theta) d\phi,
\end{equation}
\end{subequations}
and can use those to read off the partials of all the unit vectors
%
\begin{equation}\label{eqn:stressTensorVectorForm:190}
\begin{aligned}
\frac{\partial \rcap}{\partial \{r,\theta, \phi\}} &= \{0, \thetacap, \phicap \sin\theta \} \\
\frac{\partial \thetacap}{\partial \{r,\theta, \phi\}} &= \{0, -\rcap, \phicap \cos\theta \} \\
\frac{\partial \phicap}{\partial \{r,\theta, \phi\}} &= \{0, 0, -\rcap \sin\theta -\thetacap \cos\theta \}.
\end{aligned}
\end{equation}
Finally, our velocity in spherical coordinates is just
%
\begin{equation}\label{eqn:stressTensorVectorForm:210}
\Bu = \rcap u_r + \thetacap u_\theta + \phicap u_\phi,
\end{equation}
from which we can now compute the curl, and the directional derivative.  Starting with the curl we have
%
\begin{equation}\label{eqn:stressTensorVectorForm:1230}
\begin{aligned}
\spacegrad \wedge \Bu
&=
\left( \rcap \PD{r}{} + \thetacap \inv{r }\PD{\theta}{} + \phicap \inv{r \sin\theta} \PD{\phi}{} \right) \wedge
\left( \rcap u_r + \thetacap u_\theta + \phicap u_\phi \right) \\
&=
\rcap \wedge \thetacap
\left( \partial_r u_\theta - \inv{r} \partial_\theta u_r
\right)
\\
& +
\thetacap \wedge \phicap
\left(
\inv{r} \partial_\theta u_\phi - \inv{r \sin\theta} \partial_\phi u_\theta
\right)
\\
& +
\phicap \wedge \rcap
\left(
\inv{r \sin\theta} \partial_\phi u_r - \partial_r u_\phi
\right)
\\
& +
\inv{r} \thetacap \wedge \left(
u_\theta \mathLabelBox{\partial_\theta \thetacap}{\(-\rcap\)}
+
u_\phi \mathLabelBox{\partial_\theta \phicap}{0}
\right)
\\
& +
\inv{r \sin\theta} \phicap \wedge \left(
u_\theta \mathLabelBox{\partial_\phi \thetacap}{\(\phicap \cos\theta\)}
+
u_\phi
\mathLabelBox
[
   labelstyle={below of=m\themathLableNode, below of=m\themathLableNode}
]
{\partial_\phi \phicap}{\(-\rcap \sin\theta - \thetacap \cos\theta\)}
\right),
\end{aligned}
\end{equation}
so we have
%
\begin{equation}\label{eqn:stressTensorVectorForm:230}
\begin{aligned}
\spacegrad \wedge \Bu
&=
\rcap \wedge \thetacap
\lr{ \partial_r u_\theta - \inv{r} \partial_\theta u_r + \frac{u_\theta}{r} } \\
&\quad +
\thetacap \wedge \phicap
\lr{
\inv{r} \partial_\theta u_\phi - \inv{r \sin\theta} \partial_\phi u_\theta
+ \frac{u_\phi \cot\theta}{r}
} \\
&\quad +
\phicap \wedge \rcap
\lr{
\inv{r \sin\theta} \partial_\phi u_r - \partial_r u_\phi
- \frac{u_\phi}{r}
}
.
\end{aligned}
\end{equation}
\subsection{Outwards radial normal \texorpdfstring{\(\ncap = \rcap\)}{ncap equals rcap}.}
The directional derivative portion of our strain is
%
\begin{equation}\label{eqn:stressTensorVectorForm:1250}
\begin{aligned}
2 (\rcap \cdot \spacegrad) \Bu
&=
2 \partial_r (
\rcap u_r + \thetacap u_\theta + \phicap u_\phi ) \\
&=
2 (
\rcap \partial_r u_r + \thetacap \partial_r u_\theta + \phicap \partial_r u_\phi ).
\end{aligned}
\end{equation}
The other portion of our strain tensor is
%
\begin{equation}\label{eqn:stressTensorVectorForm:1270}
\begin{aligned}
(\spacegrad \wedge \Bu) \cdot \rcap
&=
(\rcap \wedge \thetacap) \cdot \rcap
\left( \partial_r u_\theta - \inv{r} \partial_\theta u_r + \frac{u_\theta}{r}
\right)
\\
& +
(\thetacap \wedge \phicap) \cdot \rcap
\left(
\inv{r} \partial_\theta u_\phi - \inv{r \sin\theta} \partial_\phi u_\theta
+ \frac{u_\phi \cot\theta}{r}
\right)
\\
& +
(\phicap \wedge \rcap) \cdot \rcap
\left(
\inv{r \sin\theta} \partial_\phi u_r - \partial_r u_\phi
- \frac{u_\phi}{r}
\right) \\
&=
-\thetacap
\left( \partial_r u_\theta - \inv{r} \partial_\theta u_r + \frac{u_\theta}{r}
\right)
\\
& +
\phicap
\left(
\inv{r \sin\theta} \partial_\phi u_r - \partial_r u_\phi
- \frac{u_\phi}{r}
\right).
\end{aligned}
\end{equation}
Putting these together we find
%
\begin{equation}\label{eqn:stressTensorVectorForm:1290}
\begin{aligned}
2 {\Be}_{\rcap}
&=
2 (\rcap \cdot \spacegrad)\Bu + (\spacegrad \wedge \Bu) \cdot \rcap \\
&=
2 (
\rcap \partial_r u_r + \thetacap \partial_r u_\theta + \phicap \partial_r u_\phi )
-\thetacap
\left(
\partial_r u_\theta - \inv{r} \partial_\theta u_r + \frac{u_\theta}{r}
\right) \\
&\quad +
\phicap
\left(
\inv{r \sin\theta} \partial_\phi u_r - \partial_r u_\phi
- \frac{u_\phi}{r}
\right) \\
&=
\rcap
\left(
2 \partial_r u_r
\right)
+
\thetacap
\left(
2 \partial_r u_\theta
-\partial_r u_\theta + \inv{r} \partial_\theta u_r - \frac{u_\theta}{r}
\right) \\
&\quad +
\phicap
\left(
2 \partial_r u_\phi
+ \inv{r \sin\theta} \partial_\phi u_r - \partial_r u_\phi
- \frac{u_\phi}{r}
\right),
\end{aligned}
\end{equation}
which gives
%
\begin{equation}\label{eqn:stressTensorVectorForm:270}
\begin{aligned}
2 {\Be}_{\rcap}
&=
\rcap
\left(
2 \partial_r u_r
\right)
+
\thetacap
\left(
\partial_r u_\theta
+ \inv{r} \partial_\theta u_r - \frac{u_\theta}{r}
\right) \\
&+
\phicap
\left(
\partial_r u_\phi
+ \inv{r \sin\theta} \partial_\phi u_r
- \frac{u_\phi}{r}
\right).
\end{aligned}
\end{equation}
For our stress tensor
%
\begin{equation}\label{eqn:stressTensorVectorForm:290b}
\Bsigma_{\rcap} = - p \rcap + 2 \mu e_{\rcap}.
\end{equation}
We can now read off our components by taking dot products
%
\begin{subequations}
\begin{equation}\label{eqn:stressTensorVectorForm:310}
\sigma_{rr}
=
-p + 2 \mu \PD{r}{u_r},
\end{equation}
\begin{equation}\label{eqn:stressTensorVectorForm:330}
\sigma_{r \theta}
=
\mu \left(
\PD{r}{u_\theta}
+ \inv{r} \PD{\theta}{u_r} - \frac{u_\theta}{r}
\right),
\end{equation}
\begin{equation}\label{eqn:stressTensorVectorForm:350}
\sigma_{r \phi}
=
\mu \left(
\PD{r}{u_\phi}
+ \inv{r \sin\theta} \PD{\phi}{u_r}
- \frac{u_\phi}{r}
\right).
\end{equation}
\end{subequations}
This is consistent with (15.20) from \citep{landau1987course} (after adjusting for minor notational differences).
\subsection{Polar normal \texorpdfstring{\(\ncap = \thetacap\)}{ncap equals thetacap}.}
Now let us do the \(\thetacap\) direction.  The directional derivative portion of our strain will be a bit more work to compute because we have \(\theta\) variation of the unit vectors
\begin{equation}\label{eqn:stressTensorVectorForm:1310}
\begin{aligned}
&(\thetacap \cdot \spacegrad) \Bu \\
&=
\inv{r} \partial_\theta (
\rcap u_r + \thetacap u_\theta + \phicap u_\phi
 ) \\
&=
\inv{r} \left( \rcap \partial_\theta u_r + \thetacap \partial_\theta u_\theta + \phicap \partial_\theta u_\phi \right)
+
\inv{r} \left( (\partial_\theta \rcap) u_r + (\partial_\theta \thetacap) u_\theta + (\partial_\theta \phicap) u_\phi \right) \\
&=
\inv{r}\left(
\rcap \partial_\theta u_r + \thetacap \partial_\theta u_\theta + \phicap \partial_\theta u_\phi  \right)
+
\inv{r} \left( \thetacap u_r - \rcap u_\theta  \right),
\end{aligned}
\end{equation}
so we have
\begin{equation}\label{eqn:stressTensorVectorForm:370}
2 (\thetacap \cdot \spacegrad) \Bu
=
\frac{2}{r} \rcap (\partial_\theta u_r
- u_\theta
)
+ \frac{2}{r} \thetacap (\partial_\theta u_\theta
+ u_r
) + \frac{2}{r} \phicap \partial_\theta u_\phi,
\end{equation}
and can move on to projecting our curl bivector onto the \(\thetacap\) direction.  That portion of our strain tensor is
\begin{equation}\label{eqn:stressTensorVectorForm:1330}
\begin{aligned}
(\spacegrad &\wedge \Bu) \cdot \thetacap \\
&=
(\rcap \wedge \thetacap) \cdot \thetacap
\left( \partial_r u_\theta - \inv{r} \partial_\theta u_r + \frac{u_\theta}{r}
\right)
\\
&\qquad +
(\thetacap \wedge \phicap) \cdot \thetacap
\left(
\inv{r} \partial_\theta u_\phi - \inv{r \sin\theta} \partial_\phi u_\theta
+ \frac{u_\phi \cot\theta}{r}
\right)
\\
&\qquad +
(\phicap \wedge \rcap) \cdot \thetacap
\left(
\inv{r \sin\theta} \partial_\phi u_r - \partial_r u_\phi
- \frac{u_\phi}{r}
\right) \\
&=
\rcap
\left( \partial_r u_\theta - \inv{r} \partial_\theta u_r + \frac{u_\theta}{r}
\right)
-\phicap
\left(
\inv{r} \partial_\theta u_\phi - \inv{r \sin\theta} \partial_\phi u_\theta
+ \frac{u_\phi \cot\theta}{r}
\right).
\end{aligned}
\end{equation}
Putting these together we find
\begin{equation}\label{eqn:stressTensorVectorForm:1350}
\begin{aligned}
2 {\Be}_{\thetacap}
&=
2 (\thetacap \cdot \spacegrad)\Bu + (\spacegrad \wedge \Bu) \cdot \thetacap \\
&=
  \frac{2}{r} \rcap (\partial_\theta u_r - u_\theta )
+ \frac{2}{r} \thetacap (\partial_\theta u_\theta + u_r )
+ \frac{2}{r} \phicap \partial_\theta u_\phi \\
&+\rcap
\left(
\partial_r u_\theta - \inv{r} \partial_\theta u_r + \frac{u_\theta}{r}
\right) \\
&\quad -\phicap
\left(
   \inv{r} \partial_\theta u_\phi - \inv{r \sin\theta} \partial_\phi u_\theta + \frac{u_\phi \cot\theta}{r}
\right).
\end{aligned}
\end{equation}
Which gives
\begin{equation}\label{eqn:stressTensorVectorForm:390}
\begin{aligned}
2 {\Be}_{\thetacap}
&=
\rcap \lr{
  \frac{1}{r} \partial_\theta u_r
 + \partial_r u_\theta
- \frac{u_\theta}{r}
} \\
&\quad +
\thetacap \lr{
 \frac{2}{r} \partial_\theta u_\theta
+ \frac{2}{r} u_r
} \\
&\quad +
\phicap \lr{
\frac{1}{r} \partial_\theta u_\phi
+ \inv{r \sin\theta} \partial_\phi u_\theta
- \frac{u_\phi \cot\theta}{r}
}
.
\end{aligned}
\end{equation}
For our stress tensor
%
\begin{equation}\label{eqn:stressTensorVectorForm:410}
\Bsigma_{\thetacap} = - p \thetacap + 2 \mu e_{\thetacap},
\end{equation}
so we can now read off our components by taking dot products
%
\begin{subequations}
\begin{equation}\label{eqn:stressTensorVectorForm:430}
\sigma_{\theta \theta}
=
-p
+
\mu \left(
 \frac{2}{r} \PD{\theta}{u_\theta}
+ \frac{2}{r} u_r
\right),
\end{equation}
\begin{equation}\label{eqn:stressTensorVectorForm:450}
\sigma_{\theta \phi}
=
\mu \left(
\frac{1}{r} \PD{\theta}{u_\phi}
+ \inv{r \sin\theta} \PD{\phi}{u_\theta}
- \frac{u_\phi \cot\theta}{r}
\right),
\end{equation}
\begin{equation}\label{eqn:stressTensorVectorForm:470}
\sigma_{\theta r}
= \mu \left(
\frac{1}{r} \PD{\theta}{u_r} + \PD{r}{u_\theta}
- \frac{u_\theta}{r}
\right).
\end{equation}
\end{subequations}
This again is consistent with (15.20) from \citep{landau1987course}.
\subsection{Azimuthal normal \texorpdfstring{\(\ncap = \phicap\)}{ncap equals phicap}.}
Finally, let us do the \(\phicap\) direction.  This directional derivative portion of our strain will also be a bit more work to compute because we have \(\phicap\) variation of the unit vectors
%
\begin{equation}\label{eqn:stressTensorVectorForm:1370}
\begin{aligned}
(\phicap &\cdot \spacegrad) \Bu \\
&=
\inv{r \sin\theta} \partial_\phi
(
\rcap u_r + \thetacap u_\theta + \phicap u_\phi
) \\
&=
\inv{r \sin\theta}
(
\rcap
\partial_\phi
u_r
+
\thetacap
\partial_\phi
u_\theta
+
\phicap
\partial_\phi
u_\phi
+
(\partial_\phi
\rcap )
u_r
+
(\partial_\phi
\thetacap )
u_\theta
+
(\partial_\phi
\phicap )
u_\phi
) \\
&=
\inv{r \sin\theta}
\biglr{
\rcap
\partial_\phi
u_r
+
\thetacap
\partial_\phi
u_\theta \\
&\qquad +
\phicap
\partial_\phi
u_\phi
+
\phicap \sin\theta
u_r
+
\phicap \cos\theta
u_\theta \\
&\qquad-
(
\rcap \sin\theta
+ \thetacap \cos\theta
)
u_\phi
},
\end{aligned}
\end{equation}
so we have
\begin{equation}\label{eqn:stressTensorVectorForm:490}
\begin{aligned}
2 (\phicap \cdot \spacegrad) \Bu
&=
2 \rcap
\lr{
   \inv{r \sin\theta} \partial_\phi u_r - \frac{u_\phi}{r}
} \\
&\quad+
2 \thetacap
\lr{
   \inv{r \sin\theta} \partial_\phi u_\theta
   -
   \inv{r} \cot\theta u_\phi
} \\
&\quad +
2 \phicap
\lr{
   \inv{r \sin\theta} \partial_\phi u_\phi
   + \inv{r} u_r
   + \inv{r} \cot\theta u_\theta
}
,
\end{aligned}
\end{equation}
and can move on to projecting our curl bivector onto the \(\phicap\) direction.  That portion of our strain tensor is
\begin{equation}\label{eqn:stressTensorVectorForm:1390}
\begin{aligned}
(\spacegrad \wedge \Bu) \cdot \phicap
&=
(\rcap \wedge \thetacap) \cdot \phicap
\left( \partial_r u_\theta - \inv{r} \partial_\theta u_r + \frac{u_\theta}{r}
\right)
\\
& +
(\thetacap \wedge \phicap) \cdot \phicap
\left(
\inv{r} \partial_\theta u_\phi - \inv{r \sin\theta} \partial_\phi u_\theta
+ \frac{u_\phi \cot\theta}{r}
\right)
\\
& +
(\phicap \wedge \rcap) \cdot \phicap
\left(
\inv{r \sin\theta} \partial_\phi u_r - \partial_r u_\phi
- \frac{u_\phi}{r}
\right) \\
&=
\thetacap
\left(
\inv{r} \partial_\theta u_\phi - \inv{r \sin\theta} \partial_\phi u_\theta
+ \frac{u_\phi \cot\theta}{r}
\right)
\\
&
-\rcap
\left(
\inv{r \sin\theta} \partial_\phi u_r - \partial_r u_\phi
- \frac{u_\phi}{r}
\right).
\end{aligned}
\end{equation}
Putting these together we find
%
\begin{equation}\label{eqn:stressTensorVectorForm:1410}
\begin{aligned}
2 &{\Be}_{\thetacap} \\
&=
2 (\phicap \cdot \spacegrad)\Bu + (\spacegrad \wedge \Bu) \cdot \phicap \\
&=
2 \rcap
\left(
\inv{r \sin\theta} \partial_\phi u_r - \frac{u_\phi}{r}
\right) 
+
2 \thetacap
\left(
\inv{r \sin\theta} \partial_\phi u_\theta
-
\inv{r} \cot\theta u_\phi
\right) \\
&\quad +
2 \phicap
\left(
\inv{r \sin\theta} \partial_\phi u_\phi
+ \inv{r} u_r
+ \inv{r} \cot\theta u_\theta
\right) \\
&\quad+
\thetacap
\left(
\inv{r} \partial_\theta u_\phi - \inv{r \sin\theta} \partial_\phi u_\theta
+ \frac{u_\phi \cot\theta}{r}
\right) \\
&\quad
-\rcap
\left(
\inv{r \sin\theta} \partial_\phi u_r - \partial_r u_\phi
- \frac{u_\phi}{r}
\right),
\end{aligned}
\end{equation}
which gives
%
\begin{equation}\label{eqn:stressTensorVectorForm:510}
\begin{aligned}
2 {\Be}_{\phicap}
&=
\rcap \left(
 \frac{ \partial_\phi u_r }{r \sin\theta}
- \frac{u_\phi}{r}
+ \partial_r u_\phi
\right)
+
\thetacap \left(
\frac{
\partial_\phi u_\theta
}{r \sin\theta}
- \frac{u_\phi \cot\theta}{r}
+\frac{
\partial_\theta u_\phi
}{r}
\right) \\
&+
2 \phicap \left(
\frac{
\partial_\phi u_\phi
}{r \sin\theta}
+ \frac{
u_r
}{r}
+ \frac{
\cot\theta u_\theta
}{r}
\right).
\end{aligned}
\end{equation}
For our stress tensor
%
\begin{equation}\label{eqn:stressTensorVectorForm:530}
\Bsigma_{\phicap} = - p \phicap + 2 \mu e_{\phicap},
\end{equation}
so we can now read off our components by taking dot products
%
\begin{subequations}
\begin{equation}\label{eqn:stressTensorVectorForm:550}
\sigma_{\phi \phi}
=
-p
+
2 \mu \left(
\inv{r \sin\theta} \PD{\phi}{u_\phi}
+ \frac{
u_r
}{r}
+ \frac{
\cot\theta u_\theta
}{r}
\right),
\end{equation}
\begin{equation}\label{eqn:stressTensorVectorForm:570}
\sigma_{\phi r}
=
\mu \left(
  \frac{1}{r \sin\theta} \PD{\phi}{u_r}
- \frac{u_\phi}{r}
+ \PD{r}{u_\phi}
\right),
\end{equation}
\begin{equation}\label{eqn:stressTensorVectorForm:590}
\sigma_{\phi \theta}
= \mu \left(
\frac{1}{r \sin\theta} \PD{\phi}{u_\theta}
- \frac{u_\phi \cot\theta}{r}
+\inv{r} \PD{\theta}{u_\phi}
\right).
\end{equation}
\end{subequations}
This again is consistent with (15.20) from \citep{landau1987course}.
\subsection{Summary.}
%
\begin{subequations}
\begin{equation}\label{eqn:stressTensorVectorForm:310b}
\sigma_{rr}
=
-p + 2 \mu \PD{r}{u_r},
\end{equation}
\begin{equation}\label{eqn:stressTensorVectorForm:430b}
\sigma_{\theta \theta}
=
-p
+
2 \mu \left(
 \frac{1}{r} \PD{\theta}{u_\theta}
+ \frac{ u_r }{r}
\right),
\end{equation}
\begin{equation}\label{eqn:stressTensorVectorForm:550b}
\sigma_{\phi \phi}
=
-p
+
2 \mu \left(
\inv{r \sin\theta} \PD{\phi}{u_\phi}
+ \frac{
u_r
}{r}
+ \frac{
\cot\theta u_\theta
}{r}
\right),
\end{equation}
\begin{equation}\label{eqn:stressTensorVectorForm:330b}
\sigma_{r \theta}
=
\mu \left(
\PD{r}{u_\theta}
+ \inv{r} \PD{\theta}{u_r} - \frac{u_\theta}{r}
\right)
\end{equation}
\begin{equation}\label{eqn:stressTensorVectorForm:590b}
\sigma_{\theta \phi}
= \mu \left(
\frac{1}{r \sin\theta} \PD{\phi}{u_\theta}
- \frac{u_\phi \cot\theta}{r}
+\inv{r} \PD{\theta}{u_\phi}
\right),
\end{equation}
\begin{equation}\label{eqn:stressTensorVectorForm:570b}
\sigma_{\phi r}
=
\mu \left(
  \frac{1}{r \sin\theta} \PD{\phi}{u_r}
- \frac{u_\phi}{r}
+ \PD{r}{u_\phi}
\right).
\end{equation}
\end{subequations}
