%
% Copyright � 2012 Peeter Joot.  All Rights Reserved.
% Licenced as described in the file LICENSE under the root directory of this GIT repository.
%
\label{chap:fourierSeries}
\section{A Fourier series refresher}

Here is a quick re-derivation of how to obtain the Fourier coefficients \index{Fourier coefficient} for a trigonometric Fourier series in exponential form.  This is performed over an arbitrary interval to make it easy to apply to more specific problems.

%I had used the wrong scaling in a Fourier series over a \([0, 1]\) interval.  Here is a reminder to self what the right way to do this is.

Suppose we have a function that is defined in terms of a trigonometric Fourier sum
%
\begin{equation}\label{eqn:fourierSeries:10}
\phi(x) = \sum c_k e^{i \omega k x},
\end{equation}
%
where the domain of interest is \(x \in [a, b]\).  Stating the problem this way avoids any issue of existence.  We know \(c_k\) exists, but just want to find what they are given some other representation of the function.

Multiplying and integrating over our domain we have
%
%
\begin{dmath}\label{eqn:fourierSeries:30}
\int_a^b \phi(x) e^{-i \omega m x} dx
= \sum c_k \int_a^b e^{i \omega (k -m) x} dx
= c_m (b - a) + \sum_{k \ne m} \frac{e^{i \omega(k-m) b} - e^{i \omega(k-m)a}}{i \omega (k -m)} .
\end{dmath}
%
We want all the terms in the sum to be be zero, requiring equality of the exponentials, or
%
\begin{equation}\label{eqn:fourierSeries:50}
e^{i \omega (k -m) (b -a )} = 1,
\end{equation}
%
or
%
\begin{equation}\label{eqn:fourierSeries:70}
\omega = \frac{2 \pi}{b - a}.
\end{equation}
%
This fixes our Fourier coefficients
%
\begin{equation}\label{eqn:fourierSeries:90}
c_m = \inv{b - a} \int_a^b \phi(x) e^{- 2 \pi i m x/(b - a)} dx.
\end{equation}
%
So, for example, if we wished for the correct (but unnormalized) Fourier basis for a \([0, 1]\) interval, we see that we use the functions \(e^{2 \pi i x}\), or the sine and cosine equivalents, as our basis elements.
