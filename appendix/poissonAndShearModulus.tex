%
% Copyright � 2012 Peeter Joot.  All Rights Reserved.
% Licenced as described in the file LICENSE under the root directory of this GIT repository.
%
\label{chap:appendix:poissonAndShearModulus}

Young's modulus is given in \eqnref{eqn:continuumL5:330} (equation (43) in the Professor's notes) as

\begin{equation}\label{eqn:continuumL6:490}
E = \frac{\mu(3 \lambda + 2 \mu)}{\lambda + \mu },
\end{equation}

and for Poisson's ratio \eqnref{eqn:continuumL5:410} (equation (46) in the Professor's notes) we have

\begin{equation}\label{eqn:continuumL6:510}
\nu = -\frac{e_{22}}{e_{11}} = \frac{\lambda}{2 (\lambda + \mu)}.
\end{equation}

%(these are consistent with what I have got above).

Let us derive the other stated relationships (equation (47) in the Professor's notes).  I get

\begin{equation}\label{eqn:poissonAndShearModulus:550}
\begin{aligned}
2 (\lambda + \mu) \nu = \lambda \\
\implies \\
\lambda ( 2 \nu - 1 ) = - 2\mu\nu
\end{aligned}
\end{equation}

or

\begin{equation}\label{eqn:poissonAndShearModulus:570}
\begin{aligned}
\lambda = \frac{ 2 \mu \nu} { 1 - 2 \nu }
\end{aligned}
\end{equation}

For substitution into the Young's modulus equation calculate

\begin{equation}\label{eqn:poissonAndShearModulus:590}
\begin{aligned}
\lambda + \mu
&= \frac{ 2 \mu \nu} { 1 - 2 \nu } + \mu \\
&= \mu \left( \frac{ 2 \nu} { 1 - 2 \nu } + 1 \right)  \\
&= \mu \frac{ 2 \nu + 1 - 2 \nu} { 1 - 2 \nu }  \\
&= \frac{ \mu} { 1 - 2 \nu }  \\
\end{aligned}
\end{equation}

and

\begin{equation}\label{eqn:poissonAndShearModulus:610}
\begin{aligned}
3 \lambda + 2 \mu
&= 3 \frac{ \mu} { 1 - 2 \nu } - \mu \\
&= \mu \frac{ 3 - (1 - 2 \nu)} { 1 - 2 \nu } \\
&= \mu \frac{ 2 + 2 \nu} { 1 - 2 \nu } \\
&= 2 \mu \frac{ 1 + \nu} { 1 - 2 \nu } \\
\end{aligned}
\end{equation}

Putting these together we find

\begin{equation}\label{eqn:poissonAndShearModulus:630}
\begin{aligned}
E
&= \frac{\mu(3 \lambda + 2 \mu)}{\lambda + \mu } \\
&= \mu 2 \mu \frac{ 1 + \nu} { 1 - 2 \nu } \frac{ 1 - 2 \nu}{\mu} \\
&= 2 \mu ( 1 + \nu ) \\
\end{aligned}
\end{equation}

Rearranging we have

\begin{equation}\label{eqn:continuumL6:530}
\mu = \frac{E}{2 (1 + \nu)}.
\end{equation}

This matches (5.9) in the text (where \(\sigma\) is used instead of \(\nu\)).
%, but does not match your equation (47).

We also find

\begin{equation}\label{eqn:poissonAndShearModulus:650}
\begin{aligned}
\lambda
&= \frac{ 2 \mu \nu} { 1 - 2 \nu } \\
&= \frac{ \nu} { 1 - 2 \nu } \frac{E }{1 + \nu}.
\end{aligned}
\end{equation}

%(also different than the Prof's notes).
