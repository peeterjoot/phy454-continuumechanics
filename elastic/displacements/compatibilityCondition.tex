%
% Copyright � 2012 Peeter Joot.  All Rights Reserved.
% Licenced as described in the file LICENSE under the root directory of this GIT repository.
%

%
%
%\chapter{PHY454H1S\\Continuum Mechanics.  Lecture 6: Compatibility condition and elastostatics.  Taught by Prof. K. Das}
\section{Elastic waves.}
\index{elastic wave}
%
Reading: Chapter I \S 7, chapter III (\S 22 - \S 26) of the text \citep{landau1960theory}.

Example: sound or water waves (i.e. waves in a solid or liquid material that comes back to its original position.)
%
\makedefinition{Elastic Wave}{dfn:continuumL6:10}{An \textAndIndex{elastic wave} is a type of mechanical wave that propagates through or on the surface of a medium.  The elasticity of the material provides the restoring force (that returns the material to its original state).  The displacement and the restoring force are assumed to be linearly related.}

In symbols we say
%
\begin{equation}\label{eqn:continuumL6:130}
e_i(x_j, t) \quad \mbox{related to force},
\end{equation}
%
and specifically
%
\begin{equation}\label{eqn:continuumL6:150}
\rho \PDSq{t}{e_i} = F_i = \PD{x_j}{\sigma_{ij}}.
\end{equation}
%
This is just Newton's second law, \(F = ma\), but expressed in terms of a unit volume.

Should we have an external body force (per unit volume) \(f_i\) acting on the body then we must modify this, writing
%
\boxedEquation{eqn:continuumL6:170}{
\rho \PDSq{t}{e_i} = \PD{x_j}{\sigma_{ij}} + f_i.
}
%
Note that we are separating out the ``original'' forces that produced the stress and strain on the object from any constant external forces that act on the body (i.e. a gravitational field).

With
%
\begin{equation}\label{eqn:continuumL6:190}
e_{ij} =
\inv{2} \left(
\PD{x_j}{e_i}
+ \PD{x_i}{e_j} \right),
\end{equation}
%
we can expand the stress divergence, for the case of homogeneous deformation, in terms of the Lam\'e parameters
%
\begin{equation}\label{eqn:continuumL6:210}
\sigma_{ij} = \lambda e_{kk} \delta_{ij} + 2 \mu e_{ij}.
\end{equation}
%
We compute
%
\begin{equation}\label{eqn:compatibilityCondition:310}
\begin{aligned}
\PD{x_j}{\sigma_{ij}}
&=
\lambda
\PD{x_j}{
e_{kk}
}
\delta_{ij} + 2 \mu
\PD{x_j}{
}
\inv{2} \left(
\PD{x_j}{e_i}
+ \PD{x_i}{e_j} \right),
 \\
&=
\lambda
\PD{x_i}{
e_{kk}
}
+ \mu
\left(
\PDSq{x_j}{
e_{i}
}
+
\frac{\partial^2 e_{j} }{ \partial x_j \partial x_i}
\right) \\
&=
%\sum_k
\lambda
\PD{x_i}{
}
\PD{x_k}{e_k}
+ \mu
\left(
\PDSq{x_j}{
e_{i}
}
+
\frac{\partial^2 e_{k} }{ \partial x_k \partial x_i}
\right) \\
&=
(\lambda + \mu)
\PD{x_i}{
}
\PD{x_k}{e_k}
+ \mu
\PDSq{x_j}{
e_{i}
}.
\end{aligned}
\end{equation}
%
%With
%
%\begin{equation}\label{eqn:continuumL6:230}
%e_{kk} = e_{11} +e_{22} +e_{33}
%=
%\PD{x_1}{e_1}
%+\PD{x_2}{e_2}
%+\PD{x_3}{e_3}
%\end{equation}
%
We find, for homogeneous deformations, that the force per unit volume on our element of mass, in the absence of external forces (the body forces), takes the form
%
%\begin{equation}\label{eqn:continuumL6:250}
%\PD{x_j}{\sigma_{ij}} = (\lambda + \mu) \frac{\partial^2 e_j}{\partial x_i \partial x_j}
%+ \mu
%\frac{\partial^2 e_i}
%{\partial x_j^2
%}
%\end{equation}
%
\begin{equation}\label{eqn:continuumL6:270}
\rho \PDSq{t}{e_i} = (\lambda + \mu) \frac{\partial^2 e_k}{\partial x_i \partial x_k}
+ \mu
\frac{\partial^2 e_i}
{\partial x_j^2
}.
\end{equation}
%
This can be seen to be equivalent to the vector relationship
%
\boxedEquation{eqn:continuumL6:290}{
\rho \PDSq{t}{\Be} = (\lambda + \mu) \spacegrad (\spacegrad \cdot \Be) + \mu \spacegrad^2 \Be.
}
%
\FIXME{What form do the stress and strain tensors take in vector form?}
