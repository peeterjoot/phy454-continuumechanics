%
% Copyright � 2012 Peeter Joot.  All Rights Reserved.
% Licenced as described in the file LICENSE under the root directory of this GIT repository.
%

%
%
\section{Summary}

\subsection{Elastic displacement equation} \index{elastic displacement}
It was argued that the equation relating the time evolution of a one of the vector displacement coordinates was given by

\begin{equation}\label{eqn:continuumElasticityReview:570}
\rho \PDSq{t}{u_i} = \PD{x_j}{\sigma_{ij}} + f_i,
\end{equation}

where the divergence term \(\PDi{x_j}{\sigma_{ij}}\) is the internal force per unit volume on the object and \(f_i\) is the external force.  Employing the constitutive relation we showed that this can be expanded as

\begin{equation}\label{eqn:continuumElasticityReview:590}
\rho \PDSq{t}{u_i} = (\lambda + \mu) \frac{\partial^2 u_k}{\partial x_i \partial x_k}
+ \mu
\frac{\partial^2 u_i}
{\partial x_j^2
},
\end{equation}

or in vector form

\begin{equation}\label{eqn:continuumElasticityReview:610}
\rho \PDSq{t}{\Bu} = (\lambda + \mu) \spacegrad (\spacegrad \cdot \Bu) + \mu \spacegrad^2 \Bu.
\end{equation}

\subsection{Equilibrium}

When a body is in static equilibrium \eqnref{eqn:continuumElasticityReview:570} reduces to just a simple force balance

\begin{equation}\label{eqn:continuumFluidsReviewXX:1090b}
f_i = - \PD{x_j}{\sigma_{ij}}.
\end{equation}

In particular, if there are no external forces then all of these divergences must be zero.

\subsection{P-waves} \index{p-wave}

Operating on \eqnref{eqn:continuumElasticityReview:610} with the divergence operator, and writing \(\Theta = \spacegrad \cdot \Bu\), a quantity that was our relative change in volume in the diagonal strain basis, we were able to find this divergence obeys a wave equation

\begin{equation}\label{eqn:continuumElasticityReview:630}
\PDSq{t}{\Theta} = \frac{\lambda + 2 \mu}{\rho} \spacegrad^2 \Theta.
\end{equation}

We called these P-waves.

\subsection{S-waves} \index{s-wave}

Similarly, operating on \eqnref{eqn:continuumElasticityReview:610} with the curl operator, and writing \(\Bomega = \spacegrad \cross \Bu\), we were able to find this curl also obeys a wave equation

\begin{equation}\label{eqn:continuumElasticityReview:650}
\rho \PDSq{t}{\Bomega} = \mu \spacegrad^2 \Bomega.
\end{equation}

These we called S-waves.  We also noted that the (transverse) compression waves (P-waves) with speed \(C_T = \sqrt{\mu/\rho}\), traveled faster than the (longitudinal) vorticity (S) waves with speed \(C_L = \sqrt{(\lambda + 2 \mu)/\rho}\) since \(\lambda > 0\) and \(\mu > 0\), and

\begin{equation}\label{eqn:continuumElasticityReview:670}
\frac{C_L}{C_T} = \sqrt{\frac{ \lambda + 2 \mu}{\mu}} = \sqrt{ \frac{\lambda}{\mu} + 2}.
\end{equation}

\subsection{Scalar and vector potential representation}

Assuming a vector displacement representation with gradient and curl components

\begin{equation}\label{eqn:continuumElasticityReview:690}
\Bu = \spacegrad \phi + \spacegrad \cross \BH,
\end{equation}

We found that the displacement time evolution equation split nicely into curl free and divergence free terms

\begin{equation}\label{eqn:continuumElasticityReview:710}
\spacegrad
\left(
\rho \PDSq{t}{\phi} - (\lambda + 2\mu) \spacegrad^2 \phi
\right)
+
\spacegrad \cross
\left(
\rho \PDSq{t}{\BH} - \mu \spacegrad^2 \BH
\right)
= 0.
\end{equation}

When neglecting boundary value effects this could be written as a pair of independent equations

\begin{subequations}
\begin{equation}\label{eqn:continuumElasticityReview:730}
\rho \PDSq{t}{\phi} - (\lambda + 2\mu) \spacegrad^2 \phi = 0
\end{equation}
\begin{equation}\label{eqn:continuumElasticityReview:750}
\rho \PDSq{t}{\BH} - \mu \spacegrad^2 \BH
= 0.
\end{equation}
\end{subequations}

This are the irrotational (curl free) P-wave and solenoidal (divergence free) S-wave equations respectively.

%This theory led to no actual calculation work, just a few videos that illustrated what we would presumably be able to calculate if we were to attempt to apply these concepts.

\subsection{Phasor description}

It was mentioned that we could assume a phasor representation for our potentials, writing

\begin{subequations}
\begin{equation}\label{eqn:continuumElasticityReview:770}
\phi = A \exp\left( i ( \Bk \cdot \Bx - \omega t) \right)
\end{equation}
\begin{equation}\label{eqn:continuumElasticityReview:790}
\BH = \BB \exp\left( i ( \Bk \cdot \Bx - \omega t) \right)
\end{equation}
\end{subequations}

finding

\begin{equation}\label{eqn:continuumElasticityReview:810}
\Bu = i \Bk \phi + i \Bk \cross \BH.
\end{equation}

We did nothing with neither the potential nor the phasor theory for solid displacement time evolution, and presumably will not on the exam either.

\subsection{Some wave types}

Some time was spent on qualitative descriptions and review of descriptions for solutions of the time evolution elasticity equations.
\FIXME{Unfortunately there was not any attempt to find these solutions or do any analysis}

\begin{itemize}
\item P-waves \citep{wiki:pwave}.  Irrotational, non volume preserving body wave.
\item S-waves \citep{wiki:swave}.  Divergence free body wave.  Shearing forces are present and volume is preserved (slower than S-waves)
\item Rayleigh wave \citep{wiki:rayleighwave}.  A surface wave that propagates near the surface of a body without penetrating into it.  It is pointed out in the class notes in the seismogram figure that these, while moving slower than the P (primary) or S (secondary) waves, have larger amplitude and are therefore the most destructive.
\item Love wave \citep{wiki:lovewave}.  A polarized shear surface wave with the shear displacements moving perpendicular to the direction of propagation.
\end{itemize}

