%
% Copyright � 2012 Peeter Joot.  All Rights Reserved.
% Licenced as described in the file LICENSE under the root directory of this GIT repository.
%

%
%
%\chapter{PHY454H1S\\Continuum Mechanics.  Lecture 7: P-waves and S-waves.  Taught by Prof. K. Das}
%\section{P-waves and S-waves}
\label{chap:continuumL7}

%
%\section{Setup}
%
%
%We got as far as expressing the vector displacement \(\Be\) for an isotropic material at a given point in terms of the Lam\'e parameters
%
%\begin{equation}\label{eqn:continuumL7:10}
%\rho \PDSq{t}{\Be} = (\lambda + \mu) \spacegrad (\spacegrad \cdot \Be) + \mu \spacegrad^2 \Be.
%\end{equation}
%
\section{P-waves.}
\index{p-wave}
%
Reading: \S 22 from \citep{landau1960theory}.

Operating on this with the divergence operator, and writing \(\theta = \spacegrad \cdot \Be\), we have
%
\begin{equation}\label{eqn:continuumL7:30}
\rho \PDSq{t}{\spacegrad \cdot \Be} = (\lambda + \mu) \spacegrad \cdot \spacegrad (\spacegrad \cdot \Be) + \mu \spacegrad^2 (\spacegrad \cdot \Be),
\end{equation}
%
or
%
\begin{equation}\label{eqn:continuumL7:50}
\PDSq{t}{\theta} = \frac{\lambda + 2 \mu}{\rho} \spacegrad^2 \theta.
\end{equation}
%
We see that our divergence is governed by a wave equation where the speed of the wave \(C_L\) is specified by
%
\begin{equation}\label{eqn:continuumL7:70}
C_L^2 = \frac{\lambda + 2 \mu}{\rho},
\end{equation}
%
so the displacement wave equation is given by
%
\begin{equation}\label{eqn:continuumL7:90}
\PDSq{t}{\theta} = C_L^2 \spacegrad^2 \theta.
\end{equation}
%
Let us look at the divergence of the displacement vector in some more detail.  By definition this is just
%
\begin{equation}\label{eqn:continuumL7:110}
\spacegrad \cdot \Be =
\PD{x_1}{e_1}
+\PD{x_2}{e_2}
+\PD{x_3}{e_3}.
\end{equation}
%
Recall that the strain tensor \(e_{ij}\) was defined as
%
\begin{equation}\label{eqn:continuumL7:130}
e_{ij} = \inv{2} \left(
\PD{x_j}{e_i}
+
\PD{x_i}{e_j}
\right),
\end{equation}
%
so we have
%
\begin{equation}\label{eqn:continuumL7:150}
\begin{aligned}
\PD{x_1}{e_1} &= e_{11} \\
\PD{x_2}{e_2} &= e_{22} \\
\PD{x_3}{e_3} &= e_{33}.
\end{aligned}
\end{equation}
%
So the divergence in question can be written in terms of the strain tensor
%
\begin{equation}\label{eqn:continuumL7:170}
\spacegrad \cdot \Be =
e_{11}
+e_{22}
+e_{33} = e_{ii}.
\end{equation}
%
We also found that the trace of the strain tensor was the relative change in volume.  We call this the dilatation.  A measure of change in volume as illustrated (badly) in \cref{fig:continuumL7:continuumL7fig1}
%
\imageFigure{../figures/phy454-continuumechanics/lec7_Illustrating_changes_in_a_control_volumeFig1}{Illustrating changes in a control volume.}{fig:continuumL7:continuumL7fig1}{0.2}
%
This idea can be found nicely animated in the wikipedia page \citep{wiki:pwave}.
%
\section{S-waves.}
\index{s-wave}
%
Now let us operate on our \cref{eqn:continuumL6:290} with the curl operator
%
\begin{equation}\label{eqn:continuumL7:190}
\rho \PDSq{t}{\spacegrad \cross \Be} = (\lambda + \mu) \spacegrad \cross (\spacegrad (\spacegrad \cdot \Be)) + \mu \spacegrad^2 (\spacegrad \cross \Be).
\end{equation}
%
Writing
%
\begin{equation}\label{eqn:continuumL7:210}
\Bomega = \spacegrad \cross \Be,
\end{equation}
%
and observing that \(\spacegrad \cross \spacegrad f = 0\) (with \(f = \spacegrad \cdot \Be\)), we find
%
\begin{equation}\label{eqn:continuumL7:230}
\rho \PDSq{t}{\Bomega} = \mu \spacegrad^2 \Bomega.
\end{equation}
%
We call this the S-wave equation, and write \(C_T\) for the speed of this wave
%
\begin{equation}\label{eqn:continuumL7:250}
C_T^2 = \frac{\mu}{\rho},
\end{equation}
%
so that we have
%
\begin{equation}\label{eqn:continuumL7:270}
\PDSq{t}{\Bomega} = C_T^2 \spacegrad^2 \Bomega.
\end{equation}
%
Again, we can find nice animations of this on wikipedia \citep{wiki:swave}.
%
\section{Relative speeds of the p-waves and s-waves.}
%
Taking ratios of the wave speeds we find
%
\begin{equation}\label{eqn:continuumL7:290}
\frac{C_L}{C_T} = \sqrt{\frac{ \lambda + 2 \mu}{\mu}} = \sqrt{ \frac{\lambda}{\mu} + 2}.
\end{equation}
%
Since both \(\lambda > 0\) and \(\mu > 0\), we have
%
\begin{equation}\label{eqn:continuumL7:310}
C_L > C_T.
\end{equation}
%
Divergence (p-waves) are faster than rotational (s-waves) waves.

In terms of the Poisson ratio \(\nu = \lambda/(2(\lambda + \mu))\), we find
%
\begin{equation}\label{eqn:continuumL7:330}
\frac{\mu}{\lambda} = \inv{2 \nu} - 1.
\end{equation}
%
we see that Poisson's ratio characterizes the speeds of the waves for the medium
%
\begin{equation}\label{eqn:continuumL7:350}
\frac{C_L}{C_T} = \sqrt{\frac{2(1-\nu)}{1 - 2\nu}}.
\end{equation}
%
\section{Assuming a gradient plus curl representation.}
\index{gradient}
\index{curl}
Let us assume that our displacement can be written in terms of a gradient and curl as we do for the electric field
\begin{equation}\label{eqn:continuumL7:370}
\Be = \spacegrad \phi + \spacegrad \cross \BH.
\end{equation}
Inserting this into \cref{eqn:continuumL6:290} we find
\begin{equation}\label{eqn:continuumL7:390}
\begin{aligned}
\rho &\PDSq{t}{(\spacegrad \phi + \spacegrad \cross \BH)} \\
&= (\lambda + \mu) \spacegrad (\spacegrad \cdot (\spacegrad \phi + \spacegrad \cross \BH)) + \mu \spacegrad^2 (\spacegrad \phi + \spacegrad \cross \BH).
\end{aligned}
\end{equation}
%
The first term on the RHS can be simplified.  First note that the divergence of the gradient is just a Laplacian
\begin{equation}\label{eqn:continuumL7:410}
\spacegrad \cdot \spacegrad \phi = \spacegrad^2 \phi,
\end{equation}
and then note that the divergence of a curl is zero
\begin{equation}\label{eqn:elasticWaves:510}
\begin{aligned}
\spacegrad \cdot (\spacegrad \cross \BH)
=
\partial_k (\partial_a H_b \epsilon_{abk}) =
0.
\end{aligned}
\end{equation}
%
The zero follows from the fact that the antisymmetric sum of symmetric partials is zero (assuming sufficient continuity).  Grouping terms we have
\begin{equation}\label{eqn:continuumL7:450}
\spacegrad
\left(
\rho \PDSq{t}{\phi} - (\lambda + 2\mu) \spacegrad^2 \phi
\right)
+
\spacegrad \cross
\left(
\rho \PDSq{t}{\BH} - \mu \spacegrad^2 \BH
\right)
= 0.
\end{equation}
%
When the material is infinite in scope, so that boundary value coupling is not a factor, we can write this as a set of independent P-wave and S-wave equations
\begin{equation}\label{eqn:continuumL7:470}
\rho \PDSq{t}{\phi} - (\lambda + 2\mu) \spacegrad^2 \phi = 0.
\end{equation}
The P-wave is irrotational (curl free).
\begin{equation}\label{eqn:continuumL7:490}
\rho \PDSq{t}{\BH} - \mu \spacegrad^2 \BH = 0.
\end{equation}
The S-wave is solenoidal (divergence free).
%
\section{A couple summarizing statements.}
\begin{itemize}
\item
P-waves: irrotational.  Volume not preserved.
\item
S-waves: divergence free.  Shearing forces are present and volume is preserved.
\item
P-waves are faster than S-waves.
\end{itemize}
