%
% Copyright � 2012 Peeter Joot.  All Rights Reserved.
% Licenced as described in the file LICENSE under the root directory of this GIT repository.
%
%
%
\makeoproblem{\(\BP\)-waves, \(\BS\)-waves, Love-waves}{problem:elastic:displacements:midtermQ1a}
{2012 midterm, p1 a}
{
Show that in \(\BP\)-waves the divergence of the displacement vector represents a measure of the relative change in the volume of the body.
} % makeoproblem

\makeanswer{problem:elastic:displacements:midtermQ1a}{
The \(\BP\)-wave equation was a result of operating on the displacement equation with the divergence operator
\begin{equation}\label{eqn:continuumMidTermReflection:10}
\spacegrad \cdot \left(
\rho \PDSq{t}{\Be} = (\lambda + \mu) \spacegrad (\spacegrad \cdot \Be) + \mu \spacegrad^2 \Be
\right)
\end{equation}
we obtain
\begin{equation}\label{eqn:continuumMidTermReflection:30}
\PDSq{t}{} \left( \spacegrad \cdot \Be \right) = \frac{\lambda + 2 \mu}{\rho} \spacegrad^2 (\spacegrad \cdot \Be).
\end{equation}
%
We have a wave equation where the ``waving'' quantity is \(\Theta = \spacegrad \cdot \Be\).  Explicitly
%
\begin{equation}\label{eqn:problems:3230}
\begin{aligned}
\Theta
&= \spacegrad \cdot \Be \\
&=
\PD{x}{e_1}
+\PD{y}{e_2}
+\PD{z}{e_3}
\end{aligned}
\end{equation}
%
Recall that, in a coordinate basis for which the strain \(e_{ij}\) is diagonal we have
\begin{equation}\label{eqn:continuumMidTermReflection:50}
\begin{aligned}
dx' &= \sqrt{1 + 2 e_{11}} dx \\
dy' &= \sqrt{1 + 2 e_{22}} dy \\
dz' &= \sqrt{1 + 2 e_{33}} dz.
\end{aligned}
\end{equation}
%
Expanding in Taylor series to \(O(1)\) we have for \(i = 1, 2, 3\) (no sum)
\begin{equation}\label{eqn:continuumMidTermReflection:70}
dx_i' \approx (1 + e_{ii}) dx_i.
\end{equation}
so the displaced volume is
\begin{equation}\label{eqn:problems:3250}
\begin{aligned}
dV' &=
dx_1
dx_2
dx_3
(1 + e_{11})
(1 + e_{22})
(1 + e_{33}) \\
&=
dx_1
dx_2
dx_3
( 1  + e_{11} + e_{22} + e_{33} + O(e_{kk}^2) )
\end{aligned}
\end{equation}
%
Since
\begin{equation}\label{eqn:continuumMidTermReflection:90}
\begin{aligned}
e_{11} &= \inv{2} \left( \PD{x}{e_1} +\PD{x}{e_1} \right) = \PD{x}{e_1} \\
e_{22} &= \inv{2} \left( \PD{y}{e_2} +\PD{y}{e_2} \right) = \PD{y}{e_2} \\
e_{33} &= \inv{2} \left( \PD{z}{e_3} +\PD{z}{e_3} \right) = \PD{z}{e_3}.
\end{aligned}
\end{equation}
%
We have
\begin{equation}\label{eqn:continuumMidTermReflection:110}
dV' = (1 + \spacegrad \cdot \Be) dV,
\end{equation}
or
\begin{equation}\label{eqn:continuumMidTermReflection:130}
\frac{dV' - dV}{dV} = \spacegrad \cdot \Be.
\end{equation}
%
The relative change in volume can therefore be expressed as the divergence of \(\Be\), the displacement vector, and it is this relative volume change that is ``waving'' in the \(\BP\)-wave equation as illustrated in the following \cref{fig:continuumMidtermReflection:continuumMidtermReflectionFig1} sample 1D compression wave
\imageFigure{../figures/phy454-continuumechanics/midtermReflectionA_1D_compression_waveFig1}{A 1D compression wave.}{fig:continuumMidtermReflection:continuumMidtermReflectionFig1}{0.2}
} % end answer

\makeoproblem{\(\BP\)-waves and \(\BS\)-waves.}{problem:elastic:displacements:midtermQ1b}
{2012 midterm, p1 b}
{
Between a \(\BP\)-wave and an \(\BS\)-wave which one is longitudinal and which one is transverse?
} % makeoproblem

\makeanswer{problem:elastic:displacements:midtermQ1b}{
\(\BP\)-waves are longitudinal.
\(\BS\)-waves are transverse.
} % end answer

\makeoproblem{Speed of \(\BP\)-waves and \(\BS\)-waves}{problem:elastic:displacements:midtermQ1c}
{2012 midterm, p1 c}
{
Whose speed is higher?
} % makeoproblem

\makeanswer{problem:elastic:displacements:midtermQ1c}{
From the (midterm) formula sheet we have
%
\begin{equation}\label{eqn:problems:3270}
\begin{aligned}
\left( \frac{c_L}{c_T} \right)^2
&= \frac{ \lambda + 2 \mu}{\rho} \frac{\rho}{\mu}  \\
&= \frac{\lambda}{\mu} + 2  \\
&> 1
\end{aligned}
\end{equation}
%
so \(\BP\)-waves travel faster than \(\BS\)-waves.
} % end answer

\makeoproblem{Love waves}{problem:elastic:displacements:midtermQ1d}
{2012 midterm, p1 d}
{
Is Love wave a body wave or a surface wave?
} % makeoproblem

\makeanswer{problem:elastic:displacements:midtermQ1d}{
Love waves are surface waves, traveling in a medium that can slide on top of another surface.  They are characterized by shear displacements perpendicular to the direction of propagation.

Reviewing for the final I see that I had answered this wrong, and have corrected it.  I had described a Rayleigh wave (also a surface wave).  A Rayleigh wave is characterized by vorticity rotating backwards compared to the direction of propagation as shown in \cref{fig:continuumMidtermReflection:continuumMidtermReflectionFig2}
%
\pdfTexFigure{../figures/phy454-continuumechanics/continuumMidtermReflectionFig2.pdf_tex}{Rayleigh wave illustrated.}{fig:continuumMidtermReflection:continuumMidtermReflectionFig2}{0.6}
} % end answer
\makeproblem{Equilibrium.}{problem:elastic:displacements:exampractiseEquilibrium}{
Suppose that the state of a body is given by
%
\begin{dmath}\label{eqn:continuumFluidsReviewXX:3130}
\sigma_{11} = A x^4 y^3
\sigma_{22} = 3 B x^2 y^5
\sigma_{12} = -C x^3 y^4
\end{dmath}
%
Determine the constants \(A\), \(B\) and \(C\) so that the body is in equilibrium (2011 Final Exam question II).
} % makeproblem
\makeanswer{problem:elastic:displacements:exampractiseEquilibrium}{
We have
%
\begin{dmath}\label{eqn:continuumFluidsReviewXX:3150}
0
= \PD{x_j}{\sigma_{1j}}
=
\PD{x}{\sigma_{11}} + \PD{y}{\sigma_{12}}
= 4 A x^3 y^3 - 4 C x^3 y^3,
\end{dmath}
%
and
%
%
\begin{dmath}\label{eqn:continuumFluidsReviewXX:3170}
0
= \PD{x_j}{\sigma_{2j}}
=
\PD{x}{\sigma_{21}} + \PD{y}{\sigma_{22}}
= -3 C x^2 y^4 + 15 B x^2 y^4
\end{dmath}
%
We must then have
%
\begin{dmath}\label{eqn:continuumFluidsReviewXX:3190}
0 = A - C
0 = -C + 5 B,
\end{dmath}
or
\begin{dmath}\label{eqn:continuumFluidsReviewXX:3210}
A = C
B = \frac{C}{5}.
\end{dmath}
} % end answer

\makeoproblem{Tsunami}{problem:elastic:displacements:exampractiseTsunami}
{2011 final}
{
Explain how the strain energy of tectonic plates causes Tsunami.
} % makeoproblem

\makeanswer{problem:elastic:displacements:exampractiseTsunami}{
The root cause of the Tsunami is the earthquake under the body of water.  Once that earthquake occurs we will have a body wave in the mantle, which will trigger a much more destructive (higher amplitude) surface wave (probably of the Rayleigh type).  Looking back to the connection with strain energy, we see that once we have a change in the strain divergence, we will have to have a restoring force to put things back in equilibrium.  That restoring force can come either from the surrounding mantle or the fluid above it, and it is that fluid restoring force that induces the wave as a side effect.
} % end answer

\FIXME{This is from the 2012 midterm.  We never got any real problems on elastic waves.  My preference would have been for actual problems that require solutions to the wave equations under various conditions.  If we then examined those solutions and characterized them (Love, Rayleigh, ...) we would not just have a requirement to restate memorized descriptive stuff, a task of little value.}
\makeproblem{Wave equation solutions.}{problem:elastic:displacements:placeholderWaveEquation}{
PLACEHOLDER.
\FIXME{We never did get any homework assignments with actual problems where we find Rayleigh or Love wave solutions, so that we could get a feel for how to apply the formalism.  This would be a good place to put some.}
} % makeproblem
