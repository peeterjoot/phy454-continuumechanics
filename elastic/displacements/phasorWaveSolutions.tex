%
% Copyright � 2012 Peeter Joot.  All Rights Reserved.
% Licenced as described in the file LICENSE under the root directory of this GIT repository.
%

%
%
\section{Phasor description of elastic waves.}
\index{phasor}
\index{elastic wave}
%\chapter{PHY454H1S\\Continuum Mechanics.  Lecture 8: Phasor description of elastic waves.  Fluid dynamics.  Taught by Prof. K. Das}
\label{chap:continuumL8}
%
%\section{Review.  Elastic wave equation}
%
%Starting with
%
%\begin{equation}\label{eqn:continuumL8:10}
%\rho \PDSq{t}{\Be} = (\lambda + \mu) \spacegrad (\spacegrad \cdot \Be) + \mu \spacegrad^2 \Be
%\end{equation}
%
%and applying a divergence operation we find
%
%\begin{align}\label{eqn:continuumL8:30}
%\rho \PDSq{t}{\theta} &= C_L^2 \spacegrad^2 \theta \\
%\theta &= \spacegrad \cdot \Be \\
%C_L^2 &= \frac{\lambda + 2\mu}{\rho}.
%\end{align}
%
%This is the P-wave equation.  Applying a curl operation we find
%
%\begin{align}\label{eqn:continuumL8:50}
%\rho \PDSq{t}{\Bomega} &= C_T^2 \spacegrad^2 \Bomega \\
%\Bomega &= \spacegrad \cross \Be \\
%C_T^2 &= \frac{\lambda + 2\mu}{\rho}.
%\end{align}
%
%This is the S-wave equation.  We also found that
%
%\begin{equation}\label{eqn:continuumL8:70}
%\frac{C_L}{C_T} > 1,
%\end{equation}
%
%and concluded that P waves are faster than S waves.  What we have not shown is that the P waves are longitudinal, and that the S waves are transverse.
%
%Assuming a gradient and curl description of our displacement
%
%\begin{equation}\label{eqn:continuumL8:90}
%\Be = \spacegrad \phi + \spacegrad \cross \BH = \BP + \BS,
%\end{equation}
%
%we found
%
%\begin{align}\label{eqn:continuumL8:110}
%(\lambda + 2 \mu) \spacegrad^2 \phi - \rho \PDSq{t}{\phi} &= 0 \\
%\mu \spacegrad^2 \BH - \rho \PDSq{t}{\BH} &= 0,
%\end{align}
%
%allowing us to separately solve for the P and the S wave solutions respectively.
%
Let us introduce a phasor representation (again following \S 22 of the text \citep{landau1960theory})
%
\begin{equation}\label{eqn:continuumL8:130}
\begin{aligned}
\phi &= A \exp\left( i ( \Bk \cdot \Bx - \omega t) \right) \\
\BH &= \BB \exp\left( i ( \Bk \cdot \Bx - \omega t) \right).
\end{aligned}
\end{equation}
%
Operating with the gradient we find
%
\begin{equation}\label{eqn:phasorWaveSolutions:230}
\begin{aligned}
\BP
&= \spacegrad \phi \\
&= \Be_k \partial_k A \exp\left( i ( \Bk \cdot \Bx - \omega t) \right) \\
&= \Be_k \partial_k A \exp\left( i ( k_m x_m - \omega t) \right) \\
&= \Be_k i k_k A \exp\left( i ( k_m x_m - \omega t) \right) \\
&= i \Bk A \exp\left( i ( \Bk \cdot \Bx - \omega t) \right) \\
&= i \Bk \phi.
\end{aligned}
\end{equation}
%
We can also write
%
\begin{equation}\label{eqn:continuumL8:150}
\BP = \Bk \phi',
\end{equation}
%
where \(\phi'\) is the derivative of \(\phi\) ``with respect to its argument''.   Here argument must mean the entire phase \(\Bk \cdot \Bx - \omega t\).
%
\begin{equation}\label{eqn:continuumL8:170}
\phi' = \frac{ d\phi( \Bk \cdot \Bx - \omega t )}{ d(\Bk \cdot \Bx - \omega t) } = i \phi.
\end{equation}
%
Actually, argument is a good label here, since we can use the word in the complex number sense.

For the curl term we find
%
\begin{equation}\label{eqn:phasorWaveSolutions:250}
\begin{aligned}
\BS
&= \spacegrad \cross \BH \\
&= \Be_a \partial_b H_c \epsilon_{a b c} \\
&= \Be_a \partial_b \epsilon_{a b c} B_c \exp\left( i ( \Bk \cdot \Bx - \omega t) \right) \\
&= \Be_a \partial_b \epsilon_{a b c} B_c \exp\left( i ( k_m x_m - \omega t) \right) \\
&= \Be_a i k_b \epsilon_{a b c} B_c \exp\left( i ( \Bk \cdot \Bx - \omega t) \right) \\
&= i \Bk \cross \BH.
\end{aligned}
\end{equation}
%
Again writing
\begin{equation}\label{eqn:continuumL8:190}
\BH' = \frac{ d\BH( \Bk \cdot \Bx - \omega t )}{ d(\Bk \cdot \Bx - \omega t) } = i \BH
\end{equation}
%
we can write the S wave as
%
\begin{equation}\label{eqn:continuumL8:210}
\BS = \Bk \cross \BH'.
\end{equation}
%
\section{Some wave types described.}
%
The following wave types were noted, but not defined:
\begin{itemize}
\item Rayleigh wave.  This is discussed in \S 24 of the text (a wave that propagates near the surface of a body without penetrating into it).  Wikipedia has an illustration of one possible mode of propagation \citep{wiki:rayleighwave}.
\item Love wave.  These are not discussed in the text, but wikipedia \citep{wiki:lovewave} describes them as polarized shear waves (where the figure indicates that the shear displacements are perpendicular to the direction of propagation).
\end{itemize}
Some illustrations from the class notes were also shown.
