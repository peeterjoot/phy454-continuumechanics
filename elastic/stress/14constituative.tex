%
% Copyright � 2012 Peeter Joot.  All Rights Reserved.
% Licenced as described in the file LICENSE under the root directory of this GIT repository.
%
\section{Constitutive relation.}
%
Reading: \S 2, \S 4 and \S 5 from the text \citep{landau1960theory}.

We can find the relationship between stress and strain, both analytically and experimentally, and call this the Constitutive relation.  We prefer to deal with ranges of distortion that are small enough that we can make a linear approximation for this relation.  In general such a linear relationship takes the form
%
\begin{equation}\label{eqn:continuumL5:70}
\sigma_{ij} = c_{ijkl} e_{kl}.
\end{equation}
%
Materials for which the stress and strain tensors are linearly related are called Newtonian \index{Newtonian material}.  We will not consider any non-Newtonian materials in this course.

Consider the number of components that we are talking about for various rank tensors
%
\begin{equation}\label{eqn:continuumL5:90}
\begin{array}{l l}
\mbox{\(0^{\text{th}}\) rank tensor} & \mbox{\(3^0 = 1\) components}, \\
\mbox{\(1^{\text{st}}\) rank tensor} & \mbox{\(3^1 = 3\) components}, \\
\mbox{\(2^{\text{nd}}\) rank tensor} & \mbox{\(3^2 = 9\) components}, \\
\mbox{\(3^{\text{rd}}\) rank tensor} & \mbox{\(3^3 = 81\) components}.
\end{array}
\end{equation}
%
We have a lot of components, even for a linear relation between stress and strain.  For isotropic materials we model the constitutive relation instead as
\boxedEquation{eqn:continuumL5:110}{
\sigma_{ij} = \lambda e_{kk} \delta_{ij} + 2 \mu e_{ij}.
}
It can be shown \citep{feynman1963flp:elasticMaterials} that a relationship between stress and strain of this form is actually required by isotropy.

For such a modeling of the material the (measured) values \(\lambda\) and \(\mu\) (shear modulus or modulus of rigidity) are called the Lam\'e parameters.

It will be useful to compute the trace of the stress tensor in the form of the constitutive relation for the isotropic model.  We find
%
\begin{equation}\label{eqn:14constituative:170}
\begin{aligned}
\sigma_{ii}
&= \lambda e_{kk} \delta_{ii} + 2 \mu e_{ii} \\
&= 3 \lambda e_{kk} + 2 \mu e_{jj},
\end{aligned}
\end{equation}
or
%
\begin{equation}\label{eqn:continuumL5:150}
\sigma_{ii} = (3 \lambda + 2 \mu) e_{kk}.
\end{equation}
We can now also invert this, to find the trace of the strain tensor in terms of the stress tensor
%
\begin{equation}\label{eqn:continuumL5:130}
e_{ii} = \frac{\sigma_{kk}}{3 \lambda + 2 \mu}.
\end{equation}
Substituting back into our original relationship \cref{eqn:continuumL5:110}, and find
%
\begin{equation}\label{eqn:continuumL5:110b}
\sigma_{ij} = \lambda \frac{\sigma_{kk}}{3 \lambda + 2 \mu} \delta_{ij} + 2 \mu e_{ij},
\end{equation}
which finally provides an inverted expression with the strain tensor expressed in terms of the stress tensor
\boxedEquation{eqn:continuumL5:110c}{
2 \mu e_{ij} =
\sigma_{ij} - \lambda \frac{\sigma_{kk}}{3 \lambda + 2 \mu} \delta_{ij}.
}
