%
% Copyright � 2012 Peeter Joot.  All Rights Reserved.
% Licenced as described in the file LICENSE under the root directory of this GIT repository.
%
\section{Cauchy tetrahedron.}
%
To examine the question of how the stress tensor and the force relate, we project the force onto a planar surface.  This is called the \textAndIndex{Cauchy tetrahedron} as in \cref{fig:continuumL4:continuumL4fig7}
%
\imageFigure{../figures/phy454-continuumechanics/lec4_Cauchy_tetrahedronFig7}{Cauchy tetrahedron.}{fig:continuumL4:continuumL4fig7}{0.3}
%
\begin{equation}\label{eqn:continuumL4:150}
\Bf = \frac{\text{external force}}{\text{unit area}} = f_j \Be_j,
\end{equation}
%
or
\begin{equation*}%\label{eqn:continuumL4:170}
\text{internal stress} = \text{external force}.
\end{equation*}
%
We write \(\ncap\) in terms of the direction cosines
%
\begin{equation}\label{eqn:continuumL4:190}
\ncap =
n_1 \Be_1 +
n_2 \Be_2 +
n_3 \Be_3.
\end{equation}
%
Here
%
\begin{align}\label{eqn:continuumL4:210}
n_1 &= \ncap \cdot \Be_1 \\
n_2 &= \ncap \cdot \Be_2 \\
n_3 &= \ncap \cdot \Be_3,
\end{align}

or
%
\begin{equation}\label{eqn:continuumL4:230}
n_j = \ncap \cdot \Be_j = \cos\phi_j.
\end{equation}
%
This is illustrated in \cref{fig:continuumL5:continuumL5fig1}.
%
\imageFigure{../figures/phy454-continuumechanics/lec5_Cauchy_tetrahedron_direction_cosinesFig1}{Cauchy tetrahedron direction cosines.}{fig:continuumL5:continuumL5fig1}{0.2}
%
Performing a force balance on \(x_1\) direction, where we match total external force in each direction to the total internal force (\(\sigma_{ij}'s\)) as follows
%
\begin{equation}\label{eqn:continuumL4:250}
\begin{aligned}
f_1 \times \text{area ABC}
&=
\sigma_{11} \times \text{area BOC} \\
&+\sigma_{12} \times \text{area AOC} \\
&+\sigma_{13} \times \text{area AOB}.
\end{aligned}
\end{equation}
%
Similarly
%
\begin{equation}\label{eqn:continuumL4:270}
\begin{aligned}
f_2 \times \text{area ABC}
&=
\sigma_{21} \times \text{area BOC} \\
&+\sigma_{22} \times \text{area AOC} \\
&+\sigma_{23} \times \text{area AOB},
\end{aligned}
\end{equation}
%
and
%
\begin{equation}\label{eqn:continuumL4:290}
\begin{aligned}
f_3 \times \text{area ABC}
&=
\sigma_{31} \times \text{area BOC} \\
&+\sigma_{32} \times \text{area AOC} \\
&+\sigma_{33} \times \text{area AOB}.
\end{aligned}
\end{equation}
%
We can therefore write these force components like
%
\begin{equation}\label{eqn:continuumL4:310}
f_1 =
\sigma_{11} \frac{BOC}{ABC} +
\sigma_{12} \frac{AOC}{ABC} +
\sigma_{13} \frac{AOB}{ABC}
\end{equation}
%
but these ratios are really just the projections of the areas as illustrated in \cref{fig:continuumL4:continuumL4fig8}
%
\imageFigure{../figures/phy454-continuumechanics/lec4_Area_projectionFig8}{Area projection.}{fig:continuumL4:continuumL4fig8}{0.2}
%
where an arbitrary surface with area \(\Delta S\) can be decomposed into projections
%
\begin{equation}\label{eqn:continuumL4:330}
\Delta S \cos\phi_j,
\end{equation}
%
utilizing the direction cosines.  We can therefore write
%
\begin{align}\label{eqn:continuumL4:350}
f_1 &= \sigma_{11} n_1 + \sigma_{12} n_2 + \sigma_{13} n_3 \\
f_2 &= \sigma_{21} n_1 + \sigma_{22} n_2 + \sigma_{23} n_3 \\
f_3 &= \sigma_{31} n_1 + \sigma_{32} n_2 + \sigma_{33} n_3,
\end{align}
or in matrix notation
%
\begin{equation}\label{eqn:continuumL4:370}
\begin{bmatrix}
f_1  \\
f_2  \\
f_3
\end{bmatrix}
=
\begin{bmatrix}
\sigma_{11} & \sigma_{12} & \sigma_{13} \\
\sigma_{21} & \sigma_{22} & \sigma_{23} \\
\sigma_{31} & \sigma_{32} & \sigma_{33}
\end{bmatrix}
\begin{bmatrix}
n_1 \\
n_2 \\
n_3 \\
\end{bmatrix}.
\end{equation}
%
This is just
%
\boxedEquation{eqn:continuumL4:390}{
f_i = \sigma_{ij} n_j.
}
%
This force with components \(f_i\) is also called the \textAndIndex{traction vector}
%
\begin{equation}\label{eqn:continuumL4:410}
\tau_i = \sigma_{ij} n_j.
\end{equation}
%
In matrix form the traction vector is
%
\begin{equation}\label{eqn:continuumL5:50}
\Btau = \Bsigma \cdot \ncap
=
\begin{bmatrix}
\sigma_{11} & \sigma_{12} & \sigma_{13} \\
\sigma_{21} & \sigma_{22} & \sigma_{23} \\
\sigma_{31} & \sigma_{32} & \sigma_{33}
\end{bmatrix}
\begin{bmatrix}
n_1 \\
n_2 \\
n_3
\end{bmatrix}.
\end{equation}
