%
% Copyright � 2012 Peeter Joot.  All Rights Reserved.
% Licenced as described in the file LICENSE under the root directory of this GIT repository.
%
%
%
\makeoproblem{Strain tensor from stress tensor.}{problem:continuumProblemSet1:q1}
{2012 ps1, p1}
{
For the stress tensor
\begin{equation}\label{eqn:continuumProblemSet1:10}
\sigma =
\begin{bmatrix}
6 & 0 & 2 \\
0 & 1 & 1 \\
2 & 1 & 3
\end{bmatrix}
\text{M Pa}.
\end{equation}
Find the corresponding strain tensor, assuming an isotropic solid with Young's modulus \(E = 200 \times 10^9 \text{N}/\text{m}^2\) and Poisson's ration \(\nu = 0.35\).
} % makeoproblem
\makeanswer{problem:continuumProblemSet1:q1}{
We need to express the relation between stress and strain in terms of Young's modulus and Poisson's ratio.  In terms of Lam\'e parameters our model for the relations between stress and strain for an isotropic solid was given as
\begin{equation}\label{eqn:continuumProblemSet1:110}
\sigma_{ij} = \lambda e_{kk} \delta_{ij} + 2 \mu e_{ij}.
\end{equation}
%
Computing the trace
\begin{equation}\label{eqn:continuumProblemSet1:130}
\sigma_{kk} = (3 \lambda + 2 \mu) e_{kk},
\end{equation}
allows us to invert the relationship
\begin{equation}\label{eqn:continuumProblemSet1:150}
2 \mu e_{ij} = \sigma_{ij} - \lambda \frac{\sigma_{kk}}{3 \lambda + 2 \mu} \delta_{ij}.
\end{equation}
%
In terms of Poisson's ratio \(\nu\) and Young's modulus \(E\), our Lam\'e parameters were found to be
\begin{equation}\label{eqn:continuumProblemSet1:170}
\begin{aligned}
\lambda &= \frac{ E \nu }{(1 - 2 \nu)(1 + \nu)} \\
\mu &= \frac{E}{2(1 + \nu)},
\end{aligned}
\end{equation}
and
\begin{equation}\label{eqn:problems:571}
\begin{aligned}
3 \lambda + 2 \mu
&= \frac{ 3 E \nu }{(1 - 2 \nu)(1 + \nu)} + \frac{E}{1 + \nu} \\
&= \frac{E}{1 + \nu} \left( \frac{3 \nu}{1 - 2 \nu} + 1\right) \\
&= \frac{E}{1 + \nu} \frac{1 + \nu}{1 - 2 \nu} \\
&= \frac{E}{1 - 2 \nu}.
\end{aligned}
\end{equation}
%
Our stress strain model for the relationship for an isotropic solid becomes
\begin{equation}\label{eqn:problems:591}
\begin{aligned}
\frac{E}{1 + \nu} e_{ij}
&=
\sigma_{ij}
-
\frac{ E \nu }{(1 - 2 \nu)(1 + \nu)} \frac{1 - 2 \nu}{E}
\sigma_{kk} \delta_{ij} \\
&=
\sigma_{ij}
-
\frac{ \nu }{1 + \nu}
\sigma_{kk} \delta_{ij},
\end{aligned}
\end{equation}
or
\begin{equation}\label{eqn:continuumProblemSet1:190}
e_{ij}
=
\inv{E}
\left(
(1 + \nu)
\sigma_{ij}
-
\nu
\sigma_{kk} \delta_{ij}
\right).
\end{equation}
%
As a sanity check note that this matches (5.12) of \citep{landau1960theory}, although they use a notation of \(\sigma\) instead of \(\nu\) for Poisson's ratio.  We are now ready to tackle the problem.  First we need the trace of the stress tensor
\begin{equation}\label{eqn:continuumProblemSet1:210}
\sigma_{kk} = (6 + 1 + 3) \text{M Pa} = 10 \text{M Pa},
\end{equation}
\begin{equation}\label{eqn:problems:611}
\begin{aligned}
e_{ij}
&=
\inv{E}
\left(
(1 + \nu)
\begin{bmatrix}
6 & 0 & 2 \\
0 & 1 & 1 \\
2 & 1 & 3
\end{bmatrix}
-
10 \nu
\begin{bmatrix}
1 & 0 & 0 \\
0 & 1 & 0 \\
0 & 0 & 1 \\
\end{bmatrix}
\right)
\text{M Pa} \\
&=
\inv{E}
\left(
\begin{bmatrix}
6 & 0 & 2 \\
0 & 1 & 1 \\
2 & 1 & 3
\end{bmatrix}
+ 0.35
\begin{bmatrix}
-4 & 0 & 2 \\
0 & -9 & 1 \\
2 & 1 & -7
\end{bmatrix}
\right)
\text{M Pa} \\
&=
\inv{2 \times 10^{5}}
\left(
\begin{bmatrix}
6 & 0 & 2 \\
0 & 1 & 1 \\
2 & 1 & 3
\end{bmatrix}
+ 0.35
\begin{bmatrix}
-4 & 0 & 2 \\
0 & -9 & 1 \\
2 & 1 & -7
\end{bmatrix}
\right)
\end{aligned}
\end{equation}
%
Expanding out the last bits of arithmetic the strain tensor is found to have the form
\begin{equation}\label{eqn:continuumProblemSet1:230}
e_{ij}
=
\begin{bmatrix}
 23 & 0 & 13.5 \\
 0 & -10.75 & 6.75 \\
 13.5 & 6.75 & 2.75
\end{bmatrix}
 10^{-6}.
\end{equation}
%
Note that this is dimensionless, unlike the stress.

Associated Mathematica notebook for this problem (\nbref{continuumProblemSet1Q1.cdf})
} % end answer

\makeoproblem{Small stress and strain.}{problem:continuumProblemSet1:q2b}
{2012 ps1, p2 b}
{
For the problem \ref{problem:strain:ps1q2a}, is the body under compression or expansion?
} % makeoproblem
%
\makeanswer{problem:continuumProblemSet1:q2b}{
To consider this question, suppose that as in the previous part, we determine a basis for which our strain tensor \(e_{ij} = p_i \delta_{ij}\) is diagonal with respect to that basis at a given point \(\Bx_0\).  We can then simplify the form of the stress tensor at that point in the object
\begin{equation}\label{eqn:problems:631}
\begin{aligned}
\sigma_{ij}
&=
\frac{E}{1 + \nu} \left(
e_{ij} + \frac{\nu}{1 - 2 \nu} e_{mm} \delta_{ij}
\right) \\
&=
\frac{E}{1 + \nu} \left(
p_i
 + \frac{\nu}{1 - 2 \nu} e_{mm}
\right)
\delta_{ij}.
\end{aligned}
\end{equation}
%
We see that the stress tensor at this point is also necessarily diagonal if the strain is diagonal in that basis (with the implicit assumption here that we are talking about an isotropic material).  Noting that the Poisson ratio is bounded according to
%
\begin{equation}\label{eqn:continuumProblemSet1:350}
-1 \le \nu \le \inv{2},
\end{equation}
%
so if our trace is positive (as it is in this problem for all points \(x_2 > 1/2\)), then any positive principle strain value will result in a positive stress along that direction).  For example at the point \((1,2,4)\) of the previous part of this problem (for which \(x_2 > 1/2\)), we have
%
\begin{equation}\label{eqn:continuumProblemSet1:370}
\sigma_{ij}
=
\frac{E}{1 + \nu}
\begin{bmatrix}
5.19684
+ \frac{3 \nu}{1 - 2 \nu}  & 0 & 0 \\
0 & -4.53206
+ \frac{3 \nu}{1 - 2 \nu}  & 0 \\
0 & 0 & 2.33522
+ \frac{3 \nu}{1 - 2 \nu}
\end{bmatrix}.
\end{equation}
%
We see that at this point the \((1,1)\) and \((3,3)\) components of stress is positive (expansion in those directions) regardless of the material, and provided that
%
\begin{equation}\label{eqn:continuumProblemSet1:390}
\frac{3 \nu}{1 - 2 \nu} > 4.53206,
\end{equation}
%
(i.e. \(\nu > 0.375664\)) the material is under expansion in all directions.  For \(\nu < 0.375664\) the material at that point is expanding in the \(\pcap_1\) and \(\pcap_3\) directions, but under compression in the \(\pcap_2\) directions.
%(save to disk and run with either Mathematica or the free Wolfram CDF player ( http://www.wolfram.com/cdf-player/  ) )
For a visualization of this part of this problem see (\nbref{continuumProblemSet1Q2animated.cdf}).  This animates the stress tensor associated with the problem, for different points \((x,y,z)\) and values of Poisson's ratio \(\nu\), with Mathematica manipulate sliders available to alter these (as well as a zoom control to scale the graphic, keeping the orientation and scale fixed with any variation of the other parameters).  This generalizes the solution of the problem (assuming I got it right for the specific \((1,2,4)\) point of the problem).  The vectors are the orthonormal eigenvectors of the tensor, scaled by the magnitude of the eigenvectors of the stress tensor (also diagonal in the basis of the diagonalized strain tensor at the point in question).  For those directions that are under expansive stress, I have colored the vectors blue, and for compressive directions, I have colored the vectors red.

A confirmation of the characteristic equation calculated manually is also available (\nbref{continuumProblemSet1Q2.cdf}).
} % end answer
\makeoproblem{Traction vector.}{problem:continuumProblemSet1:q3}
{2012 ps1, p3}
{
The stress tensor at a point has components given by
%
\begin{equation}\label{eqn:continuumProblemSet1:50}
\sigma =
\begin{bmatrix}
1 & -2 & 2 \\
-2 & 3 & 1 \\
2 & 1 & -1
\end{bmatrix}.
\end{equation}
%
Find the traction vector across an area normal to the unit vector
%
\begin{equation}\label{eqn:continuumProblemSet1:70}
\ncap = ( \sqrt{2} \Be_1 - \Be_2 + \Be_3)/2.
\end{equation}
%
Can you construct a tangent vector \(\Btau\) on this plane by inspection?  What are the components of the force per unit area along the normal \(\ncap\) and tangent \(\Btau\) on that surface?  (hint: projection of the traction vector.)
} % makeoproblem
%
\makeanswer{problem:continuumProblemSet1:q3}{
The traction vector, the force per unit volume that holds a body in equilibrium, in coordinate form was
%
\begin{equation}\label{eqn:continuumProblemSet1:410}
P_i = \sigma_{ik} n_k,
\end{equation}
%
where \(n_k\) was the coordinates of the normal to the surface with area \(df_k\).  In matrix form, this is just
%
\begin{equation}\label{eqn:continuumProblemSet1:430}
\BP = \sigma \ncap,
\end{equation}
%
so our traction vector for this stress tensor and surface normal is just
%
\begin{equation}\label{eqn:problems:651}
\begin{aligned}
\BP &=
\inv{2}
\begin{bmatrix}
1 & -2 & 2 \\
-2 & 3 & 1 \\
2 & 1 & -1
\end{bmatrix}
\begin{bmatrix}
\sqrt{2} \\
-1 \\
1
\end{bmatrix} \\
&=
\inv{2}
\begin{bmatrix}
\sqrt{2} + 2 + 2 \\
-2\sqrt{2} - 3 + 1 \\
2\sqrt{2} - 1 -1
\end{bmatrix} \\
&=
\begin{bmatrix}
\sqrt{2}/2 + 2 \\
-\sqrt{2} -1 \\
\sqrt{2} - 1
\end{bmatrix}.
\end{aligned}
\end{equation}
%
We also want a vector in the plane, and can pick
%
\begin{equation}\label{eqn:continuumProblemSet1:450}
\Btau =
\inv{\sqrt{2}}
\begin{bmatrix}
0 \\
1 \\
1
\end{bmatrix},
\end{equation}
%
or
%
\begin{equation}\label{eqn:continuumProblemSet1:470}
\Btau' =
\begin{bmatrix}
\inv{\sqrt{2}} \\
\inv{2} \\
-\inv{2}
\end{bmatrix}.
\end{equation}
%
It is clear that either of these is normal to \(\ncap\) (the first can also be computed by normalizing \(\ncap \cross \Be_1\), and the second with one round of Gram-Schmidt).  However, neither of these vectors in the plane are particularly interesting since they are completely arbitrary.  Let us instead compute the projection and rejection of the traction vector with respect to the normal.  We find for the projection
%
\begin{equation}\label{eqn:problems:671}
\begin{aligned}
(\BP \cdot \ncap) \ncap
&=
\inv{4}
\left(
\begin{bmatrix}
\sqrt{2}/2 + 2 \\
-\sqrt{2} -1 \\
\sqrt{2} - 1
\end{bmatrix}
\cdot
\begin{bmatrix}
\sqrt{2} \\
-1 \\
1
\end{bmatrix}
\right)
\begin{bmatrix}
\sqrt{2} \\
-1 \\
1
\end{bmatrix}  \\
&=
\inv{4}
\left(
1 + 2\sqrt{2}
+\sqrt{2} +1
+\sqrt{2} - 1
\right)
\begin{bmatrix}
\sqrt{2} \\
-1 \\
1
\end{bmatrix}  \\
&=
\inv{2}
\left(
1 + 4\sqrt{2}
\right)
\ncap.
\end{aligned}
\end{equation}
%
Our rejection, the component of the traction vector in the plane, is
%
\begin{equation}\label{eqn:problems:691}
\begin{aligned}
(\BP \wedge \ncap) \ncap
&=
\BP - (\BP \cdot \ncap)\ncap \\
&=
\inv{2}
\begin{bmatrix}
\sqrt{2}/2 + 2 \\
-\sqrt{2} -1 \\
\sqrt{2} - 1
\end{bmatrix}
-\inv{4}(1 + r \sqrt{2})
\begin{bmatrix}
\sqrt{2} \\
-1 \\
1
\end{bmatrix} \\
&=
\inv{4}
\begin{bmatrix}
\sqrt{2} \\
-3 \\
-5
\end{bmatrix}.
\end{aligned}
\end{equation}
%
This gives us a another vector perpendicular to the normal \(\ncap\)
%
\begin{equation}\label{eqn:continuumProblemSet1:490}
\taucap =
\inv{6}
\begin{bmatrix}
\sqrt{2} \\
-3 \\
-5
\end{bmatrix}.
\end{equation}
%
Wrapping up, we find the decomposition of the traction vector in the direction of the normal and its projection onto the plane to be
%
\begin{equation}\label{eqn:continuumProblemSet1:510}
\BP
=
\inv{2}(1 + 4\sqrt{2}) \ncap
+
\frac{3}{2} \taucap.
\end{equation}
%
The components we can read off by inspection.

Associated Mathematica notebook for this problem ({continuumProblemSet1Q3.cdf}).
} % end answer

\makeoproblem{Stress and equilibrium.}{problem:continuumProblemSet1:q4}
{2012 ps1, p4}
{
The stress tensor of a body is given by
%
\begin{equation}\label{eqn:continuumProblemSet1:90}
\sigma =
\begin{bmatrix}
A \cos x & y^2 & C x \\
y^2 & B \sin y & z \\
C x & z & z^3
\end{bmatrix}.
\end{equation}
%
Determine the constant \(A\), \(B\), and \(C\) if the body is in equilibrium.
} % makeoproblem
\makeproblem{Compute stress tensors for some typical 3D forces.}{problem:stressFIXME:p1}{
PLACEHOLDER.  Pick some 3D forces acting on unit volumes, and compute the stress tensors associated with them.
} % makeproblem
\makeproblem{Compute stress tensors for some typical 3D forces.}{problem:stressFIXME:p2}{
Referring to \cref{fig:continuumL3:continuumL3fig6}, what form would the stress tensor take?
PLACEHOLDER.  I had assume that we have to use a radial form of the tensor?
} % makeproblem
\makeanswer{problem:continuumProblemSet1:q4}{
In the absence of external forces our equilibrium condition was
%
\begin{equation}\label{eqn:continuumProblemSet1:530}
\partial_k \sigma_{ik} = 0.
\end{equation}
%
In matrix form we wish to operate (to the left) with the gradient coordinate vector
%
\begin{equation}\label{eqn:problems:710}
\begin{aligned}
0
&= \sigma \lspacegrad \\
&=
\begin{bmatrix}
A \cos x & y^2 & C x \\
y^2 & B \sin y & z \\
C x & z & z^3
\end{bmatrix}
\begin{bmatrix}
\lpartial_x \\
\lpartial_y \\
\lpartial_z \\
\end{bmatrix} \\
&=
\begin{bmatrix}
\partial_x (A \cos x) + \partial_y(y^2) + \cancel{\partial_z(C x)} \\
\cancel{\partial_x (y^2)} + \partial_y(B \sin y) + \partial_z(z) \\
\partial_x (C x) + \cancel{\partial_y(z)} + \partial_z(z^3)
\end{bmatrix} \\
&=
\begin{bmatrix}
-A \sin x + 2 y \\
B \cos y + 1 \\
C + 3 z^2
\end{bmatrix} \\
\end{aligned}
\end{equation}
%
So, our conditions for equilibrium will be satisfied when we have
\begin{equation}\label{eqn:continuumProblemSet1:550}
\begin{aligned}
A &= \frac{2 y }{\sin x} \\
B &= -\frac{1}{\cos y} \\
C &= -3 z^2,
\end{aligned}
\end{equation}
%
provided \(y \ne 0\), and \(y \ne \pi/2 + n\pi\) for integer \(n\).  If equilibrium is to hold along the \(y = 0\) plane, then we must either also have \(A = 0\) or also impose the restriction \(x = m \pi\) (for integer \(m\)).

} % end answer
