%
% Copyright � 2012 Peeter Joot.  All Rights Reserved.
% Licenced as described in the file LICENSE under the root directory of this GIT repository.
%
\section{Constitutive relation for uniaxial stress.}
\index{constitutive relation}
\index{uniaxial stress}
%
Referring to \cref{fig:continuumL3:continuumL3fig4} and expanding out \eqnref{eqn:continuumL5:110c} we have for the \(1,1\) element of the strain tensor
%Again illustrated in the plane as in \cref{fig:continuumL5:continuumL5fig2}
%\imageFigure{../figures/phy454-continuumechanics/lec5_Uniaxial_stressFig2}{Uniaxial stress.}{fig:continuumL5:continuumL5fig2}{0.2}
%
%2 \mu e_{ij} = \sigma_{ij} - \lambda \frac{\sigma_{kk}}{3 \lambda + 2 \mu} \delta_{ij}.
%
\begin{equation}\label{eqn:continuumL5:290}
\Bsigma =
\begin{bmatrix}
\sigma_{11} & 0 & 0\\
0 & 0 & 0 \\
0 & 0 & 0
\end{bmatrix},
\end{equation}
%
and
\begin{equation}\label{eqn:16constituativeUniaxial:490}
\begin{aligned}
2 \mu e_{11}
&= \sigma_{11} - \frac{\lambda ( \sigma_{11} + \cancel{\sigma_{22}} ) }{3 \lambda + 2 \mu} \\
&= \sigma_{11} \frac{3 \lambda + 2 \mu - \lambda }{3 \lambda + 2 \mu} \\
&= 2 \sigma_{11} \frac{\lambda + \mu }{3 \lambda + 2 \mu},
\end{aligned}
\end{equation}
%
or
%
\begin{equation}\label{eqn:continuumL5:310}
%\frac{e_{11}}{\sigma_{11}} = \frac{\lambda + \mu }{\mu(3 \lambda + 2 \mu)}
\frac{\sigma_{11}}{e_{11}} = \frac{\mu(3 \lambda + 2 \mu)}{\lambda + \mu } = E,
\end{equation}
%
where \(E\) is Young's modulus.  Young's modulus in the text (5.3) is given in terms of the bulk modulus \(K\).  Using \(\lambda = K - 2\mu/3\) we find
%
\begin{equation}\label{eqn:16constituativeUniaxial:510}
\begin{aligned}
E
&=
\frac{\mu(3 \lambda + 2 \mu)}{\lambda + \mu } \\
&=
\frac{\mu(3 (K - 2\mu/3)+ 2 \mu)}{K - 2\mu/3 + \mu } \\
&=
\frac{3 K \mu}{ K + \mu/3 }.
\end{aligned}
\end{equation}
That is
%
\boxedEquation{eqn:continuumL5:330}{
E =
\frac{\mu(3 \lambda + 2 \mu)}{\lambda + \mu } =
\frac{9 K \mu}{ 3 K + \mu }.
}
%
\FIXME{random: there is no \cref{fig:continuumL5:continuumL5fig3} reference that I can find?
%
\imageFigure{../figures/phy454-continuumechanics/lec5_stress_associated_with_Youngs_modulusFig3}{Stress associated with Young's modulus.}{fig:continuumL5:continuumL5fig3}{0.2}
}
We define Poisson's ratio \(\nu\) as the quantity
%
\begin{equation}\label{eqn:continuumL5:350}
\frac{e_{22}}{e_{11}} = \frac{e_{33}}{e_{11}} = - \nu.
\end{equation}
%
Note that we are still talking about uniaxial stress here.  Referring back to \eqnref{eqn:continuumL5:110c} we have
%
\begin{equation}\label{eqn:16constituativeUniaxial:530}
\begin{aligned}
%2 \mu e_{i j} = \sigma_{i j} - \lambda \frac{\sigma_{k k}}{3 \lambda + 2 \mu} \delta_{i j}
2 \mu e_{2 2}
&= \sigma_{2 2} - \lambda \frac{\sigma_{k k}}{3 \lambda + 2 \mu} \delta_{2 2} \\
&= \sigma_{2 2} - \lambda \frac{\sigma_{k k}}{3 \lambda + 2 \mu} \\
&= - \frac{\lambda \sigma_{11}}{3 \lambda + 2 \mu}.
\end{aligned}
\end{equation}
%
Recall \eqnref{eqn:continuumL5:310} that we had
%
\begin{equation}\label{eqn:continuumL5:370}
\sigma_{11} = \frac{\mu (3 \lambda + 2 \mu)}{\lambda + \mu} e_{11}.
\end{equation}
%
Inserting this gives us
\begin{equation}\label{eqn:16constituativeUniaxial:550}
\begin{aligned}
2 \mu e_{22}
%&= - \frac{\lambda}{3 \lambda + 2 \mu} \frac{ \sigma_{11}}{e_{11}} e_{11} \\
= - \frac{\lambda}{\cancel{3 \lambda + 2 \mu}} \frac{ \mu (\cancel{3 \lambda + 2\mu})}{\lambda + \mu} e_{11},
\end{aligned}
\end{equation}
%
so
%\begin{equation}\label{eqn:continuumL5:390}
%\frac{e_{22}}{e_{11}} = -\frac{ \lambda \mu }{2 \mu (\lambda + \mu)}
%\end{equation}
%
\boxedEquation{eqn:continuumL5:410}{
\nu = -\frac{e_{22}}{e_{11}} = \frac{\lambda}{2 (\lambda + \mu)}.
}
%
We can also relate the Poisson's ratio \(\nu\) to the shear modulus \(\mu\) (see the appendix: \ref{chap:appendix:poissonAndShearModulus})
%
\begin{equation}\label{eqn:continuumL5:430}
% prof had:
%\mu = \frac{E}{1 + 4 \nu}
\mu = \frac{E}{2(1 + \nu)},
\end{equation}
%
\begin{equation}\label{eqn:continuumL5:450}
% prof had:
%\lambda = \frac{2 E \nu}{(1 - 2 \nu)(1 + 4 \mu)}
\lambda = \frac{E \nu}{(1 - 2 \nu)(1 + \nu)},
\end{equation}
%
\begin{equation}\label{eqn:continuumL5:470}
\begin{aligned}
e_{11} &= \inv{E}\left( \sigma_{11} - \nu(\sigma_{22} + \sigma_{33}) \right),\\
e_{22} &= \inv{E}\left( \sigma_{22} - \nu(\sigma_{11} + \sigma_{33}) \right),\\
e_{33} &= \inv{E}\left( \sigma_{33} - \nu(\sigma_{11} + \sigma_{22}) \right).
\end{aligned}
\end{equation}
%
These ones are (5.14) in the text, and are easy enough to verify (not done here).
