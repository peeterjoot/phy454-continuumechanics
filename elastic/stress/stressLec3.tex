%
% Copyright � 2012 Peeter Joot.  All Rights Reserved.
% Licenced as described in the file LICENSE under the root directory of this GIT repository.
%
\section{Force per unit volume}
%
Reading for this section is \S 2 from \citep{landau1960theory}.

We would like to consider a macroscopic model that contains the net effects of all the internal forces in the object as depicted in \cref{fig:continuumL3:continuumL3fig2}
%
\imageFigure{../figures/phy454-continuumechanics/lec3_Internal_forcesFig2}{Internal forces.}{fig:continuumL3:continuumL3fig2}{0.2}
%
We will consider a volume big enough that we will not have to consider the individual atomic interactions, only the average effects of those interactions.  Will look at the force per unit volume on a differential volume element.
The total force on the body is
%
\begin{equation}\label{eqn:continuumL3:230}
\iiint \BF dV,
\end{equation}
%
where \(\BF\) is the force per unit volume.  We will evaluate this by utilizing the divergence theorem.  Recall that this was
%
\begin{equation}\label{eqn:continuumL3:250}
\iiint (\spacegrad \cdot \BA) dV
= \iint \BA \cdot d\Bs
\end{equation}
%
We have a small problem, since we have a non-divergence expression of the force here, and it is not immediately obvious that we can apply the divergence theorem.  We can deal with this by assuming that we can find a vector valued tensor, so that if we take the divergence of this tensor, we end up with the force.  We introduce the vector valued quantity
%
\begin{equation}\label{eqn:continuumL3:270}
\BF = \Be_i \PD{x_k}{\sigma_{ik}},
\end{equation}
%
and then apply the divergence theorem
%
\begin{equation}\label{eqn:continuumL3:290}
\iiint \BF dV
= \iiint \Be_i \PD{x_k}{\sigma_{ik}} d\Bx^3
=
\iint \Be_i \sigma_{ik} ds_k,
\end{equation}
%
where \(ds_k\) is a surface element.  We identify this tensor
%
\begin{equation}\label{eqn:continuumL3:310}
\sigma_{ik} = \frac{\text{Force} \cdot \Be_i}{\text{Unit Area}},
\end{equation}
%
often writing it in matrix form
%
\begin{equation}\label{eqn:continuumL3:350}
\begin{bmatrix}
\sigma_{11} & \sigma_{12} & \sigma_{13} \\
\sigma_{21} & \sigma_{22} & \sigma_{23} \\
\sigma_{31} & \sigma_{32} & \sigma_{33}
\end{bmatrix}.
\end{equation}
%
So, starting with a desire to quantify the force per unit area acting on a body by expressing the components of that force as a set of divergence relations
%
\begin{equation}\label{eqn:revTextContinuumL3:250}
f_i = \PD{x_k}{\sigma_{i k}},
\end{equation}
%
we find that the total force acting on the surface is given by the matrix product of the stress with the triplet of surface area elements
%
\begin{equation}\label{eqn:continuumL3:330}
f_i = \sigma_{ik} ds_k,
\end{equation}
%
as the force on the surface element \(ds_k\).   We have yet to find how the stress tensor can be related to deformations (via strain) and physical parameters such as pressure and the modulus of elasticity.

Unlike the strain, we do not have any expectation that this tensor is symmetric, and identify the diagonal components (no sum) \(\sigma_{i i}\) as quantifying the amount of compressive or contractive force per unit area, whereas the cross terms of the stress tensor introduce shearing deformations in the solid.  Let us attempt to get a feel for this graphically.

% from review text:
%With force balance arguments (the Cauchy tetrahedron) we found that the force per unit area on the solid, for a surface with unit normal pointing into the solid, was
%
%\begin{equation}\label{eqn:revTextContinuumL3:270}
%\Bt = \Be_i t_i = \Be_i \sigma_{ij} n_j.
%\end{equation}
%
\makeexample{Stretch, 2 opposing directions}{ex:stress:uniaxial}{
\imageFigure{../figures/phy454-continuumechanics/lec3_Opposing_stresses_in_one_directionFig4}{Opposing stresses in one direction.}{fig:continuumL3:continuumL3fig4}{0.2}
Here, as illustrated in \cref{fig:continuumL3:continuumL3fig4}, the associated (2D) stress tensor takes the simple form
\begin{equation}\label{eqn:continuumL3:370}
\begin{bmatrix}
\sigma_{11} & 0 \\
0 & 0
\end{bmatrix}.
\end{equation}
This is called \textAndIndex{uniaxial stress}.
}
\makeexample{Stretch, mutually perpendicular directions.}{ex:stress:biaxial}{
For a pair of perpendicular forces applied in two dimensions, as illustrated in \cref{fig:continuumL3:continuumL3fig5}
\imageFigure{../figures/phy454-continuumechanics/lec3_Mutually_perpendicular_forcesFig5}{Mutually perpendicular forces.}{fig:continuumL3:continuumL3fig5}{0.15}
our stress tensor now just takes the form
\begin{equation}\label{eqn:continuumL3:390}
\begin{bmatrix}
\sigma_{11} & 0 \\
0 & \sigma_{22}
\end{bmatrix}.
\end{equation}
This is called \textAndIndex{biaxial stress}.

It is easy to imagine now how to get some more general stress tensors, should we make a change of basis that rotates our frame.
}

\makeexample{Radial stretch}{ex:stress:radial}{
Suppose we have a fire fighter's safety net, used to catch somebody jumping from a burning building (do they ever do that outside of movies?), as in \cref{fig:continuumL3:continuumL3fig6}.  Each of the firefighters contributes to the stretch.
\imageFigure{../figures/phy454-continuumechanics/lec3_Radial_forcesFig6}{Radial forces.}{fig:continuumL3:continuumL3fig6}{0.2}
}
