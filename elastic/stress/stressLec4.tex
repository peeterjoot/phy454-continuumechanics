%
% Copyright � 2012 Peeter Joot.  All Rights Reserved.
% Licenced as described in the file LICENSE under the root directory of this GIT repository.
%
\label{chap:continuumL4}
\section{Stress tensor in 2D}
%
In two dimensions this is illustrated in \cref{fig:continuumL3:continuumL3fig3}
%
\imageFigure{../figures/phy454-continuumechanics/lec3_2D_stress_tensorFig3}{2D stress tensor.}{fig:continuumL3:continuumL3fig3}{0.15}
%
Observe that we use the index \(i\) above as the direction of the force, and index \(k\) as the direction normal to the surface.

We will show later that this tensor is in fact symmetric.
\FIXME{was this done?}

For the stress tensor
%
\begin{equation}\label{eqn:continuumL4:10}
\sigma_{ij},
\end{equation}
%
a second rank tensor, the first index \(i\) defines the direction of the force, and the second index \(j\) defines the surface.

Observe that the dimensions of \(\sigma_{ij}\) is force per unit area, just like pressure.  We will in fact show that this tensor is akin to the pressure, and the diagonalized components of this tensor represent the pressure.

We have illustrated the stress tensor in a couple of 2D examples.  The first we call \textAndIndex{uniaxial stress}, having just the \(1,1\) element of the matrix as illustrated in \cref{fig:continuumL3:continuumL3fig4}.
%\cref{fig:continuumL4:continuumL4fig1}
%\imageFigure{../figures/phy454-continuumechanics/lec4_Uniaxial_stressFig1}{Uniaxial stress.}{fig:continuumL4:continuumL4fig1}{0.3}
%
\begin{equation}\label{eqn:continuumL4:30}
\sigma =
\begin{bmatrix}
\sigma_{11} & 0 \\
0 & 0
\end{bmatrix}.
\end{equation}
%
A \textAndIndex{biaxial stress} is illustrated in \cref{fig:continuumL3:continuumL3fig5}.
%\cref{fig:continuumL4:continuumL4fig2}
%\imageFigure{../figures/phy454-continuumechanics/lec4_Biaxial_stressFig2}{Biaxial stress.}{fig:continuumL4:continuumL4fig2}{0.3}
%
where for \(\sigma_{11} \ne \sigma_{22}\) our tensor takes the form
%
\begin{equation}\label{eqn:continuumL4:50}
\sigma =
\begin{bmatrix}
\sigma_{11} & 0 \\
0 & \sigma_{22}
\end{bmatrix}.
\end{equation}
%
In the general case we have
%
\begin{equation}\label{eqn:continuumL4:70}
\sigma =
\begin{bmatrix}
\sigma_{11} & \sigma_{12} \\
\sigma_{21} & \sigma_{22}
\end{bmatrix}.
\end{equation}
%
We can attempt to illustrate this, but it becomes much harder to visualize as shown in \cref{fig:continuumL4:continuumL4fig3}
\imageFigure{../figures/phy454-continuumechanics/lec4_General_stressFig3}{General stress.}{fig:continuumL4:continuumL4fig3}{0.2}
%
In equilibrium we must have
%
\begin{equation}\label{eqn:continuumL4:90}
\sigma_{12} = \sigma_{21}.
\end{equation}
%
We can use similar arguments to show that the stress tensor is symmetric.
\FIXME{perhaps there was a verbal argument in class here.  This is not a sensible explanation of the symmetry requirement as is}
%
%\section{Diagonalization}
We will look at the two dimensional case in some detail, as in \cref{fig:continuumL4:continuumL4fig6}
%
\imageFigure{../figures/phy454-continuumechanics/lec4_Area_element_under_stress_with_and_without_rotationFig6}{Area element under stress with and without rotation.}{fig:continuumL4:continuumL4fig6}{0.25}
%
Under this coordinate transformation, a rotation, the diagonal stress tensor is taken to a non-diagonal form \index{diagonalization}
%
\begin{equation}\label{eqn:continuumL4:130}
\begin{bmatrix}
\sigma_{11} & 0 \\
0 & \sigma_{22}
\end{bmatrix}
\leftrightarrow
\begin{bmatrix}
\sigma_{11}' & \sigma_{12}' \\
\sigma_{21}' & \sigma_{22}'
\end{bmatrix}
\end{equation}
