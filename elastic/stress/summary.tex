%
% Copyright � 2012 Peeter Joot.  All Rights Reserved.
% Licenced as described in the file LICENSE under the root directory of this GIT repository.
%
\section{Summary}
\subsection{Stress tensor}

We sought and found a representation of the force per unit area acting on a body by expressing the components of that force as a set of divergence relations
%
\begin{equation}\label{eqn:continuumElasticityReview:250}
f_i = \partial_k \sigma_{i k},
\end{equation}
%
and call the associated tensor \(\sigma_{ij}\) the \textit{stress}.

Unlike the strain, we do not have any expectation that this tensor is symmetric, and identify the diagonal components (no sum) \(\sigma_{i i}\) as quantifying the amount of compressive or contractive force per unit area, whereas the cross terms of the stress tensor introduce shearing deformations in the solid.

With force balance arguments (the Cauchy tetrahedron) we found that the force per unit area on the solid, for a surface with unit normal pointing into the solid, was
%
\begin{equation}\label{eqn:continuumElasticityReview:270}
\Btau = \Be_i \tau_i = \Be_i \sigma_{ij} n_j.
\end{equation}
%
\subsection{Constitutive relation}

In the scope of this course we considered only Newtonian materials, those for which the stress and strain tensors are linearly related
%
\begin{equation}\label{eqn:continuumElasticityReview:290}
\sigma_{ij} = c_{ijkl} e_{kl},
\end{equation}
%
and further restricted our attention to isotropic materials, which can be shown to have the form
%
\begin{equation}\label{eqn:continuumElasticityReview:310}
\sigma_{ij} = \lambda e_{kk} \delta_{ij} + 2 \mu e_{ij},
\end{equation}
%
where \(\lambda\) and \(\mu\) are the Lame parameters and \(\mu\) is called the shear modulus (and viscosity in the context of fluids).

By computing the trace of the stress \(\sigma_{ii}\) we can invert this to find
%
\begin{equation}\label{eqn:continuumElasticityReview:330}
2 \mu e_{ij} = \sigma_{ij} - \frac{\lambda}{3 \lambda + 2 \mu} \sigma_{kk} \delta_{ij}.
\end{equation}
%
\subsection{Uniform hydrostatic compression}

With only normal components of the stress (no shear), and the stress having the same value in all directions, we find
%
\begin{equation}\label{eqn:continuumElasticityReview:350}
\sigma_{ij} = ( 3 \lambda + 2 \mu ) e_{ij},
\end{equation}
%
and identify this combination \(-3 \lambda - 2 \mu\) as the pressure, linearly relating the stress and strain tensors
%
\begin{equation}\label{eqn:continuumElasticityReview:370}
\sigma_{ij} = -p e_{ij}.
\end{equation}
%
With \(e_{ii} = (dV' - dV)/dV = \Delta V/V\), we formed the Bulk modulus \(K\) with the value
%
\begin{equation}\label{eqn:continuumElasticityReview:390}
K = \left( \lambda + \frac{2 \mu}{3} \right) = -\frac{p V}{\Delta V}.
\end{equation}
%
\subsection{Uniaxial stress.  Young's modulus.  Poisson's ratio}

For the special case with only one non-zero stress component (we used \(\sigma_{11}\)) we were able to compute Young's modulus \(E\), the ratio between stress and strain in that direction
%
\begin{equation}\label{eqn:continuumElasticityReview:410}
E = \frac{\sigma_{11}}{e_{11}} = \frac{\mu(3 \lambda + 2 \mu)}{\lambda + \mu }  = \frac{3 K \mu}{K + \mu/3}.
\end{equation}
%
Just because only one component of the stress is non-zero, does not mean that we have no deformation in any other directions.  Introducing Poisson's ratio \(\nu\) in terms of the ratio of the strains relative to the strain in the direction of the force we write and then subsequently found
%
\begin{equation}\label{eqn:continuumElasticityReview:430}
\nu = -\frac{e_{22}}{e_{11}} = -\frac{e_{33}}{e_{11}} = \frac{\lambda}{2(\lambda + \mu)}.
\end{equation}
%
We were also able to find

We can also relate the Poisson's ratio \(\nu\) to the shear modulus \(\mu\)
%
\begin{equation}\label{eqn:continuumElasticityReview:450}
\mu = \frac{E}{2(1 + \nu)}
\end{equation}
%
\begin{equation}\label{eqn:continuumElasticityReview:470}
\lambda = \frac{E \nu}{(1 - 2 \nu)(1 + \nu)}
\end{equation}
%
\begin{align}\label{eqn:continuumElasticityReview:490}
e_{11} &= \inv{E}\left( \sigma_{11} - \nu(\sigma_{22} + \sigma_{33}) \right) \\
e_{22} &= \inv{E}\left( \sigma_{22} - \nu(\sigma_{11} + \sigma_{33}) \right) \\
e_{33} &= \inv{E}\left( \sigma_{33} - \nu(\sigma_{11} + \sigma_{22}) \right)
\end{align}

