%
% Copyright � 2012 Peeter Joot.  All Rights Reserved.
% Licenced as described in the file LICENSE under the root directory of this GIT repository.
%
% total dup material: omit it all:
%%%\section{Summary.}
%%%\subsection{Strain Tensor.}
%%%%
%%%Identifying a point in a solid with coordinates \(x_i\) and the coordinates of that portion of the solid after displacement, we formed the difference as a measure of the displacement
%%%%
%%%\begin{equation}\label{eqn:continuumElasticityReview:10}
%%%u_i = x_i' - x_i.
%%%\end{equation}
%%%%
%%%With \(du_i = \PDi{x_j}{u_i} dx_j\), we computed the difference in length (squared) for an element of the displaced solid and found
%%%%
%%%\begin{equation}\label{eqn:continuumElasticityReview:30}
%%%dx_k' dx_k' - dx_k dx_k =
%%%\left(
%%%\PD{x_i}{u_j} +
%%%\PD{x_j}{u_i} +
%%%\PD{x_i}{u_k}
%%%\PD{x_j}{u_k}
%%%\right)
%%%dx_i dx_j,
%%%\end{equation}
%%%%
%%%or defining the \textit{strain tensor} \(e_{ij}\), we have
%%%%
%%%\begin{subequations}
%%%\begin{equation}\label{eqn:continuumElasticityReview:50}
%%%(d\Bx')^2 - (d\Bx)^2
%%%= 2 e_{ij} dx_i dx_j,
%%%\end{equation}
%%%\begin{equation}\label{eqn:continuumElasticityReview:70}
%%%e_{ij}
%%%=
%%%\inv{2}
%%%\left(
%%%\PD{x_i}{u_j} +
%%%\PD{x_j}{u_i} +
%%%\PD{x_i}{u_k}
%%%\PD{x_j}{u_k}
%%%\right).
%%%\end{equation}
%%%\end{subequations}
%%%%
%%%In this course we use only the linear terms and write
%%%%
%%%\begin{equation}\label{eqn:continuumElasticityReview:90}
%%%e_{ij}
%%%=
%%%\inv{2}
%%%\left(
%%%\PD{x_i}{u_j} +
%%%\PD{x_j}{u_i}
%%%\right).
%%%\end{equation}
%%%%
%%%\FIXME{Unresolved: Relating displacement and position by strain
%%%
%%%In \citep{feynman1963flp:elasticMaterials} it is pointed out that this strain tensor simply relates the displacement vector coordinates \(u_i\) to the coordinates at the point at which it is measured
%%%%
%%%\begin{equation}\label{eqn:continuumElasticityReview:110}
%%%u_i = e_{ij} x_j.
%%%\end{equation}
%%%%
%%%When we get to fluid dynamics we perform a linear expansion of \(du_i\) and find something similar
%%%%
%%%\begin{equation}\label{eqn:continuumElasticityReview:530}
%%%dx_i' - dx_i = du_i = \PD{x_k}{u_i} dx_k = e_{ij} dx_k + \omega_{ij} dx_k
%%%\end{equation}
%%%%
%%%where
%%%%
%%%\begin{equation}\label{eqn:continuumElasticityReview:550}
%%%\omega_{ij} = \inv{2} \left( \PD{x_i}{u_j} +\PD{x_j}{u_i} \right).
%%%\end{equation}
%%%%
%%%Except for the antisymmetric term, note the structural similarity of \cref{eqn:continuumElasticityReview:110} and \cref{eqn:continuumElasticityReview:530}.  Why is it that we neglect the vorticity tensor in statics?
%%%%  If we are approximating the displacement, it appears to have a natural place in things, as we can see
%%%%
%%%%\begin{align*}
%%%%u_i
%%%%&\approx \PD{x_j}{u_i} x_j \\
%%%%&=
%%%%e_{ij} x_j + \omega_{ik} x_k.
%%%%\end{align*}
%%%%
%%%%This is easily seen to be the case, recovering \cref{eqn:continuumElasticityReview:90} by taking derivatives of \cref{eqn:continuumElasticityReview:110}, plus an assumption that \(e_{ij}\) is symmetric.
%%%}
%%%%
%%%\subsection{Diagonal strain representation.}
%%%%
%%%In a basis for which the strain tensor is diagonal, it was pointed out that we can write our difference in squared displacement as (for \(k = 1, 2, 3\), no summation convention)
%%%%
%%%\begin{equation}\label{eqn:continuumElasticityReview:130}
%%%(dx_k')^2 - (dx_k)^2 = 2 e_{kk} dx_k dx_k,
%%%\end{equation}
%%%%
%%%from which we can rearrange, take roots, and apply a first order Taylor expansion to find (again no summation convention)
%%%%
%%%\begin{equation}\label{eqn:continuumElasticityReview:150}
%%%dx_k' \approx (1 + e_{kk}) dx_k.
%%%\end{equation}
%%%%
%%%An approximation of the displaced volume was then found in terms of the strain tensor trace (summation convention back again)
%%%%
%%%\begin{equation}\label{eqn:continuumElasticityReview:170}
%%%dV' \approx (1 + e_{kk}) dV,
%%%\end{equation}
%%%%
%%%allowing us to identify this trace as a relative difference in displaced volume
%%%%
%%%\begin{equation}\label{eqn:continuumElasticityReview:190}
%%%e_{kk} \approx \frac{dV' - dV}{dV}.
%%%\end{equation}
%%%%
%%%\subsection{Strain in cylindrical coordinates.}
%%%\index{strain}
%%%\index{cylindrical coordinates}
%%%%
%%%Useful in many practice problems are the cylindrical coordinate representation of the strain tensor
%%%%
%%%\begin{equation}\label{eqn:continuumElasticityReview:210}
%%%\begin{aligned}
%%%2 e_{rr} &= \PD{r}{u_r}  \\
%%%2 e_{\phi\phi} &= \inv{r} \PD{\phi}{u_\phi} +\inv{r} u_r  \\
%%%2 e_{zz} &= \PD{z}{u_z}  \\
%%%2 e_{zr} &= \PD{z}{u_r} + \PD{r}{u_z} \\
%%%2 e_{r\phi} &= \PD{r}{u_\phi} - \inv{r} u_\phi + \inv{r} \PD{\phi}{u_r} \\
%%%2 e_{\phi z} &= \PD{z}{u_\phi} +\inv{r} \PD{\phi}{u_z}.
%%%\end{aligned}
%%%\end{equation}
%%%%
%%%This can be found in \citep{landau1960theory}.  It was not derived there or in class, but is not too hard, even using the second order methods we used for the Cartesian form of the tensor.
%%%
%%%An easier way to do this derivation (and understand what the coordinates represent) follows from the relation found in \S 6 of \citep{acheson1990elementary}
%%%%
%%%\begin{equation}\label{eqn:continuumElasticityReview:230}
%%%2 \Be_i e_{ij} n_j = 2 (\ncap \cdot \spacegrad) \Bu + \ncap \cross (\spacegrad \cross \Bu),
%%%\end{equation}
%%%%
%%%where \(\ncap\) is the normal to the surface at which we are measuring a force applied to the solid (our Cauchy tetrahedron).
%%%
%%%The cylindrical tensor coordinates of \cref{eqn:continuumElasticityReview:210} follow from
%%%\cref{eqn:continuumElasticityReview:230} nicely taking \(\ncap = \rcap, \phicap, \zcap\) in turn.
%%%%
%%%\subsection{Compatibility condition.}
%%%%
%%%For a 2D strain tensor we found an interrelationship between the components of the strain tensor
%%%%
%%%\begin{equation}\label{eqn:continuumElasticityReview:510}
%%%2 \frac{\partial^2 e_{12}}{\partial x_1 \partial x_2}
%%%=
%%%\PDSq{x_1}{e_{22}}
%%%+\PDSq{x_2}{e_{11}},
%%%\end{equation}
%%%%
%%%and called this the compatibility condition.  It was claimed, but not demonstrated that this is what is required to ensure a deformation maintained a coherent solid geometry.
%%%
%%%I was not able to find any references to this compatibility condition in any of the texts I have, but found \citep{wiki:compatibilityMechanics}, \citep{wiki:infinitesimalStrainTheory}, and \citep{wiki:saintVenantCompat}.  It is not terribly surprising to see Christoffel symbol and differential forms references on those pages, since one can imagine that we would wish to look at the mappings of all the points in the object as it undergoes the transformation from the original to the deformed state.
%%%
%%%Even with just three points in a plane, say \(\Ba\), \(\Bb\), \(\Bc\), the general deformation of an object does not seem like it is the easiest thing to describe.  We can imagine that these have trajectories in the deformation process \(\Ba = \Ba(\alpha\), \(\Bb = \Bb(\beta)\), \(\Bc = \Bc(\gamma)\), with \(\Ba', \Bb', \Bc'\) at the end points of the trajectories.  We would want to look at displacement vectors \(\Bu_a, \Bu_b, \Bu_c\) along each of these trajectories, and then see how they must be related.  Doing that carefully must result in this compatibility condition.
%%%
