%
% Copyright � 2012 Peeter Joot.  All Rights Reserved.
% Licenced as described in the file LICENSE under the root directory of this GIT repository.
%

%
%
\makeoproblem{Strain tensor, small displacement}{problem:strain:ps1q2a}
{2012 ps1, p2 a}
{
%\ExePart*[]
%
Small displacement field in a material is given by
\begin{equation}\label{eqn:continuumProblemSet1:30}
\begin{aligned}
e_1 &= 2 x_1 x_2 \\
e_2 &= x_3^2 \\
e_3 &= x_1^2 - x_3.
\end{aligned}
\end{equation}
Find
\makesubproblem{Infinitesimal strain tensor \(e_{ij}\)}{problem:strain:ps1q2a1}
\makesubproblem{Principal strains and axes at \((x_1, x_2, x_3) = (1, 2, 4)\)}{problem:strain:ps1q2a2}
} % makeoproblem

%Solution: \ref{solutions:ps1q2a}.
\makeproblem{Computing stretch in any given direction.}{problem:strain:todo}{
In class, it was stated ``How do we use the strain tensor?  Strain is the measure of stretching, so given a strain tensor, we should be able to compute the stretch in any given direction.''.
%PLACEHOLDER: find or create problem and try this.
%\imageFigure{../figures/phy454-continuumechanics/lec3_Stretched_line_elementsFig1}{Stretched line elements.}{fig:continuumL3:continuumL3fig1}{0.2}
\FIXME{Related to the following question created while reviewing for the exam
In \citep{feynman1963flp:elasticMaterials} it is pointed out that this strain tensor simply relates the displacement vector coordinates \(u_i\) to the coordinates at the point at which it is measured
\begin{equation}\label{eqn:strainProblems:110}
u_i = e_{ij} x_j.
\end{equation}
When we get to fluid dynamics we perform a linear expansion of \(du_i\) and find something similar
\begin{equation}\label{eqn:strainProblems:530}
dx_i' - dx_i = du_i = \PD{x_k}{u_i} dx_k = e_{ij} dx_k + \omega_{ij} dx_k
\end{equation}
where
\begin{equation}\label{eqn:strainProblems:550}
\omega_{ij} = \inv{2} \left( \PD{x_i}{u_j} +\PD{x_j}{u_i} \right).
\end{equation}
%
Except for the antisymmetric term, note the structural similarity of \eqnref{eqn:strainProblems:110} and \eqnref{eqn:strainProblems:530}.  Why is it that we neglect the vorticity tensor in statics?
%  If we are approximating the displacement, it appears to have a natural place in things, as we can see
%
%\begin{align*}
%u_i
%&\approx \PD{x_j}{u_i} x_j \\
%&=
%e_{ij} x_j + \omega_{ik} x_k.
%\end{align*}
%
%This is easily seen to be the case, recovering \eqnref{eqn:strainProblems:90} by taking derivatives of \eqnref{eqn:strainProblems:110}, plus an assumption that \(e_{ij}\) is symmetric.
}
} % makeproblem.
\makeproblem{Derive the 3D compatibility conditions.}{problem:strainFIXME:todo3d}{
PLACEHOLDER.  We would been promised some homework for this, presumably with hints that walked us through the complexity of the proof, but that never materialized.
} % makeproblem
\makeanswer{problem:strain:ps1q2a}{
\makesubanswer{For the infinitesimal strain tensor \(e_{ij}\), we have}{problem:strain:ps1q2a1}
%
\begin{equation}\label{eqn:problems:570}
\begin{aligned}
e_{11}
&= \PD{x_1}{e_1} \\
&= \PD{x_1}{}2 x_1 x_2 \\
&= 2 x_2
\end{aligned}
\end{equation}
%
\begin{equation}\label{eqn:problems:590}
\begin{aligned}
e_{22}
&= \PD{x_2}{e_2} \\
&= \PD{x_2}{} x_3^2 \\
&= 0
\end{aligned}
\end{equation}
%
\begin{equation}\label{eqn:problems:610}
\begin{aligned}
e_{33}
&= \PD{x_3}{e_3} \\
&= \PD{x_3}{} ( x_1^2 - x_3 ) \\
&= -1
\end{aligned}
\end{equation}
%
\begin{equation}\label{eqn:problems:630}
\begin{aligned}
e_{12}
&=
\inv{2} \left(
\PD{x_1}{e_2}
+
\PD{x_2}{e_1}
\right) \\
&=
\inv{2}
\left(
\cancel{\PD{x_1}{} x_3^2 }
+
\PD{x_2}{} 2 x_1 x_2
\right) \\
&=
x_1
\end{aligned}
\end{equation}
%
\begin{equation}\label{eqn:problems:650}
\begin{aligned}
e_{23}
&=
\inv{2} \left(
\PD{x_2}{e_3}
+
\PD{x_3}{e_2}
\right) \\
&=
\inv{2}
\left(
\cancel{\PD{x_2}{} (x_1^2 - x_3 )}
+
\PD{x_3}{} x_3^2
\right) \\
&=
x_3
\end{aligned}
\end{equation}
%
\begin{equation}\label{eqn:problems:670}
\begin{aligned}
e_{31}
&=
\inv{2} \left(
\PD{x_3}{e_1}
+
\PD{x_1}{e_3}
\right) \\
&=
\inv{2}
\left(
\cancel{\PD{x_3}{} 2 x_1 x_2 }
+
\PD{x_1}{} (x_1^2 - x_3 )
\right) \\
&=
x_1
\end{aligned}
\end{equation}
%
In matrix form we have
%
\begin{equation}\label{eqn:continuumProblemSet1:250}
\Be =
\begin{bmatrix}
2 x_2 & x_1 & x_1 \\
x_1 & 0 & x_3 \\
x_1 & x_3 & -1 \\
\end{bmatrix}
\end{equation}
%
\makesubanswer{For the principle strains and axes}{problem:strain:ps1q2a2}

At the point \((1, 2, 4)\) the strain tensor has the value
%
\begin{equation}\label{eqn:continuumProblemSet1:270}
\Be =
\begin{bmatrix}
4 & 1 & 1 \\
1 & 0 & 4 \\
1 & 4 & -1
\end{bmatrix}.
\end{equation}
%
We wish to diagonalize this, solving the characteristic equation for the eigenvalues \(\lambda\)
%
\begin{equation}\label{eqn:problems:690}
\begin{aligned}
0 &=
\begin{vmatrix}
4 -\lambda & 1 & 1 \\
1 & -\lambda & 4 \\
1 & 4 & -1 -\lambda
\end{vmatrix} \\
&=
(4 -\lambda )
\begin{vmatrix}
 -\lambda & 4 \\
 4 & -1 -\lambda
\end{vmatrix}
-
\begin{vmatrix}
1 & 1 \\
4 & -1 -\lambda
\end{vmatrix}
+
\begin{vmatrix}
1 & 1 \\
-\lambda & 4 \\
\end{vmatrix} \\
&=
(4 - \lambda)(\lambda^2 + \lambda - 16)
-(-1 -\lambda - 4)
+(4 + \lambda) \\
\end{aligned}
\end{equation}
%
We find the characteristic equation to be
%
\begin{equation}\label{eqn:continuumProblemSet1:290}
0 = -\lambda^3 + 3 \lambda^2 + 22\lambda - 55.
\end{equation}
%
This does not appear to lend itself easily to manual solution (there are no obvious roots to factor out).  As expected, since the matrix is symmetric, a plot \cref{fig:continuumL8:continuumProblemSet1Q2fig1} shows that all our roots are real.
%
\imageFigure{../figures/phy454-continuumechanics/problemSet1Q2__Characteristic_equationFig1}{Q2.  Characteristic equation.}{fig:continuumL8:continuumProblemSet1Q2fig1}{0.2}
%
Numerically, we determine these roots to be
%
\begin{equation}\label{eqn:continuumProblemSet1:310}
\{5.19684, -4.53206, 2.33522\}
\end{equation}
%
with the corresponding basis (orthonormal eigenvectors), the principle axes are
%
\begin{equation}\label{eqn:continuumProblemSet1:330}
\left\{
\pcap_1,
\pcap_2,
\pcap_3
\right\}
=
\left\{
\begin{bmatrix}
0.76291 \\
0.480082 \\
0.433001
\end{bmatrix},
\begin{bmatrix}
-0.010606 \\
-0.660372 \\
0.750863
\end{bmatrix},
\begin{bmatrix}
-0.646418 \\
0.577433 \\
0.498713
\end{bmatrix}
\right\}.
\end{equation}
%
} % end answer

\makeoproblem{Constitutive relation}{problem:fluids:midterm:1b}
{2012 midterm, p1 b}
{
In continuum mechanics what is meant by the \textit{constitutive relation}?
} % makeoproblem

\makeanswer{problem:fluids:midterm:1b}{
The constitutive relation is the stress-strain relation, generally
%
\begin{equation}\label{eqn:continuumMidTermReflection:150}
\sigma_{ij} = c_{abij} e_{ab}
\end{equation}
%
for isotropic solids we model this as
%
\begin{equation}\label{eqn:continuumMidTermReflection:170}
\sigma_{ij} = \lambda e_{kk} \delta_{ij} + 2 \mu e_{ij}
\end{equation}
%
and for Newtonian fluids
%
\begin{equation}\label{eqn:continuumMidTermReflection:190}
\sigma_{ij} = -p \delta_{ij} + 2 \mu e_{ij}
\end{equation}
} % end answer

