%
% Copyright � 2012 Peeter Joot.  All Rights Reserved.
% Licenced as described in the file LICENSE under the root directory of this GIT repository.
%

%
%

\section{Compatibility condition compatibility condition for 2D strain} \index{compatibility condition}

\begin{equation}\label{eqn:continuumL6:50}
e_{ij} =
\begin{bmatrix}
e_{11} & e_{12} \\
e_{21} & e_{22}
\end{bmatrix}
\end{equation}

From \eqnref{eqn:revTextcontinuumL2:90} we see that we have

\begin{equation}\label{eqn:continuumL6:70}
\begin{aligned}
e_{11} &= \PD{x_1}{e_1} \\
e_{22} &= \PD{x_2}{e_2} \\
e_{12} = e_{21} &=
\inv{2} \left(
\PD{x_1}{e_2}
+ \PD{x_2}{e_1}
\right).
\end{aligned}
\end{equation}

We have a relationship between these displacements (called the compatibility relationship), which is

\boxedEquation{eqn:continuumL6:110}{
\PDSq{x_2}{e_{11}} +
\PDSq{x_1}{e_{22}} =
2
\frac{\partial^2 e_{12}}{\partial x_1 \partial x_2}.
}

We find this by straight computation

\begin{equation}\label{eqn:4compatibility:149}
\begin{aligned}
\PDSq{x_2}{e_{11}}
&=
\PDSq{x_2}{}\left(
\PD{x_1}{e_1}
\right) \\
&=
\frac{\partial^3 e_1}{\partial x_1 \partial x_2^2},
\end{aligned}
\end{equation}

and

\begin{equation}\label{eqn:4compatibility:169}
\begin{aligned}
\PDSq{x_1}{e_{22}}
&=
\PDSq{x_1}{}\left(
\PD{x_2}{e_2}
\right) \\
&=
\frac{\partial^3 e_2}{\partial x_2 \partial x_1^2},
\end{aligned}
\end{equation}

Now, looking at the cross term we find

\begin{equation}\label{eqn:4compatibility:189}
\begin{aligned}
2 \frac{\partial^2 e_{12}}{\partial x_1 \partial x_2}
&=
\frac{\partial^2 e_{12}}{\partial x_1 \partial x_2}
\left(
\PD{x_1}{e_2}
+ \PD{x_2}{e_1}
\right) \\
&=
\left(
\frac{\partial^3 e_1}{\partial x_1 \partial x_2^2}
+
\frac{\partial^3 e_2}{\partial x_2 \partial x_1^2}
\right)
\end{aligned}
\end{equation}

We have found an interrelationship between the components of the strain

\boxedEquation{eqn:continuumL6:129}{
2 \frac{\partial^2 e_{12}}{\partial x_1 \partial x_2}
=
\PDSq{x_1}{e_{22}}
+\PDSq{x_2}{e_{11}}.
}

This relationship is called the \textit{compatibility condition}, and ensures that we do not have a disjoint deformation of the form in \cref{fig:continuumL6:continuumL6fig1}.

\imageFigure{../../figures/phy454/lec6_disjoint_deformation_illustratedFig1}{disjoint deformation illustrated}{fig:continuumL6:continuumL6fig1}{0.3}

I went looking for something to substantiate the claim that the compatibility condition \eqnref{eqn:continuumL6:129} is what is required to ensure a deformation maintained a coherent solid geometry.  I was not able to find any references to this compatibility condition in any of the texts I have, but found \citep{wiki:compatibilityMechanics}, \citep{wiki:infinitesimalStrainTheory}, and \citep{wiki:saintVenantCompat}.  It is not terribly surprising to see Christoffel symbol and differential forms references on those pages, since one can imagine that we would wish to look at the mappings of all the points in the object as it undergoes the transformation from the original to the deformed state.

Even with just three points in a plane, say \(\Ba\), \(\Bb\), \(\Bc\), the general deformation of an object does not seem like it is the easiest thing to describe.  We can imagine that these have trajectories in the deformation process \(\Ba = \Ba(\alpha\), \(\Bb = \Bb(\beta)\), \(\Bc = \Bc(\gamma)\), with \(\Ba', \Bb', \Bc'\) at the end points of the trajectories.  We would want to look at displacement vectors \(\Bu_a, \Bu_b, \Bu_c\) along each of these trajectories, and then see how they must be related.  Doing that carefully must result in this compatibility condition.

\section{Compatibility condition for 3D strain}

While we have 9 components in the tensor, not all of these are independent.  The sets above and below the diagonal can be related.  It can be shown that there are 6 relationships between the components of the general three dimensional strain tensor \(e_{ij}\).
%, as illustrated in \cref{fig:continuumL6:continuumL6fig2}.
%
%\imageFigure{../../figures/phy454/continuumL6fig2}{continuumL6fig2}{fig:continuumL6:continuumL6fig2}{0.2}


