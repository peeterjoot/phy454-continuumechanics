%
% Copyright � 2012 Peeter Joot.  All Rights Reserved.
% Licenced as described in the file LICENSE under the root directory of this GIT repository.
%

%
%
%\chapter{PHY454H1S Continuum Mechanics.  Lecture 13: Hydrostatics.  Surface normals and tangent vectors.  Taught by Prof. K. Das}
\label{chap:continuumL13}
%
\section{Steady state and static fluids.}
%
Consider a sample volume of water, not moving with respect to the rest of the surrounding water.  If it is not moving the forces must be in balance.  What are the forces acting on this bit of fluid, considering a cylinder of the fluid above it as in \cref{fig:continuumL13:continuumL13Fig3a}
\imageFigure{../figures/phy454-continuumechanics/lec13_A_control_volume_of_fluid_in_a_fluidFig3a}{A control volume of fluid in a fluid.}{fig:continuumL13:continuumL13Fig3a}{0.2}
%
In the column of fluid above the control volume \cref{fig:continuumL13:continuumL13Fig3b} we have
\imageFigure{../figures/phy454-continuumechanics/lec13_column_of_fluid_above_a_control_volumeFig3b}{Column of fluid above a control volume.}{fig:continuumL13:continuumL13Fig3b}{0.2}
%
\begin{equation}\label{eqn:continuumL13:20}
h A_w \rho g + p_A A_w = p_w A_w
\end{equation}
%
so
\begin{equation}\label{eqn:continuumL13:40}
p_w = h \rho g + p_A
\end{equation}
%
If we were to replace this blob of water with something of equal density, it should not change the dynamics (or statics) of the situations and that would not move.

We call this the
%
\makedefinition{Buoyancy force}{dfn:continuumL13:60}{Buoyancy force = weight of the equivalent volume of water - weight of the foreign body. \index{buoyancy force}}

If the densities are not equal, then we would have motion of the new bit of mass as depicted in \cref{fig:continuumL13:continuumL13Fig4}
\imageFigure{../figures/phy454-continuumechanics/lec13_A_mass_of_different_density_in_a_fluidFig4}{A mass of different density in a fluid.}{fig:continuumL13:continuumL13Fig4}{0.2}
%
Consider a volume of ice floating on the surface of water, one with solid ice and one with partially frozen ice (with water or air or dirt or an anchor or anything else in it) as in \cref{fig:continuumL13:continuumL13Fig5}
%
\imageFigure{../figures/phy454-continuumechanics/lec13_Various_floating_ice_configurations_on_waterFig5}{Various floating ice configurations on water.}{fig:continuumL13:continuumL13Fig5}{0.2}
%
No matter the situation, the water level will not change if the ice melts, because the total weight of the displaced water must have been matched by the weight of the unmelted ice plus additives.

Now what happens when we have fluid flows?  Consider \cref{fig:continuumL13:continuumL13Fig6}
\imageFigure{../figures/phy454-continuumechanics/lec13_flow_through_channel_with_different_aperturesFig6}{Flow through channel with different apertures.}{fig:continuumL13:continuumL13Fig6}{0.1}
%
Conservation of mass is going to mean that the masses of fluid flowing through any pair of cross sections will have to be equal
%
\begin{equation}\label{eqn:continuumL13:80}
\rho_1 A_1 v_1 = \rho_2 A_2 v_2,
\end{equation}
%
With incompressible fluids (\(\rho = \rho_1 = \rho_2\)) we have
%
\begin{equation}\label{eqn:continuumL13:100}
A_1 v_1 = A_2 v_2,
\end{equation}
%
so that if
%
\begin{equation}\label{eqn:continuumL13:120}
A_1 > A_2,
\end{equation}
%
we must have
\begin{equation}\label{eqn:continuumL13:140}
v_1 < v_2,
\end{equation}
%
to balance this.

In class this was illustrated with a pair of computer animations, one showing the deformation of patches of the fluid, and another showing how the velocities vary through the channel.  This is crudely depicted in \cref{fig:continuumL13:continuumL13Fig7}
%
\imageFigure{../figures/phy454-continuumechanics/lec13_area_and_velocity_flows_in_unequal_aperture_channel_configurationFig7}{Area and velocity flows in unequal aperture channel configuration.}{fig:continuumL13:continuumL13Fig7}{0.1}
%
We see the same behavior for channels that return to the original diameter after widening as in \cref{fig:continuumL13:continuumL13Fig8}
%
\imageFigure{../figures/phy454-continuumechanics/lec13_velocity_variation_in_channel_with_bulgeFig8}{Velocity variation in channel with bulge.}{fig:continuumL13:continuumL13Fig8}{0.1}
%
If we consider half of such a channel as in \cref{fig:continuumL13:continuumL13Fig9a}
\imageFigure{../figures/phy454-continuumechanics/lec13_vorticity_induction_due_to_pressure_gradients_in_unequal_aperture_channelFig9a}{Vorticity induction due to pressure gradients in unequal aperture channel.}{fig:continuumL13:continuumL13Fig9a}{0.2}
%
considering the flow around a small triangular section we must have a \textAndIndex{pressure gradient}, which induces a \textAndIndex{vorticity} flow.  We would see something similar in a rectangular channel where there is a block in the channel, as depicted in \cref{fig:continuumL13:continuumL13Fig9b}
%
\imageFigure{../figures/phy454-continuumechanics/lec13_vorticity_due_to_rectangular_blockageFig9b}{Vorticity due to rectangular blockage.}{fig:continuumL13:continuumL13Fig9b}{0.2}
%
\section{Height matching in odd geometries.}
Let us consider an arbitrarily weird channel as in \cref{fig:continuumL13:continuumL13Fig10}
\imageFigure{../figures/phy454-continuumechanics/lec13_height_matching_in_odd_geometriesFig10}{Height matching in odd geometries.}{fig:continuumL13:continuumL13Fig10}{0.2}
%
This was also illustrated with a glass blown container in class as in \cref{fig:continuumL13:continuumL13Fig11}
\imageFigure{../figures/phy454-continuumechanics/lec13_a_physical_demonstration_with_glass_blown_apparatusFig11}{A physical demonstration with glass blown apparatus.}{fig:continuumL13:continuumL13Fig11}{0.2}
%
In this real apparatus, we did not have exactly the same height (because of bubbles and capillary effects (surface tension induced meniscus curves), but we see first hand what we are talking about.

To account for this, we need to consider the situation in pieces as in \cref{fig:continuumL13:continuumL13Fig12}
%
\imageFigure{../figures/phy454-continuumechanics/lec13_column_volume_element_decomposition_for_odd_geometriesFig12}{Column volume element decomposition for odd geometries.}{fig:continuumL13:continuumL13Fig12}{0.2}
%
Breaking down the total pressure effects into individual bits, any column of fluid contributes to the pressure below it, even if that column of fluid is not directly on top of a continuous column of fluid all the way to the ``bottom''.

