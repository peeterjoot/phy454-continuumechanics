%
% Copyright � 2012 Peeter Joot.  All Rights Reserved.
% Licenced as described in the file LICENSE under the root directory of this GIT repository.
%
\section{Summary}
\subsection{Hydrostatics}

We covered hydrostatics as a separate topic, where it was argued that the pressure \(p\) in a fluid, given atmospheric pressure \(p_a\) and height from the surface was
%
\begin{equation}\label{eqn:continuumFluidsReview:1850}
p = p_a + \rho g h.
\end{equation}

As noted below in the surface tension problem, this is also a consequence of Navier-Stokes for \(\Bu = 0\) (following from \(0 = -\spacegrad p + \rho \Bg\)).

We noted that replacing the a mass of water with something of equal density would not change the non-dynamics of the situation.  We then went on to define Buoyancy force, the difference in weight of the equivalent volume of fluid and the weight of the object.

\subsection{Mass conservation through apertures}

It was noted that mass conservation provides a relationship between the flow rates through apertures in a closed pipe, since we must have
%
\begin{equation}\label{eqn:continuumFluidsReview:1870}
\rho_1 A_1 v_1 = \rho_2 A_2 v_2,
\end{equation}

and therefore for incompressible fluids
%
\begin{equation}\label{eqn:continuumFluidsReview:1890}
A_1 v_1 = A_2 v_2.
\end{equation}

So if \(A_1 > A_2\) we must have \(v_1 < v_2\).
