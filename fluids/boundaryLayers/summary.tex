%
% Copyright � 2012 Peeter Joot.  All Rights Reserved.
% Licenced as described in the file LICENSE under the root directory of this GIT repository.
%
\section{Summary}
\subsection{Impulsive flow}

We looked at the time dependent unidirectional flow where
%
\begin{align}\label{eqn:continuumFluidsReview:2230}
\Bu &= u(y, t) \xcap \\
\PD{t}{u} &= \nu \PDSq{y}{u} \\
u(0, t) &=
\left\{
\begin{array}{l l}
0 & \quad \mbox{for \(t < 0\)} \\
U & \quad \mbox{for \(t \ge 0\)}
\end{array}
\right.
\end{align}

and utilized a similarity variable \(\eta\) with \(u = U f(\eta)\)
%
\begin{equation}\label{eqn:continuumFluidsReview:2250}
\eta = \frac{y}{2 \sqrt{\nu t}}
\end{equation}
%
and were able to show that
%
\begin{equation}\label{eqn:continuumFluidsReview:2270}
u(y, t) = U_0 \left(1 - \erf\left(\frac{y}{2 \sqrt{\nu t}}\right) \right).
\end{equation}
%
The aim of this appears to be as an illustration that the boundary layer thickness \(\delta\) grows with \(\sqrt{\nu t}\).

\FIXME{Really need to plot \eqnref{eqn:continuumFluidsReview:2270}}

\subsection{Oscillatory flow}

Another worked problem in the boundary layer topic was the Stokes boundary layer problem with a driving interface of the form
%
\begin{equation}\label{eqn:continuumFluidsReview:2370}
U(t) = U_0 e^{i \Omega t}
\end{equation}
%
with an assumed solution of the form
%
\begin{equation}\label{eqn:continuumFluidsReview:2390}
u(y, t) = f(y) e^{i \Omega t},
\end{equation}
%
we found
%
\begin{subequations}
\begin{equation}\label{eqn:continuumFluidsReview:2410}
u(y, t) =
U_0 e^{-\lambda y} \cos\left( -i (\lambda y - \Omega t) \right).
\end{equation}
\begin{equation}\label{eqn:continuumFluidsReview:2430}
\lambda = \sqrt{\frac{\Omega}{2 \nu}}
\end{equation}
\end{subequations}
%
This was a bit more obvious as a boundary layer illustration since we see the exponential drop off with every distance multiple of \(\sqrt{\frac{2 \nu}{\Omega}}\).

\subsection{Blassius problem (\textAndIndex{boundary layer} thickness in flow over plate)}

We examined the scaling off all the terms in the Navier-Stokes equations given a velocity scale \(U\), vertical length scale \(\delta\) and horizontal length scale \(L\).  This, and the application of Bernoulli's theorem allowed us to make construct an approximation for Navier-Stokes in the boundary layer
%
\begin{subequations}
\begin{equation}\label{eqn:continuumFluidsReview:2450}
u \PD{x}{u} + v \PD{y}{u} = U \frac{dU}{dx} + \nu \PDSq{y}{u}
\end{equation}
\begin{equation}\label{eqn:continuumFluidsReview:2470}
\PD{y}{p} = 0
\end{equation}
\begin{equation}\label{eqn:continuumFluidsReview:2490}
\PD{x}{u} + \PD{y}{v} = 0
\end{equation}
\end{subequations}
%
With boundary conditions
%
\begin{align}\label{eqn:continuumFluidsReview:2510}
U(x, 0) &= 0 \\
U(x, \infty) &= U(x) = U_0 \\
V(x, 0) &= 0
\end{align}

With a similarity variable
%
\begin{equation}\label{eqn:continuumFluidsReview:2530}
\eta = \frac{y}{\sqrt{2 \frac{\nu x}{U}}}
\end{equation}
%
and stream functions
\begin{align}\label{eqn:continuumFluidsReview:2550}
u &= \PD{y}{\psi} \\
v &= -\PD{x}{\psi}
\end{align}

and
%
\begin{equation}\label{eqn:continuumFluidsReview:2570}
\psi = f(\eta) \sqrt{ 2 \nu x U_0 },
\end{equation}
%
we were able to show that our velocity dependence was given by the solutions of
%
\begin{equation}\label{eqn:continuumFluidsReview:2590}
f''' + f f'' = 0.
\end{equation}
%
This was done much more clearly in \citep{acheson1990elementary} and I worked this problem myself with a hybrid approach (non-dimensionalising as done in class).

\FIXME{The end result of this is a plot (a nice one can be found in \citep{wiki:BlasiusBoundary}).  That plot ends up being one that is done in terms of the similarity variable \(\eta\).  It is not clear to me how this translates into an actual velocity profile.  Should plot these out myself to get a feel for things}
