%
% Copyright � 2012 Peeter Joot.  All Rights Reserved.
% Licenced as described in the file LICENSE under the root directory of this GIT repository.
%

%
%
\section{Fluid flow over a solid body}
\subsection{Scaling arguments}
%
We have been talking about impulsively started flow and the Stokes boundary problem.

We will now move on to a similar problem, that of fluid flow over a solid body.

Consider \cref{fig:continuumL18:continuumL18Fig1}, where we have an illustration of flow over a solid object with a boundary layer of thickness \(\delta\).
%
\imageFigure{../figures/phy454-continuumechanics/lec18_Flow_over_a_solid_object_with_boundary_layerFig1}{Flow over a solid object with boundary layer.}{fig:continuumL18:continuumL18Fig1}{0.2}
%
We have a couple scales to consider.
\begin{enumerate}
\item Velocity scale \(U\),
\item length scale in the \(y\) direction \(\delta\),
\item length scale in the \(x\) direction \(L\),
\end{enumerate}
where
%
\begin{equation}\label{eqn:continuumL18:10}
L \gg \delta.
\end{equation}
%
As always, we start with the Navier-Stokes equation, restricting ourselves to the steady state \(\PDi{t}{\Bu} = 0\) case.  In coordinates, for incompressible flows, we have our usual \(x\) momentum, \(y\) momentum, and continuity equations
%
\begin{subequations}
\begin{equation}\label{eqn:continuumL18:30}
u \PD{x}{u} + v \PD{y}{u} = - \inv{\rho} \PD{x}{p} + \nu \left(
\PDSq{x}{u}
+\PDSq{y}{u}
\right)
\end{equation}
\begin{equation}\label{eqn:continuumL18:50}
u \PD{x}{v} + v \PD{y}{v} = - \inv{\rho} \PD{y}{p} + \nu \left(
\PDSq{x}{v}
+\PDSq{y}{v}
\right)
\end{equation}
\begin{equation}\label{eqn:continuumL18:70}
\PD{x}{u}
+\PD{y}{v} = 0
\end{equation}
\end{subequations}
%
Let us look at the scaling of these equations, starting with the continuity equation \eqnref{eqn:continuumL18:70}.  This is roughly
%
\begin{equation}\label{eqn:continuumL18:90}
\begin{aligned}
\PD{x}{u} &\sim \frac{U}{L} \\
\PD{y}{v} &\sim \frac{v}{\delta}
\end{aligned}
\end{equation}
%
We require that these have to be of the same order of magnitude.

\FIXME{Why?  This does not make sense to me since in horizontal flow we had \(v = 0\), and the two components of the divergence are obviously of different scales}

If these are of the same scale we have
%
\begin{equation}\label{eqn:continuumL18:110}
\frac{U}{L} \sim \frac{v}{\delta}
\end{equation}
%
so that
%
\begin{equation}\label{eqn:continuumL18:130}
v \sim \frac{U \delta}{L}
\end{equation}
%
or
%
\begin{equation}\label{eqn:continuumL18:150}
v \ll U
\end{equation}
%
Looking at the viscous terms
%
\begin{equation}\label{eqn:continuumL18:170}
\begin{aligned}
\nu \PDSq{x}{u} &\sim \frac{\nu U}{L^2} \\
\nu \PDSq{y}{u} &\sim \frac{\nu U}{\delta^2}
\end{aligned}
\end{equation}
%
or
%
\begin{equation}\label{eqn:continuumL18:190}
\nu \PDSq{y}{u} \gg \nu \PDSq{x}{u}
\end{equation}
%
So we can neglect the \(x\) component of the Laplacian in our \(x\) momentum equation \eqnref{eqn:continuumL18:30}.

How about the inertial terms
%
\begin{equation}\label{eqn:continuumL18:210}
\begin{aligned}
u \PD{x}{u} &\sim \frac{U^2}{L} \\
v \PD{y}{u} &\sim \frac{\delta U}{L} \frac{U}{\delta} \sim \frac{U^2}{L}
\end{aligned}
\end{equation}
%
Since these are of the same order (in the boundary regions) we cannot neglect either.  We also cannot neglect the pressure gradient, since this is what induces the flow.

For the \(y\) momentum equation we have
%
\begin{equation}\label{eqn:continuumL18:230}
\begin{aligned}
\nu \PDSq{x}{v} &\sim \nu \frac{\delta U}{L} \inv{L^2} \sim \nu \frac{\delta U}{L^3} \ll \frac{\nu U}{\delta^2} \\
\nu \PDSq{y}{v} &\sim \nu \frac{\delta U}{L} \inv{\delta^2} \sim \nu \frac{U}{\delta L} \ll \frac{\nu U}{\delta^2}
\end{aligned}
\end{equation}
%
We can neglect all the Laplacian terms in the \(y\) momentum equation.

\question
Why compare the magnitude of the viscous terms for the \(y\) momentum to the magnitude of the same terms in the \(x\) momentum equation, and not to the LHS of the \(y\) momentum equation.

\answer
That is a valid point, but our equations are coupled, and contributions from one feed into the other.

We are not done yet.  For the inertial terms in the \(y\) momentum equation we have
%
\begin{equation}\label{eqn:continuumL18:250}
\begin{aligned}
u \PD{x}{v} &\sim \frac{\delta U^2}{L^2} \\
v \PD{y}{v} &\sim \frac{\delta U}{L} \frac{\delta U}{L} \inv{\delta} \sim \frac{\delta U^2}{L^2}
\end{aligned}
\end{equation}
%
Note that
%
\begin{equation}\label{eqn:continuumL18:270}
\frac{\delta U^2}{L^2} = \frac{\delta}{L} \left( \frac{U^2}{L} \right) \ll \frac{U^2}{L}
\end{equation}
%
We see that both of the \(y\) momentum inertial terms can be neglected in comparison to the \(x\) momentum equations.

Putting all of this together, our equations of motion for the boundary flow are now reduced to
%
\begin{subequations}
\begin{equation}\label{eqn:continuumL18:30a}
u \PD{x}{u} + v \PD{y}{u} = - \inv{\rho} \PD{x}{p} + \nu \PDSq{y}{u}
\end{equation}
\begin{equation}\label{eqn:continuumL18:50a}
\PD{y}{p} = 0
\end{equation}
\begin{equation}\label{eqn:continuumL18:70a}
\PD{x}{u} + \PD{y}{v} = 0
\end{equation}
\end{subequations}
%
Utilizing Bernoulli's theorem \eqnref{eqn:continuumL18:410}, we can deal with the pressure term, the magnitude of which is
%
\begin{equation}\label{eqn:scalingForBoundaryLayer:450}
\begin{aligned}
- \PD{x}{} \left( \frac{p}{\rho} \right)
&\sim \PD{x}{} \left( \inv{2} u^2 \right) \\
&\sim \frac{2}{2} u \PD{x}{u} \\
&\sim U \frac{dU}{dx},
\end{aligned}
\end{equation}
%
and our equations of motion are finally reduced to
%
\begin{subequations}
\begin{equation}\label{eqn:continuumL18:30b}
u \PD{x}{u} + v \PD{y}{u} = U \frac{dU}{dx} + \nu \PDSq{y}{u}
\end{equation}
\begin{equation}\label{eqn:continuumL18:50b}
\PD{y}{p} = 0
\end{equation}
\begin{equation}\label{eqn:continuumL18:70b}
\PD{x}{u} + \PD{y}{v} = 0
\end{equation}
\end{subequations}
%
\FIXME{
Relevant?

Observe that if we operate on \eqnref{eqn:continuumL18:370} with a divergence operation, then we do not have to assume non-viscous flow but in that case we can only say that (for irrotational flow) we have
%
\begin{equation}\label{eqn:continuumL18:430}
\spacegrad^2 \left( \frac{p}{\rho} + \chi + \inv{2} \Bu^2 \right) = 0.
\end{equation}
}
