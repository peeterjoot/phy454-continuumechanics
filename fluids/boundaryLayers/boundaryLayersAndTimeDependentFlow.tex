%
% Copyright � 2012 Peeter Joot.  All Rights Reserved.
% Licenced as described in the file LICENSE under the root directory of this GIT repository.
%

%
%
\label{chap:continuumL16}
\section{Time dependent flow.}
Suppose we have an obstacle to fluid flow, as in \cref{fig:continuumL16:continuumL16Fig1}
\imageFigure{../figures/phy454-continuumechanics/lec16_Flow_lines_around_circular_obstacleFig1}{Flow lines around circular obstacle.}{fig:continuumL16:continuumL16Fig1}{0.15}
%
we have a couple conditions on fluid flow.
\begin{enumerate}
\item No fluid can cross the solid boundary
\item Due to viscosity the tangential velocity of the fluid should match the velocity of the solid boundary.
\end{enumerate}
In study of this type of flow, we can consider the flow separated into two portions, one is a flow that is largely viscous, and the other that is largely inertial.  This is depicted in \cref{fig:continuumL16:continuumL16Fig2}
%
\pdfTexFigure{../figures/phy454-continuumechanics/continuumL16Fig2.pdf_tex}{Viscous and inviscous regions in boundary layer flow.}{fig:continuumL16:continuumL16Fig2}{0.5}
%
We call the study of these two regions boundary layer flow.
\section{Unsteady rectilinear flow.}
\index{rectilinear flow}
Given
%
\begin{equation}\label{eqn:continuumL16:10}
\Bu = u(x, y, z, t) \xcap,
\end{equation}
%
the continuity equation (incompressibility assumption) is
%
\begin{equation}\label{eqn:continuumL16:30}
\PD{x}{u} +
\cancel{\PD{y}{v}} +
\cancel{\PD{z}{w}} = 0.
\end{equation}
%
Our non-linear term of Navier-Stokes is then killed
%
\begin{equation}\label{eqn:continuumL16:50}
u \PD{x}{u} = 0
\end{equation}
%
so that Navier Stokes takes the form
%
\begin{subequations}
\begin{equation}\label{eqn:continuumL16:70}
\rho \PD{t}{u} = -\PD{x}{p} +
\mu \left( \PDSq{y}{u} +\PDSq{z}{u} \right),
\end{equation}
\begin{equation}\label{eqn:continuumL16:90}
0 = -\PD{y}{p},
\end{equation}
\begin{equation}\label{eqn:continuumL16:110}
0 = -\PD{z}{p}.
\end{equation}
\end{subequations}
%
Taking the \(x\) partial of \eqnref{eqn:continuumL16:70}, we have
%
\begin{equation}\label{eqn:continuumL16:130}
\rho \PD{t}{} \cancel{\PD{x}{u}} = -\PDSq{x}{p} + \mu \left( \PDSq{y}{} \cancel{\PD{x}{u}} +\PDSq{z}{} \cancel{\PD{x}{u}} \right).
\end{equation}
%
Since we also have zero partials in the \(y\) and \(z\) directions from \eqnref{eqn:continuumL16:90}, and \eqnref{eqn:continuumL16:110}, we must then have
%
\begin{equation}\label{eqn:continuumL16:150}
\frac{d^2 p}{dx^2} = 0.
\end{equation}
%
So, after integrating we find for the pressure
%
\begin{equation}\label{eqn:continuumL16:170}
p(x, t) = p_0(t) - G x,
\end{equation}
%
where
%
\begin{equation}\label{eqn:continuumL16:190}
G = -\frac{dp}{dx}.
\end{equation}
%
In general \(G\) is a function of \(t\), but constant in space.  Given this, we have for our Navier-Stokes equation
%
\begin{equation}\label{eqn:continuumL16:210}
\rho \PD{t}{u} = G(t) + \mu \left( \PDSq{y}{u} +\PDSq{z}{u} \right).
\end{equation}
%
\makeexample{Impulsively started flow.}{ex:boundaries:impulsive}{
Reading: \S 2 from \citep{acheson1990elementary}

Let us consider a flow driven by a moving boundary.  We have two ways that we can look at such a flow, the first of which is with the fluid fixed and the boundary moving and the second is with the fluid moving and the boundary fixed.  This are depicted respectively in \cref{fig:continuumL16:continuumL16Fig3a}) and \cref{fig:continuumL16:continuumL16Fig3b}
%
\imageFigure{../figures/phy454-continuumechanics/lec16_Lagrangian_viewFig3a}{Lagrangian view.}{fig:continuumL16:continuumL16Fig3a}{0.15}
\imageFigure{../figures/phy454-continuumechanics/lec16_Eulerian_viewFig3b}{Eulerian view.}{fig:continuumL16:continuumL16Fig3b}{0.15}
%
These two possible viewpoints can be called the Eulerian and the Lagrangian views where
\begin{itemize}
\item Lagrangian: the observer is moving with the fluid.
\item Eulerian: the observer is fixed in space, watching the fluid.
\end{itemize}
With a flow of the form
%
\begin{equation}\label{eqn:continuumL16:230}
\Bu = u(y, t) \xcap,
\end{equation}
%
the Navier-Stokes equation is
%
\boxedEquation{eqn:continuumL16:250}{
\PD{t}{u} = \nu \PDSq{y}{u}.
}
%
Our boundary value constraints are
%
\begin{equation}\label{eqn:continuumL16:270}
u(0, t) =
\left\{
\begin{array}{l l}
0 & \quad \mbox{for \(t < 0\)} \\
U & \quad \mbox{for \(t \ge 0\)},
\end{array}
\right.
\end{equation}
%
and \(u \rightarrow 0\) as \(y \rightarrow \infty\).

If we make a transformation to dimensionless arguments
%
\begin{equation}\label{eqn:continuumL16:290}
u \rightarrow U,
\end{equation}
%
so that
%
\begin{equation}\label{eqn:continuumL16:310}
\frac{u}{U} \rightarrow \text{dimensionless}.
\end{equation}
%
Then we require of the parameters
%
\begin{equation}\label{eqn:continuumL16:330}
y, \nu, t \rightarrow \frac{y}{\sqrt{\nu t}},
\end{equation}
%
so that we have a characteristic length scale of the form
%
\begin{equation}\label{eqn:continuumL16:350}
\delta \rightarrow \sqrt{\nu t}.
\end{equation}
%
We can find an approximate solution
%
\begin{equation}\label{eqn:continuumL16:370}
\frac{U}{t} \approx \frac{\nu U}{\delta^2}.
\end{equation}
%
We can introduce a similarity variable (often hard to find), of the form
%
\begin{equation}\label{eqn:continuumL16:390}
\eta = \frac{y}{2 \sqrt{\nu t}}.
\end{equation}
%
\FIXME{try attacking this more systematically using Fourier or Laplace transforms (probably a Laplace transform, because of our initial conditions.)}

Let us use our similarity variable and see what happens.  With
%
\begin{equation}\label{eqn:continuumL16:410}
u = U f(\eta),
\end{equation}
%
we find
%
\begin{equation}\label{eqn:boundaryLayersAndTimeDependentFlow:770}
\begin{aligned}
\PD{y}{u}
&= \PD{\eta}{u} \PD{y}{\eta} \\
&= U f' \inv{2 \sqrt{\nu t}},
\end{aligned}
\end{equation}
%
where
%
\begin{equation}\label{eqn:continuumL16:430}
f' = \PD{\eta}{f}.
\end{equation}
%
We then find
%
\begin{equation}\label{eqn:continuumL16:450}
\PDSq{y}{u} = U f'' \inv{4 \nu t},
\end{equation}
%
and
%
\begin{equation}\label{eqn:boundaryLayersAndTimeDependentFlow:790}
\begin{aligned}
\PD{t}{u}
&= \PD{\eta}{u} \PD{t}{\eta} \\
&= \PD{\eta}{u} \PD{t}{}\left(
\frac{y}{2 \sqrt{\nu t}}
\right) \\
&= -\inv{2} \PD{\eta}{u} \left(
\frac{y}{2 \sqrt{\nu} t^{3/2}}
\right) \\
&= - U f' \frac{\eta}{2 t}.
\end{aligned}
\end{equation}
%
Putting these all together we have
%
\begin{equation}\label{eqn:continuumL16:630}
- \cancel{U} f' \frac{\eta}{2 t} = \cancel{\nu} \cancel{U} f'' \inv{4 \cancel{\nu} t},
\end{equation}
%
or
%
\begin{equation}\label{eqn:continuumL16:490}
f'' + 2 \eta f' = 0.
\end{equation}
%
With \(g = f'\), we have
%
\begin{equation}\label{eqn:continuumL16:650}
\int \frac{dg}{g} = \int -2 \eta y.
\end{equation}
%
With solution
%
\begin{equation}\label{eqn:continuumL16:670}
\ln(f') = - \eta^2 + \ln C,
\end{equation}
%
or
%
\begin{equation}\label{eqn:continuumL16:690}
f' = C e^{- \eta^2}.
\end{equation}
%
Integrating once more, and writing the integral in terms of the error function
%
\begin{equation}\label{eqn:continuumL16:530}
\erf(\eta) = \frac{2}{\sqrt{\pi}} \int_0^\eta e^{-s^2} ds,
\end{equation}
%
We find
%
\begin{equation}\label{eqn:continuumL16:510}
f(\eta) = A \erf(\eta) + B.
\end{equation}
%
From our boundary value condition at the origin we have
%
\begin{equation}\label{eqn:continuumL16:710}
u(0, t) = U f(0) = U,
\end{equation}
%
so that
%
\begin{equation}\label{eqn:continuumL16:730}
U( A \erf(0) + B) = U.
\end{equation}
%
Since \(\erf(0) = 0\), we must have \(B = 1\).  For our boundary value constraint far from the impulse, we have
%
\begin{equation}\label{eqn:continuumL16:750}
u(\infty, t) = U( A \erf(\infty) + 1) = 0,
\end{equation}
%
but since \(\erf(\infty) = 1\), we must have \(A = -1\).  Our solution is then found to be
%
\boxedEquation{eqn:continuumL16:590}{
u(y, t) = U(1 - \erf(\eta)).
}
%
where (again)
%
\begin{equation}\label{eqn:continuumL16:610}
\eta = \frac{y}{2 \sqrt{\nu t}}.
\end{equation}
%
Explicitly, this is
%
\begin{equation}\label{eqn:continuumL16:590b}
u(y, t) = U_0 \left(1 - \erf\left(\frac{y}{2 \sqrt{\nu t}}\right) \right).
\end{equation}
%
%\unnumberedSubsection{Boundary layer thickness}
If we look at the thickness of the boundary layer for different viscosities, sampled at different times we may end up with curves as in \cref{fig:continuumL16:continuumL16Fig4a}
%
\pdfTexFigure{../figures/phy454-continuumechanics/continuumL16Fig4a.pdf_tex}{Plots of separation thickness for different viscosities.}{fig:continuumL16:continuumL16Fig4a}{0.4}
%
However, it turns out that for any fluids, regardless of the viscosities, the thickness of the boundary layers generally vary as a linear function of \(\sqrt{\nu t}\) so if \(\delta\) is plotted against that as in \cref{fig:continuumL16:continuumL16Fig4b} we see a linear relationship.
%
\pdfTexFigure{../figures/phy454-continuumechanics/continuumL16Fig4b.pdf_tex}{Linear relation between separation thickness and \(\sqrt{\nu t}\).}{fig:continuumL16:continuumL16Fig4b}{0.4}
} % end example
