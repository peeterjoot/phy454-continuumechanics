%
% Copyright � 2012 Peeter Joot.  All Rights Reserved.
% Licenced as described in the file LICENSE under the root directory of this GIT repository.
%

%
%

%\chapter{PHY454H1S Continuum Mechanics.  Lecture 17: Impulsive flow.  Boundary layers.  Oscillatory driven flow.  Taught by Prof. K. Das}
\label{chap:continuumL17}
%
%\section{Impulsive flow.  Boundary layers.  Oscillatory driven flow}
\section{Review.  Impulsively started flow.}
%
Were looking at flow driven by an impulse, a sudden motion of the plate, as in \cref{fig:continuumL17:continuumL17Fig1}
\imageFigure{../figures/phy454-continuumechanics/lec17_Impulsively_driven_time_dependent_fluid_flowFig1}{Impulsively driven time dependent fluid flow.}{fig:continuumL17:continuumL17Fig1}{0.3}
%
where the fluid at the origin is pushed so that it is given the velocity
%
\begin{equation}\label{eqn:continuumL17:20}
u(0, t) =
\left\{
\begin{array}{l l}
0 & \quad \mbox{for \(t < 0\)} \\
U(t) & \quad \mbox{for \(t \ge 0\)} \\
\end{array}
\right.
\end{equation}
%
where \(U \rightarrow 0\) as \(y \rightarrow \infty\).

Navier-Stokes takes the form
%
\begin{equation}\label{eqn:continuumL17:40}
\PD{t}{u} = \nu \PDSq{y}{u}.
\end{equation}
%
With a similarity variable
%
\begin{equation}\label{eqn:continuumL17:60}
\eta = \frac{y}{2 \sqrt{\nu t}},
\end{equation}
%
and
%
\begin{equation}\label{eqn:continuumL17:80}
u = U f(\eta),
\end{equation}
%
we found that we needed to solve
%
\begin{equation}\label{eqn:continuumL17:100}
f'' + 2 \eta f' = 0
\end{equation}
%
where
%
\begin{equation}\label{eqn:continuumL17:120}
f' = \frac{df}{d\eta}
\end{equation}
%
with solution
%
\begin{equation}\label{eqn:continuumL17:140}
u(y, t) = U(1 - \erf(\eta)).
\end{equation}
%
Here, we have used the error function
%
\begin{equation}\label{eqn:continuumL17:160}
\erf(\eta) = \frac{2}{\sqrt{\pi}} \int_0^\eta e^{-s^2} ds,
\end{equation}
%
as plotted in \cref{fig:continuumL17:continuumL17Fig2}
\imageFigure{../figures/phy454-continuumechanics/lec17_Error_functionFig2}{Error function.}{fig:continuumL17:continuumL17Fig2}{0.3}
%
\section{Boundary layers.}
%
Let us look at spacetime points which are constant in \(\eta\)
%
\begin{equation}\label{eqn:continuumL17:180}
\frac{y_1}{2 \sqrt{\nu t_1}} = \frac{y_2}{2 \sqrt{\nu t_2}},
\end{equation}
%
so that the speed at \((y_1, t_1)\) equals the speed at \((y_2, t_2)\).  This is illustrated in \cref{fig:continuumL17:continuumL17Fig3}
%
\pdfTexFigure{../figures/phy454-continuumechanics/continuumL17Fig3.pdf_tex}{Velocity profiles at different times.}{fig:continuumL17:continuumL17Fig3}{0.6}
%
\section{Universal behavior.}
%
Looking at a plot with different viscosities for position vs time scaled as \(\sqrt{\nu t}\) as in \cref{fig:continuumL17:continuumL17Fig4} we see a sort of universal behavior

%
\pdfTexFigure{../figures/phy454-continuumechanics/continuumL17Fig4.pdf_tex}{Universal behavior.}{fig:continuumL17:continuumL17Fig4}{0.6}
%
Characterizing this we introduce the concept of boundary layer thickness

\makedefinition{Boundary layer thickness}{dfn:continuumL17:200}{The length scale over which \index{viscosity} is dominant.  This is the viscous length scale. \index{boundary layer}}

This is similar to what we have in the heat equation
%
\begin{equation}\label{eqn:continuumL17:220}
\PD{t}{T} = \kappa \PDSq{y}{T},
\end{equation}
%
where the time scale for the diffusion can be expressed as
%
\begin{equation}\label{eqn:continuumL17:240}
[\kappa_t] = \frac{d^2}{\kappa}.
\end{equation}
%
We could consider a scenario such as a heated plate in a cavity of height \(\delta\) as in \cref{fig:continuumL17:continuumL17Fig5}
%
\pdfTexFigure{../figures/phy454-continuumechanics/continuumL17Fig5.pdf_tex}{Characteristic distances in heat flow problems.}{fig:continuumL17:continuumL17Fig5}{0.5}
%
with a temperature \(T\) on the bottom plate.  We can ask how fast the heat propagates through the medium.
\makeexample{Oscillating plate.}{ex:boundaryLayers:oscillating}{
Consider an oscillating plate, driving the motion of the fluid, as in \cref{fig:continuumL17:continuumL17Fig6}
\imageFigure{../figures/phy454-continuumechanics/lec17_Time_dependent_fluid_motion_due_to_oscillating_plateFig6}{Time dependent fluid motion due to oscillating plate.}{fig:continuumL17:continuumL17Fig6}{0.2}
%
\begin{equation}\label{eqn:continuumL17:260}
U(t) = U_0 \cos \Omega t = \Real\left( U_0 e^{i \Omega t} \right).
\end{equation}
%
(we are thinking here about the always oscillating case, and not an impulsive plate motion).

We write
%
\begin{equation}\label{eqn:continuumL17:280}
u(y, t) = \Real\left( f(y) e^{i \Omega t} \right)
\end{equation}
%
with substitution into
%
\begin{equation}\label{eqn:continuumL17:300}
\PD{t}{u} = \nu \PDSq{y}{u},
\end{equation}
%
we have
%
\begin{equation}\label{eqn:continuumL17:320}
i \Omega f(y) e^{i \Omega t} = \nu f'' e^{i \Omega t}
\end{equation}
%
or
%
\begin{equation}\label{eqn:continuumL17:340}
i \Omega f(y) = \nu f''
\end{equation}
%
This is an equation of the form
%
\begin{equation}\label{eqn:continuumL17:360}
f'' = m^2 f
\end{equation}
%
where
%
\begin{equation}\label{eqn:continuumL17:380}
m^2 = \frac{i \Omega}{\nu}.
\end{equation}
%
or
%
\begin{equation}\label{eqn:continuumL17:400}
m = \sqrt{\frac{i \Omega}{\nu}} = \lambda (1 + i),
\end{equation}
%
where
%
\begin{equation}\label{eqn:continuumL17:420}
\lambda = \sqrt{\frac{\Omega}{2 \nu}}.
\end{equation}
%
check:
%
\begin{equation}\label{eqn:impulsiveFlowAndBoundaryLayersAndOscillatoryDrivenFlow:660}
\begin{aligned}
m^2
&=
\frac{\Omega}{2 \nu} (i + 1)^2 \\
&=
\frac{\Omega}{2 \nu} (i^2 + 1 + 2 i) \\
&=
\frac{\Omega}{\nu} i
\end{aligned}
\end{equation}
%
Considering the boundary value constraints we have
%
\begin{equation}\label{eqn:continuumL17:440}
f(y) =
A e^{\lambda (1 + i) y}
+ B e^{-\lambda (1 + i) y}
\end{equation}
%
Since \(u(\infty, t) \rightarrow 0\) we must have
%
\begin{equation}\label{eqn:continuumL17:460}
f(\infty) = 0,
\end{equation}
%
so we must kill off the exponentially increasing (albeit also oscillating) term by setting \(A = 0\).  Also, since
%
\begin{equation}\label{eqn:continuumL17:480}
u(0, t) = U(t)
\end{equation}
%
we must have
%
\begin{equation}\label{eqn:continuumL17:500}
f(0) = U_0
\end{equation}
%
or
%
\begin{equation}\label{eqn:continuumL17:520}
B = U_0
\end{equation}
%
so
%
\begin{equation}\label{eqn:continuumL17:540}
f(y) = U_0 e^{-\lambda (1 + i) y}
\end{equation}
%
and
%
\begin{equation}\label{eqn:continuumL17:560}
u(y, t) =
\Real\left(
U_0 e^{-\lambda y} e^{ -i (\lambda y - \Omega t) }
\right)
\end{equation}
%
or
%
\begin{equation}\label{eqn:continuumL17:580}
u(y, t) =
U_0 e^{-\lambda y} \cos\left( -i (\lambda y - \Omega t) \right).
\end{equation}
%
This is a damped transverse wave function
%
\begin{equation}\label{eqn:continuumL17:600}
u(y, t) = f(y - c t),
\end{equation}
%
where
%
\begin{equation}\label{eqn:continuumL17:620}
c = \frac{\Omega}{\lambda},
\end{equation}
%
is the wave speed.

Since we have an exponential damping here, the flow of fluid will essentially be confined to a boundary layer, where after distance \(y = n/\lambda\), the oscillation falls off as
%
\begin{equation}\label{eqn:continuumL17:640}
\inv{e^n}.
\end{equation}
%
We can find a nice illustration of such a flow in \citep{wiki:StokesBoundary}.
}
