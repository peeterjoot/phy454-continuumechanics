%
% Copyright � 2012 Peeter Joot.  All Rights Reserved.
% Licenced as described in the file LICENSE under the root directory of this GIT repository.
%

%
%
\makeoproblem
%{Velocity phase in oscillatory boundary layer problem}
{Oscillatory boundary, phase angle.}
{problem:fluids:ps3:q3}
{2012 ps3, p3}
{
Starting with the solution of the Stokes' boundary layer problem calculate shear stress on the plate \(y = 0\).   What is the phase difference between the velocity of the plate \(U(t) = U_0 \cos\omega t\) and the shear stress on the plate?
} % makeoproblem
\makeanswer{problem:fluids:ps3:q3}{
We found in class that the velocity of the fluid was given by
\begin{equation}\label{eqn:continuumProblemSet3:370}
u(y, t) = U_0 e^{-\lambda y} \cos(\lambda y - \omega t),
\end{equation}
where
\begin{equation}\label{eqn:continuumProblemSet3:390}
\lambda = \sqrt{\frac{\omega}{2 \nu}}.
\end{equation}
Calculating our shear stress we find
\begin{equation}\label{eqn:continuumProblemSet3:450}
\begin{aligned}
\mu \PD{y}{u}
&= U_0 \lambda \mu e^{-\lambda y}
\left(
-1
-
 \sin(\lambda y - \omega t)
\right),
\end{aligned}
\end{equation}
and on the plate (\(y = 0\)) this is just
\begin{equation}\label{eqn:continuumProblemSet3:410}
\evalbar{\mu \PD{y}{u}}{y = 0} = U_0 \lambda \mu (-1 + \sin(\omega t)).
\end{equation}
We have got a constant term, plus one that is sinusoidal.  Observing that
\begin{equation}\label{eqn:continuumProblemSet3:430}
\begin{aligned}
\cos x &= \Real( e^{ix} )  \\
\sin x &= \Real( -i e^{ix} ) = \Real( e^{i (x - \pi/2)} ).
\end{aligned}
\end{equation}
The phase difference between the non-constant portion of the shear stress at the plate, and the plate velocity \(U(t) = U_0 \cos\omega t\) is just \(-\pi/2\).  The shear stress at the plate lags the driving velocity by 90 degrees.
%\FIXME{review posted solutions.  I think Prof Das posted a \(5 \pi/4\) result?}
} % end answer
