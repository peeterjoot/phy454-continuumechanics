%
% Copyright � 2012 Peeter Joot.  All Rights Reserved.
% Licenced as described in the file LICENSE under the root directory of this GIT repository.
%
\label{chap:continuumL14}
\section{Scaling.}
%
By \textAndIndex{scaling} we mean how much detail do you want to look at in the analysis.  Consider the \cref{fig:continuumL14:continuumL14fig5a} where we imagine that we zoom in on something that appears smooth from a distance.  However, we are free to perform a change of variables on our coordinates and rescale in any arbitrary fashion.  For example
%
\imageFigure{../figures/phy454-continuumechanics/lec14_Coarse_scaling_exampleFig5a}{Coarse scaling example.}{fig:continuumL14:continuumL14fig5a}{0.2}
%
\begin{align}\label{eqn:continuumL14:210}
x &\rightarrow A u^\alpha \\
y &\rightarrow B v^\beta.
\end{align}
For a linear zoom scaling (\(\alpha = \beta = 1\)) we could perhaps find that we have something very granular close up as in \cref{fig:continuumL14:continuumL14fig5b}.  Picking the length scale to be used in this case can be very important.
%
\imageFigure{../figures/phy454-continuumechanics/lec14_Fine_grain_scaling_example_a_zoomFig5b}{Fine grain scaling example (a zoom).}{fig:continuumL14:continuumL14fig5b}{0.2}
%
The flexibility to rescale with non unity values for \(\alpha\) and \(\beta\) can, for example, come in handy, should we choose to rescale time and position differently.
%
\section{Rescaling by characteristic length and velocity.}
\index{characteristic length}
\index{characteristic velocity}
%
Suppose that a fluid is flowing with

\begin{itemize}
\item a characteristic velocity \(U\), with dimensions \([U] \sim L T^{-1}\)
\item a characteristic length scale \(L\)
\end{itemize}

Considering the dimensions of the terms in the Navier-Stokes equation
%
\begin{equation}\label{eqn:continuumL14:230}
[\rho] = M L^{-3},
\end{equation}
\begin{equation}\label{eqn:continuumL14:250}
[p] = M L T^{-2} L^{-2} = M L^{-1} T^{-2},
\end{equation}
\begin{equation}\label{eqn:continuumL14:270}
[t] = T = \frac{L}{U},
\end{equation}
%
so
\begin{equation}\label{eqn:continuumL14:290}
[p] = [\rho U^2] = M L^{-3} L^2 T^{-2}  = M L^{-1} T^{-2}.
\end{equation}
%
Now let us alter the Navier-Stokes equation using some scaling to put it into a dimensionless form
%
\begin{equation}\label{eqn:continuumL14:310}
\PD{t}{\Bu} + (\Bu \cdot \spacegrad) \Bu = - \inv{\rho} \spacegrad + \nu \spacegrad^2 \Bu
\end{equation}
%
\begin{equation}\label{eqn:continuumL14:330}
\PD{t}{\Bu} \rightarrow  \PD{\left(\frac{L}{U} t'\right)}{(U \Bu')} = \frac{U^2}{L} \PD{t'}{\Bu'}
\end{equation}
%
\begin{equation}\label{eqn:continuumL14:350}
\spacegrad =
\xcap \PD{x}{}
+\ycap \PD{y}{}
+\zcap \PD{z}{}
\rightarrow
\xcap \PD{L x'}{}
\ycap \PD{L y'}{}
\zcap \PD{L z'}{},
\end{equation}
%
so that
%
\begin{equation}\label{eqn:continuumL14:370}
\spacegrad \rightarrow \inv{L} \spacegrad'
\end{equation}
%
\begin{equation}\label{eqn:continuumL14:390}
(\Bu \cdot \spacegrad ) \Bu \rightarrow
\left( U \Bu' \cdot \inv{L} \spacegrad' \right) U \Bu' = \frac{U^2}{L} (\Bu' \cdot \spacegrad') \Bu'
\end{equation}
%
\begin{equation}\label{eqn:continuumL14:410}
\inv{\rho} \spacegrad p \rightarrow \inv{L} \frac{\spacegrad' (\cancel{\rho} U^2) }{\cancel{\rho}} p' = \frac{U^2}{L} \spacegrad' p'
\end{equation}
%
\begin{equation}\label{eqn:continuumL14:430}
\nu \spacegrad^2 \Bu \rightarrow \frac{\nu}{L^2} \spacegrad' U \Bu' = \frac{\nu U}{L^2} \spacegrad' \Bu'.
\end{equation}
%
Putting everything together, Navier-Stokes takes the form
%
\begin{equation}\label{eqn:continuumL14:450}
\frac{U^2}{L} \PD{t'}{\Bu'} + \frac{U^2}{L} (\Bu' \cdot \spacegrad') \Bu' = \frac{U^2}{L} \spacegrad' p' + \frac{\nu U}{L^2} \spacegrad' \Bu',
\end{equation}
%
or
\begin{equation}\label{eqn:continuumL14:470}
\PD{t'}{\Bu'} + (\Bu' \cdot \spacegrad') \Bu' = \spacegrad' p' + \frac{\nu}{U L} \spacegrad' \Bu'.
\end{equation}
%
Introducing the Reynold's number
%
\begin{equation}\label{eqn:continuumL14:490}
R = \frac{L U}{\nu}.
\end{equation}
%
We have Navier-Stokes in dimensionless form
%
\begin{equation}\label{eqn:continuumL14:510}
\PD{t'}{\Bu'} + (\Bu' \cdot \spacegrad') \Bu' = \spacegrad' p' + \inv{R} \spacegrad' \Bu'.
\end{equation}
%
The implications of this will be discussed further in the next lecture.

Reading: Coverage of this topic (with some problems) can be found in \S 7.6, \S 7.7 of \citep{granger1995fluid}.
