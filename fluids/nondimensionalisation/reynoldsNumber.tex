%
% Copyright � 2012 Peeter Joot.  All Rights Reserved.
% Licenced as described in the file LICENSE under the root directory of this GIT repository.
%

%
%

%\chapter{PHY454H1S Continuum Mechanics.  Lecture 15: More on surface tension and Reynold's number.  Taught by Prof. K. Das}

%\chapter{Reynold's number}
\label{chap:continuumL15}
\section{Reynold's number.}
%
In Navier-Stokes after making non-dimensionalization changes of the form
%
\begin{equation}\label{eqn:continuumL15:150}
x \rightarrow L x',
\end{equation}
%
the control parameter is like Reynold's number.

In Navier-Stokes
%
\begin{equation}\label{eqn:continuumL15:170}
\rho \PD{t}{\Bu} + \rho ( \Bu \cdot \spacegrad ) \Bu = - \spacegrad p + \mu \spacegrad^2 \Bu,
\end{equation}
%
we call the term
%
\begin{equation}\label{eqn:continuumL15:190}
\rho ( \Bu \cdot \spacegrad ) \Bu,
\end{equation}
%
the inertial term.  It is non-zero only when something is being ``carried along with the velocity''.  Consider a volume fixed in space and one that is moving along with the fluid as in \cref{fig:continuumL15:continuumL15Fig4}
\imageFigure{../figures/phy454-continuumechanics/lec15_Moving_and_fixed_frame_control_volumes_in_a_fluidFig4}{Moving and fixed frame control volumes in a fluid.}{fig:continuumL15:continuumL15Fig4}{0.2}
%
All of our viscosity dependence shows up in the Laplacian term, so we can roughly characterize the Reynold's number as the ratio
%
\begin{equation}\label{eqn:reynoldsNumber:210}
\begin{aligned}
\text{Reynold's number}
&\rightarrow
\frac{\Abs{\text{effect of inertia}}}{\Abs{\text{effect of viscosity}}}  \\
&=
\frac{ \Abs{\rho ( \Bu \cdot \spacegrad ) \Bu } }{\Abs{ \mu \spacegrad^2 \Bu}} \\
&\sim
\frac{ \rho U^2/L }{\mu U/L^2} \\
&\sim
\frac{ \rho U L }{\mu }.
\end{aligned}
\end{equation}
%
In \cref{fig:continuumL15:continuumL15Fig5}, and \cref{fig:continuumL15:continuumL15Fig6} we have two illustrations of viscous and non-viscous regions the first with a moving probe pushing its way through a surface, and the second with a wing set at an angle of attack that generates some turbulence.  Both are illustrations of the viscous and inviscous regions for the two flows.  Both of these are characterized by the Reynold's number in some way not really specified in class.  One of the points of mentioning this is that when we are in an essentially inviscous region, we can neglect the viscosity (\(\mu \spacegrad^2 \Bu\)) term of the flow.
\imageFigure{../figures/phy454-continuumechanics/continuumL15Fig5}{Viscous and non-viscous regions.}{fig:continuumL15:continuumL15Fig5}{0.2}
\imageFigure{../figures/phy454-continuumechanics/continuumL15Fig6}{Viscous and non-viscous regions.}{fig:continuumL15:continuumL15Fig6}{0.2}
