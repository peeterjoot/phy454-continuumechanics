%
% Copyright � 2012 Peeter Joot.  All Rights Reserved.
% Licenced as described in the file LICENSE under the root directory of this GIT repository.
%
\makeoproblem{Velocity non-dimensionalisation.}{problem:fluids:ps3}
{2012 ps3}
{
In fluid convection problems one can make several choices for characteristic velocity scales.  Some choices are given below for example:
\begin{enumerate}
\item \(U_1 = g \alpha d^2 \nabla T/\nu\)
\item \(U_2 = \nu/d\)
\item \(U_3 = \sqrt{ g \alpha d \nabla T }\)
\item \(U_4 = \kappa/ d\),
\end{enumerate}
where \(g\) is the acceleration due to gravity, \(\alpha = (\PDi{T}{V})/V\) is the coefficient of volume expansion, \(d\) length scale associated with the problem, \(\nabla T\) is the applied temperature difference, \(\nu\) is the kinematic viscosity and \(\kappa\) is the thermal diffusivity.
%
\makesubproblem{Verify that each of the expressions above have units of velocity.}{problem:fluids:ps3:1}
\makesubproblem{Water convection at room temperature.
For pure liquid, say pure water at room temperature, one has the following estimates in cgs units:
%
\begin{equation}\label{eqn:problems:370}
\begin{aligned}
\alpha &\sim 10^{-4} \\
\kappa &\sim 10^{-3} \\
\nu &\sim 10^{-2}.
\end{aligned}
\end{equation}
%
For a \(d \sim 1 \text{cm}\) layer depth and a ten degree temperature drop convective velocities have been experimentally measured of about \(10^{-2}\).
% \text{cm}/\text{s}
With \(g \sim 10^{-3}\), calculate the values of \(U_1\), \(U_2\), \(U_3\), and \(U_4\).  Which ones of the characteristic velocities \((U_1\), \(U_2\), \(U_3, U_4)\) do you think are suitable for nondimensionalising the velocity in Navier-Stokes/Energy equation describing the water convection problem?
}{problem:fluids:ps3:2}
%
\makesubproblem{For mantle convection, we have
\begin{equation}\label{eqn:problems:390}
\begin{aligned}
\alpha &\sim 10^{-5} \\
\nu &\sim 10^{21} \\
\kappa &\sim 10^{-2} \\
d &\sim 10^8 \\
\nabla T &\sim 10^3,
\end{aligned}
\end{equation}
and the actual convective mantle velocity is \(10^{-8}\).  Which of the characteristic velocities should we use to nondimensionalise Navier-Stokes/Energy equations describing mantle convection?
}{problem:fluids:ps3:3}
} % makeoproblem
%
\makeanswer{problem:fluids:ps3}{
\makesubanswer{Verify units.}{problem:fluids:ps3:1}
Let us check each of the velocity expressions in turn.
\begin{enumerate}
\item For \(U_1\):
Observing that
\begin{equation}\label{eqn:continuumProblemSet3:10}
\left[ \PD{t}{\Bu} \right] = [ \nu \spacegrad^2 \Bu ],
\end{equation}
we must have
\begin{equation}\label{eqn:continuumProblemSet3:30}
[\nu] = \inv{[t] [\spacegrad^2]} = \inv{T} L^2.
\end{equation}
We also find
\begin{equation}\label{eqn:continuumProblemSet3:50}
[\alpha] = \inv{[V]} \left[ \PD{T}{V} \right] = \inv{[K]},
\end{equation}
so that
\begin{equation}\label{eqn:continuumProblemSet3:70}
[U_1] = \frac{L}{T \cancel{T}} \cancel{\inv{K}} L^2 \cancel{K} \frac{\cancel{T}}{L^2} = \frac{L}{T}.
\end{equation}
\item For \(U_2\):
\begin{equation}\label{eqn:continuumProblemSet3:90}
[U_2] = \frac{L^2}{T} \inv{L} = \frac{L}{T}.
\end{equation}
%
\item For \(U_3\):
%
\begin{equation}\label{eqn:continuumProblemSet3:110}
[U_3] = \sqrt{ \frac{L}{T^2} \inv{K} L K } = \frac{L}{T}.
\end{equation}
%
\item For \(U_4\):
According to \citep{wolframThermalDiffusivity}, thermal diffusivity is defined by
%
\begin{equation}\label{eqn:continuumProblemSet3:130}
\PD{t}{T} = \kappa \spacegrad^2 T,
\end{equation}
%
so that
%
\begin{equation}\label{eqn:continuumProblemSet3:150}
[\kappa] = \inv{[t][\spacegrad^2]} = \frac{L^2}{T}.
\end{equation}
%
That gives us
%
\begin{equation}\label{eqn:continuumProblemSet3:170}
[U_4] = \frac{L^2}{T} \inv{L} = \frac{L}{T}.
\end{equation}
\end{enumerate}
%
%We have verified that all of these have dimensions of velocity.
%
\makesubanswer{Water convection.}{problem:fluids:ps3:2}
For water at room temperature, we have
%
\begin{equation}\label{eqn:continuumProblemSet3:190}
\begin{aligned}
U_1 &\sim 10^{-3} 10^{-4} (1)^2 10^1 \inv{10^{-2}} = 10^{-4} \\
U_2 &\sim 10^{-2}/1 = 10^{-2} \\
U_3 &\sim \sqrt{ 10^{-3} 10^{-4} (1) 10^1 } = 10^{-3} \\
U_4 &\sim 10^{-3}/1 = 10^{-3}.
\end{aligned}
\end{equation}
%
Use of \(U_2 = \nu/d\) gives the closest match to the measured characteristic velocity of \(10^{-2}\).
%
\makesubanswer{Mantle convection.}{problem:fluids:ps3:3}
For the mantle convection problem let us compute the characteristic velocities
\begin{equation}\label{eqn:continuumProblemSet3:210}
\begin{aligned}
U_1 &\sim \frac{10^{-3} 10^{-5} 10^{16} 10^3 }{10^{21}} = 10^{-10} \\
U_2 &\sim \frac{10^{21}}{10^8} = 10^{13} \\
U_3 &\sim \sqrt{ 10^{-3} 10^{-5} 10^8 10^3 } \sim 10^1 \\
U_4 &\sim \frac{10^{-2}}{10^8} = 10^{-10}.
\end{aligned}
\end{equation}
%
Both \(U_1\) and \(U_4\) come close to the actual convective mantle velocity of \(10^{-8}\).  Use of \(U_1\) to nondimensionalise is probably best, since it has more degrees of freedom, and includes the gravity term that is probably important for such large masses.
\FIXME{check against posted solutions}
} % end answer
%
\makeoproblem{Nondimensionalise N-S equation.}{problem:fluids:ps3:q2}
{2012 ps3, p2}
{
\begin{equation}\label{eqn:continuumProblemSet3:230}
\rho \PD{t}{\Bu} + \rho (\Bu \cdot \spacegrad) \Bu = - \spacegrad p + \mu \spacegrad^2 \Bu + \rho g \zcap,
\end{equation}
%
where \(\zcap\) is the unit vector in the \(z\) direction.  You may scale:
\begin{itemize}
\item velocity with the characteristic velocity \(U\),
\item time with \(R/U\), where \(R\) is the characteristic length scale,
\item pressure with \(\rho U^2\).
\end{itemize}
Reynolds number \(\text{Re} = R U \rho/ \mu\) and Froude number \(\text{Fr} = g R/U\).
} % makeoproblem
%
\makeanswer{problem:fluids:ps3:q2}{
Let us start by dividing by \(g \rho\), to make all terms (most obviously the \(\zcap\) term) dimensionless.
%
\begin{equation}\label{eqn:continuumProblemSet3:250}
\inv{g} \PD{t}{\Bu} + \inv{g} (\Bu \cdot \spacegrad) \Bu = - \inv{g \rho} \spacegrad p + \frac{\mu}{g \rho} \spacegrad^2 \Bu + \zcap.
\end{equation}
%
Our suggested replacements are
%
\begin{equation}\label{eqn:continuumProblemSet3:270}
\begin{aligned}
\Bu &= U \Bu' \\
\PD{t}{} &= \frac{U}{R} \PD{t'}{} \\
p &= \rho U^2 p' \\
\spacegrad &= \inv{R} \spacegrad'.
\end{aligned}
\end{equation}
%
Plugging these in we have
%
\begin{equation}\label{eqn:continuumProblemSet3:290}
\frac{U^2}{g R} \PD{t'}{\Bu'} + \frac{U^2}{g R} (\Bu' \cdot \spacegrad') \Bu' = - \frac{\cancel{\rho} U^2}{g \cancel{\rho} R} \spacegrad' p' + \frac{\mu U}{g \rho R^2} {\spacegrad'}^2 \Bu' + \zcap.
\end{equation}
%
Making a \(\text{Fr} = gR/U\) replacement, using the Froude number, we have
%
\begin{equation}\label{eqn:continuumProblemSet3:310}
\frac{U}{\text{Fr}} \PD{t'}{\Bu'} + \frac{U}{\text{Fr}} (\Bu' \cdot \spacegrad') \Bu' = - \frac{U}{\text{Fr}} \spacegrad' p' + \frac{\mu }{\text{Fr} \rho R} {\spacegrad'}^2 \Bu' + \zcap.
\end{equation}
%
Scaling by \(\text{Fr}/U\) we tidy things up a bit, and also allow for insertion of the Reynold's number
%
\begin{equation}\label{eqn:continuumProblemSet3:330}
\PD{t'}{\Bu'} + (\Bu' \cdot \spacegrad') \Bu' = - \spacegrad' p' + \frac{1}{\text{Re}} {\spacegrad'}^2 \Bu' + \frac{\text{Fr}}{U}\zcap.
\end{equation}
%
Observe that the dimensions of Froude's number is that of velocity
%
\begin{equation}\label{eqn:continuumProblemSet3:350}
[\text{Fr}] = [g] T = \frac{L}{T},
\end{equation}
%
so that the end result is dimensionless as desired.  We also see that Froude's number, characterizes the significance of the body force for fluid flow at the characteristic velocity.  This is consistent with \citep{wiki:froudeNumber} where it was stated that the Froude number is used to determine the resistance of a partially submerged object moving through water, and permits the comparison of objects of different sizes (complete with pictures of canoes of various sizes that Froude built for such study).
} % end answer
