%
% Copyright � 2012 Peeter Joot.  All Rights Reserved.
% Licenced as described in the file LICENSE under the root directory of this GIT repository.
%

%
%
\label{chap:continuumL21}
\section{Stability}
%\section{Stability.  Rayleigh-Benard problem}
\subsection{Stability.  Some graphical illustrations}

What do we mean by stability?  A configuration is stable if after a small disturbance it returns to its original position.  A couple systems to consider are shown in \cref{fig:continuumL21:continuumL21Fig1a}, \cref{fig:continuumL21:continuumL21Fig1b} and \cref{fig:continuumL21:continuumL21Fig1c}.

\imageFigure{../figures/phy454-continuumechanics/lec21_Stable_well_configurationFig1a}{Stable well configuration}{fig:continuumL21:continuumL21Fig1a}{0.3}
\imageFigure{../figures/phy454-continuumechanics/lec21_Instable_peak_configurationFig1b}{Instable peak configuration}{fig:continuumL21:continuumL21Fig1b}{0.3}
\imageFigure{../figures/phy454-continuumechanics/lec21_Stable_tabletop_configurationFig1c}{Stable tabletop configuration}{fig:continuumL21:continuumL21Fig1c}{0.3}

We can examine how a displacement \(\delta x\) changes with time after making it.  In a stable configuration without friction we will induce an oscillation as plotted in \cref{fig:continuumL21:continuumL21Fig2a} for the parabolic configuration.  With friction we will have a damping effect.  This is plotted for the parabolic well in \cref{fig:continuumL21:continuumL21Fig2b}.

\imageFigure{../figures/phy454-continuumechanics/lec21_Displacement_time_evolution_in_undamped_well_systemFig2a}{Displacement time evolution in undamped well system}{fig:continuumL21:continuumL21Fig2a}{0.3}
\imageFigure{../figures/phy454-continuumechanics/lec21_Displacement_time_evolution_in_damped_well_systemFig2b}{Displacement time evolution in damped well system}{fig:continuumL21:continuumL21Fig2b}{0.3}

For the inverted parabola our displacement takes the form of \cref{fig:continuumL21:continuumL21Fig3}
\imageFigure{../figures/phy454-continuumechanics/lec21_Time_evolution_of_displacement_in_instable_parabolic_configurationFig3}{Time evolution of displacement in instable parabolic configuration}{fig:continuumL21:continuumL21Fig3}{0.2}

For the ball on the table, assuming some friction that stops the ball, fairly quickly, we will have a displacement as illustrated in \cref{fig:continuumL21:continuumL21Fig3b}
\imageFigure{../figures/phy454-continuumechanics/lec21_Time_evolution_of_displacement_in_tabletop_configurationFig3b}{Time evolution of displacement in tabletop configuration}{fig:continuumL21:continuumL21Fig3b}{0.2}

\section{Characterizing stability}

Let us suppose that our displacement can be described in exponential form
%
\begin{equation}\label{eqn:continuumL21:10}
\delta x \sim e^{\sigma t}
\end{equation}
%
where \(\sigma\) is the \textit{growth rate of perturbation}, and is in general a complex number of the form
%
\begin{equation}\label{eqn:continuumL21:30}
\sigma = \sigma_{\text{R}} + i \sigma_{\text{I}}
\end{equation}
%
\subsection{Case I.  Oscillatory unstability}

A system of the form
%
\begin{equation}\label{eqn:stability:310}
\begin{aligned}
\sigma_{\text{R}} &= 0 \\
\sigma_{\text{I}} &> 0
\end{aligned}
\end{equation}
%
\textit{oscillatory unstable}.  An example of this is the undamped parabolic system illustrated above.

\subsection{Case II.  Marginal unstability}
%
\begin{equation}\label{eqn:stability:330}
\begin{aligned}
\delta x
&\sim e^{\sigma_{\text{R}} t} e^{i \sigma_{\text{I}} t} \\
&\sim e^{\sigma_{\text{R}} t} \left( \cos \sigma_{\text{I}} t + i \sin \sigma_{\text{I}} t \right)
\end{aligned}
\end{equation}
%
We will call systems of the form
%
\begin{equation}\label{eqn:stability:350}
\begin{aligned}
\sigma_{\text{I}} &= 0 \\
\sigma_{\text{R}} &> 0
\end{aligned}
\end{equation}
%
\textit{marginally unstable}.  We can have unstable systems with \(\sigma_{\text{I}} \ne 0\) but still \(\sigma_{\text{R}} > 0\), but these are less common.

\subsection{Case III.  Neutral stability}
%
\begin{equation}\label{eqn:stability:370}
\begin{aligned}
\sigma &= 0 \\
\sigma_{\text{R}} &= \sigma_{\text{I}} = 0
\end{aligned}
\end{equation}
%
An example of this was the billiard table example where the ball moved to a new location on the table after being bumped slightly.

\section{A mathematical description}

For a discussion of stability in fluids we will not only have to incorporate the Navier-Stokes equation as we have done, but will also have to bring in the heat equation.  Unfortunately that is not in the scope of this course to derive.  Let us consider as system heated on a bottom plate, and consider the fluid and convection due to heating.  This system is illustrated in \cref{fig:continuumL21:continuumL21Fig4}

\imageFigure{../figures/phy454-continuumechanics/lec21_Fluid_in_cavity_heated_on_the_bottom_plateFig4}{Fluid in cavity heated on the bottom plate}{fig:continuumL21:continuumL21Fig4}{0.2}

We start with Navier-Stokes as normal
%
\begin{equation}\label{eqn:continuumL21:50}
\rho \PD{t}{\Bu} + \rho (\Bu \cdot \spacegrad) \Bu = - \spacegrad p + \mu \spacegrad^2 \Bu - \rho \zcap g.
\end{equation}
%
For steady state with \(\Bu = 0\) initially (our base state), we will call the following the equation of the base state

\boxedEquation{eqn:continuumL21:70}{
\spacegrad p_s = -\rho_s \zcap g
}

We will allow perturbations of each of our variables
%
\begin{equation}\label{eqn:stability:390}
\begin{aligned}
\Bu &= \Bu_{\text{base}} + \delta \Bu = 0 + \delta \Bu \\
p &= p_s + \delta p \\
\rho &= \rho_s + \delta \rho
\end{aligned}
\end{equation}
%
After perturbation Navier-Stokes takes the form
%
%
\begin{dmath}\label{eqn:continuumL21:90}
(\rho_s + \delta \rho )\PD{t}{(0 + \delta \Bu)} + (\rho_s + \delta \rho) ((0 + \delta \Bu) \cdot \spacegrad) (0 + \delta \Bu) =
- \spacegrad (p_s + \delta p) + \mu \spacegrad^2 (0 + \delta \Bu) - (\rho_s + \delta \rho) \zcap g
\end{dmath}
%
Retaining only terms that are of first order of smallness.
%
\begin{equation}\label{eqn:continuumL21:110}
\rho_s \PD{t}{\delta \Bu} = - \spacegrad p_s - \spacegrad \delta p + \mu \spacegrad^2 \delta \Bu - \rho_s \zcap g - \delta \rho \zcap g
\end{equation}
%
applying our equation of base state \eqnref{eqn:continuumL21:70}, we have
%\spacegrad p_s = -\rho_s \zcap g
\begin{equation}\label{eqn:continuumL21:110b}
\rho_s \PD{t}{\delta \Bu} = \cancel{\rho_s \zcap g} - \spacegrad \delta p + \mu \spacegrad^2 \delta \Bu - \cancel{\rho_s \zcap g} - \delta \rho \zcap g,
\end{equation}
%
or
%
\begin{equation}\label{eqn:continuumL21:110c}
\rho_s \PD{t}{\delta \Bu} = - \spacegrad \delta p + \mu \spacegrad^2 \delta \Bu - \delta \rho \zcap g.
\end{equation}
%
we can write
%
\begin{equation}\label{eqn:continuumL21:130}
\left( \PD{t}{} - \nu \spacegrad^2 \right) \delta \Bu = -\inv{\rho_s} \spacegrad \delta p - \frac{\delta \rho}{\rho_s} \zcap g
\end{equation}
%
Applying the divergence operation on both sides, and using \(\spacegrad \cdot \Bu = 0\) so that \(\spacegrad \cdot \delta \Bu = 0\) we have
%
\begin{equation}\label{eqn:continuumL21:150}
\left( \PD{t}{} - \nu \spacegrad^2 \right) \cancel{\spacegrad \cdot \delta \Bu} = -\inv{\rho_s} \spacegrad^2 \delta p - (\zcap \cdot \spacegrad ) \frac{\delta \rho}{\rho_s} g,
\end{equation}
%
or
%
\begin{equation}\label{eqn:continuumL21:170}
\inv{\rho_s} \spacegrad^2 \delta p = - (\zcap \cdot \spacegrad ) \frac{\delta \rho}{\rho_s} g.
\end{equation}
%
Assuming that \(\rho_s\) is constant (actually that is already been done above), we can cancel it, leaving
%
\begin{equation}\label{eqn:continuumL21:190}
\spacegrad^2 \delta p = - (\zcap \cdot \spacegrad ) g \delta \rho = -g \PD{z}{} \delta \rho.
\end{equation}
%
operating once more with \(\PDi{z}{}\) we have
%
\begin{equation}\label{eqn:continuumL21:210}
\spacegrad^2 \PD{z}{\delta p} = -g \PDSq{z}{\delta \rho}.
\end{equation}
%
Going back to \eqnref{eqn:continuumL21:130} and taking only the \(z\) component we have
%
\begin{equation}\label{eqn:continuumL21:230}
\left( \PD{t}{} - \nu \spacegrad^2 \right) \delta w = -\inv{\rho_s} \PD{z}{\delta p} - \frac{\delta \rho}{\rho_s} g
\end{equation}
%
\begin{equation}\label{eqn:stability:410}
\begin{aligned}
\left( \PD{t}{} - \nu \spacegrad^2 \right) \spacegrad^2 \delta w
&= -\inv{\rho_s} \PD{z}{ \spacegrad^2 \delta p} - \frac{g}{\rho_s} \spacegrad^2 \delta \rho \\
&= -\frac{g}{\rho_s} \PDSq{z}{\delta \rho} - \frac{g}{\rho_s} \spacegrad^2 \delta \rho \\
&=
-\frac{g}{\rho_s} \left(
\PDSq{x}{}
+\PDSq{y}{}
\right)
\delta \rho \\
&=
g \alpha \left(
\PDSq{x}{}
+\PDSq{y}{}
\right)
\delta T
\end{aligned}
\end{equation}
%
in the last step we use the following assumed relation for temperature
%
\begin{equation}\label{eqn:continuumL21:250}
\delta \rho = - \rho_s \alpha \delta T.
\end{equation}
%
Here \(\alpha\) is the coefficient of thermal expansion.  This is just a statement that expansion and temperature are related (as we heat something, it expands), with the ratio of the density change relative to the original being linearly related to the change in temperature.

We have finally
%
\begin{equation}\label{eqn:continuumL21:290}
\left( \PD{t}{} - \nu \spacegrad^2 \right) \spacegrad^2 \delta w
=
g \alpha \left(
\PDSq{x}{}
+\PDSq{y}{}
\right)
\delta T.
\end{equation}
%
Solving this is the Rayleigh-Benard instability problem.

While this is a fourth order differential equation, it is still the same sort of problem logically as we have been working on.  Our boundary value conditions at \(z = 0\) are
%
\begin{equation}\label{eqn:continuumL21:430}
u, v, w, \delta u, \delta v, \delta w = 0.
\end{equation}
%
Also relevant will be a similar equation relating temperature and fluid flow rate
%
\begin{equation}\label{eqn:continuumL21:270}
\left( \PD{t}{} - \kappa \spacegrad^2 \right) \delta T = \Delta T \frac{\delta w}{d},
\end{equation}
%
which we will cover in the next (and final) lecture of the course.
