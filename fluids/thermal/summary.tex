%
% Copyright � 2012 Peeter Joot.  All Rights Reserved.
% Licenced as described in the file LICENSE under the root directory of this GIT repository.
%
\section{Summary}
\subsection{Stability}

We characterized stability in terms of displacements writing

\begin{equation}\label{eqn:continuumFluidsReview:2770}
\delta x = e^{(\sigma_R + i \sigma_I) t}
\end{equation}

and defining
\begin{enumerate}
\item Oscillatory unstability.  \(\sigma_{\text{R}} = 0, \sigma_{\text{I}} > 0\).
\item Marginal unstability.  \(\sigma_{\text{I}} = 0, \sigma_{\text{R}} > 0\).
\item Neutral stability.  \(\sigma_{\text{I}} = 0, \sigma_{\text{R}} = 0\).
\end{enumerate}

\subsection{Thermal stability: Rayleigh-Benard problem}

We considered the Rayleigh-Benard problem, looking at thermal effects in a cavity.  Assuming perturbations of the form

\begin{equation}\label{eqn:summary:3130}
\begin{aligned}
\Bu &= \Bu_{\text{base}} + \delta \Bu = 0 + \delta \Bu \\
p &= p_s + \delta p \\
\rho &= \rho_s + \delta \rho
\end{aligned}
\end{equation}

and introducing an equation for the base state

\begin{equation}\label{eqn:continuumFluidsReview:2790}
\spacegrad p_s = -\rho_s \zcap g
\end{equation}

we found

\begin{equation}\label{eqn:continuumFluidsReview:2810}
\left( \PD{t}{} - \nu \spacegrad^2 \right) \delta \Bu = -\inv{\rho_s} \spacegrad \delta p - \frac{\delta \rho}{\rho_s} \zcap g
\end{equation}

Operating on this with \(\PDi{z}{} \spacegrad \cdot ()\) we find

\begin{equation}\label{eqn:continuumFluidsReview:2830}
\spacegrad^2 \PD{z}{\delta p} = -g \PDSq{z}{\delta \rho},
\end{equation}

from which we apply back to \eqnref{eqn:continuumFluidsReview:2810} and take just the z component to find

\begin{equation}\label{eqn:continuumFluidsReview:2850}
\left( \PD{t}{} - \nu \spacegrad^2 \right) \delta w = -\inv{\rho_s} \PD{z}{\delta p} - \frac{\delta \rho}{\rho_s} g
\end{equation}

With an assumption that density change and temperature are linearly related
\begin{equation}\label{eqn:continuumFluidsReview:2870}
\delta \rho = - \rho_s \alpha \delta T,
\end{equation}

and operating with the Laplacian we end up with a relation that follows from the momentum balance equation

\begin{equation}\label{eqn:continuumFluidsReview:2890}
\left( \PD{t}{} - \nu \spacegrad^2 \right) \spacegrad^2 \delta w
=
g \alpha \left(
\PDSq{x}{}
+\PDSq{y}{}
\right)
\delta T.
\end{equation}

We also applied our perturbation to the energy balance equation

\begin{equation}\label{eqn:continuumFluidsReview:2910}
\PD{t}{T} + (\Bu \cdot \spacegrad) T = \kappa \spacegrad^2 T
\end{equation}

We determined that the base state temperature obeyed

\begin{equation}\label{eqn:continuumFluidsReview:2930}
\kappa \PDSq{z}{} T_s = 0,
\end{equation}

with solution

\begin{equation}\label{eqn:continuumFluidsReview:2950}
T_s = T_0 - \frac{\Delta T}{d} z.
\end{equation}

This and application of the perturbation gave us

\begin{equation}\label{eqn:continuumFluidsReview:2970}
\PD{t}{\delta T} + \delta \Bu \cdot \spacegrad T_s = \kappa \spacegrad^2 \delta T.
\end{equation}

We used this to non-dimensionalize with

\begin{equation}\label{eqn:continuumFluidsReview:2990}
\begin{array}{l l}
x,y,z & \quad \mbox{with \(d\)} \\
t & \quad \mbox{with \(d^2/\nu\)} \\
\delta w & \quad \mbox{with \(\kappa/d\)} \\
\delta T & \quad \mbox{with \(\Delta T\)}
\end{array}
\end{equation}

And found (primes dropped)

\begin{subequations}
\begin{equation}\label{eqn:continuumFluidsReview:3010}
\spacegrad^2 \left( \PD{t}{} - \spacegrad^2 \right) \delta w = \calR \left( \PDSq{x}{} +\PDSq{y}{} \right) \delta T
\end{equation}
\begin{equation}\label{eqn:continuumFluidsReview:3030}
\left( \text{P}_r \PD{t}{} - \spacegrad^2 \right) \delta T = \delta w,
\end{equation}
\end{subequations}

where we have introduced the Rayleigh number and Prandtl number's
\begin{subequations}
\begin{equation}\label{eqn:continuumFluidsReview:3050}
\calR = \frac{g \alpha \Delta T d^3}{\nu \kappa},
\end{equation}
\begin{equation}\label{eqn:continuumFluidsReview:3070}
\text{P}_r = \frac{\nu}{\kappa}
\end{equation}
\end{subequations}

We were able to construct some approximate solutions for a problem similar to these equations using an assumed solution form

\begin{equation}\label{eqn:continuumFluidsReview:3090}
\begin{aligned}
\delta w &= w(z) e^{ i ( k_1 x + k_2 y) + \sigma t} \\
\delta T &= \Theta(z) e^{ i ( k_1 x + k_2 y) + \sigma t}
\end{aligned}
\end{equation}

Using these we are able to show that our PDEs are similar to that of

\begin{equation}\label{eqn:continuumFluidsReview:3110}
\evalbar{w = D^2 w = D^4 w}{z = 0, 1} = 0,
\end{equation}

where \(D = \PDi{z}{}\).  Using the trig solutions that fall out of this we were able to find the constraint

\begin{equation}\label{eqn:continuumFluidsReview:490a}
0 =
\begin{vmatrix}
\left( n^2 \pi^2 + k^2 \right)^2 + \sigma \left( n^2 \pi^2 + k^2\right) & -\calR k^2 \\
-1 & n^2 \pi^2 + k^2 + \text{P}_r \sigma
\end{vmatrix},
\end{equation}

which for \(\sigma = 0\), this gives us the critical value for the Rayleigh number

\begin{equation}\label{eqn:continuumFluidsReview:510a}
\calR = \frac{(k^2 + n^2 \pi^2)^3}{k^2},
\end{equation}

which is the boundary for thermal stability or instability.

The end result was a lot of manipulation for which we did not do any sort of applied problems.  It looks like a theory that requires a lot of study to do anything useful with, so my expectation is that it will not be covered in detail on the exam.  Having some problems to know why we spent two days on it in class would have been nice.

