%
% Copyright � 2012 Peeter Joot.  All Rights Reserved.
% Licenced as described in the file LICENSE under the root directory of this GIT repository.
%
\section{Summary}
\subsection{Bernoulli equation}

With the body force specified in gradient for
%
\begin{equation}\label{eqn:continuumFluidsReview:2290}
\Bg = - \spacegrad \chi
\end{equation}
%
and utilizing the vector identity
%
\begin{equation}\label{eqn:continuumFluidsReview:2310}
(\Bu \cdot \spacegrad) \Bu = \spacegrad \cross (\spacegrad \cross \Bu) + \spacegrad \left( \inv{2} \Bu^2 \right),
\end{equation}
%
we are able to show that the steady state, irrotational, non-viscous Navier-Stokes equation takes the form
%
\begin{equation}\label{eqn:continuumFluidsReview:2330}
\spacegrad \left( \frac{p}{\rho} + \chi + \inv{2} \Bu^2 \right) = 0.
\end{equation}
%
or
\begin{equation}\label{eqn:continuumFluidsReview:2350}
\frac{p}{\rho} + \chi + \inv{2} \Bu^2 = \text{constant}
\end{equation}
%
This is the Bernoulli equation, and the constants introduce the concept of streamline.

\FIXME{I think this could probably be used to get a better idea what the tap stream radius is, than the method used above.  Consider the streamline along the outermost surface.  That way you do not have to assume that the flow is at the average velocity rate uniformly throughout the stream.  Try this later}

