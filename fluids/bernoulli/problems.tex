%
% Copyright � 2012 Peeter Joot.  All Rights Reserved.
% Licenced as described in the file LICENSE under the root directory of this GIT repository.
%
\makeproblem{Curve for tap discharge}{problem:fluids:review:q2}{
Use Bernoulli's theorem to get a rough idea what the curve for water coming out a tap would be.
} % makeproblem

\makeanswer{problem:fluids:review:q2}{
Suppose we measure the volume flux, putting a measuring cup under the tap, and timing how long it takes to fill up.  We then measure the radii at different points.  This can be done from a photo as in \cref{fig:tapFlowCropped6cm:tapFlowCropped6cmFig3}.

\imageFigure{../../figures/phy454/tapFlowMeasurementCropped6cmFig3}{Tap flow measurement}{fig:tapFlowCropped6cm:tapFlowCropped6cmFig3}{0.3}

After making the measurement, we can get an idea of the velocity between two points given a velocity estimate at a point higher in the discharge.  For a plain old falling mass, our final velocity at a point measured from where the velocity was originally measured can be found from Newton's law

\begin{equation}\label{eqn:continuumFluidsReview:1910}
\Delta v = g t
\end{equation}
\begin{equation}\label{eqn:continuumFluidsReview:1930}
\Delta z = \inv{2} g t^2 + v_0 t
\end{equation}

Solving for \(v_f = v_0 + \Delta v\), we find

\begin{equation}\label{eqn:continuumFluidsReview:1950}
v_f = v_0 \sqrt{ 1 + \frac{2 g \Delta z}{v_0^2} }.
\end{equation}

Mass conservation gives us

\begin{equation}\label{eqn:continuumFluidsReview:1970}
v_0 \pi R^2 = v_f \pi r^2
\end{equation}

or

\begin{equation}\label{eqn:continuumFluidsReview:1990}
r(\Delta z) = R \sqrt{ \frac{v_0}{v_f} } = R \left( 1 + \frac{2 g \Delta z}{v_0^2} \right)^{-1/4}.
\end{equation}

For the image above I measured a flow rate of about 250 ml in 10 seconds.  With that, plus the measured radii at 0 and \(6 \text{cm}\), I calculated that the average fluid velocity was \(0.9 \text{m}/\text{s}\), vs a free fall rate increase of \(1.3 \text{m}/\text{s}\).  Not the best match in the world, but that is to be expected since the velocity has been considered uniform throughout the stream profile, which would not actually be the case.  A proper treatment would also have to treat viscosity and surface tension.

In \cref{fig:tapFlowCropped6cm:tapFlowCropped6cmFig4} is a plot of the measured radial distance compared to what was computed with \eqnref{eqn:continuumFluidsReview:1990}.  The blue line is the measured width of the stream as measured, the red is a polynomial curve fitted to the raw data, and the green is the computed curve above.

\imageFigure{../../figures/phy454/tapFlowComparison_of_measured_stream_radii_and_calculatedCropped6cmFig4}{Comparison of measured stream radii and calculated}{fig:tapFlowCropped6cm:tapFlowCropped6cmFig4}{0.3}
} % end answer
