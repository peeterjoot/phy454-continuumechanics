%
% Copyright � 2012 Peeter Joot.  All Rights Reserved.
% Licenced as described in the file LICENSE under the root directory of this GIT repository.
%
\section{Derivation}

We start with Navier-Stokes in vector form
%
\begin{equation}\label{eqn:continuumL18:290}
\PD{t}{\Bu} + (\Bu \cdot \spacegrad) \Bu = -\spacegrad \left( \frac{p}{\rho} \right) + \nu \spacegrad^2 \Bu + \Bg.
\end{equation}
%
Writing the body force as a potential
%
\begin{equation}\label{eqn:continuumL18:310}
\Bg = -\spacegrad \chi,
\end{equation}
%
so that we have
%
\begin{equation}\label{eqn:continuumL18:330}
\PD{t}{\Bu} + (\Bu \cdot \spacegrad) \Bu = -\spacegrad \left( \frac{p}{\rho} + \chi \right) + \nu \spacegrad^2 \Bu .
\end{equation}
%
Using the vector identity
%
\begin{equation}\label{eqn:continuumL18:350}
(\Bu \cdot \spacegrad) \Bu = \spacegrad \cross (\spacegrad \cross \Bu) + \spacegrad \left( \inv{2} \Bu^2 \right),
\end{equation}
%
we can write
%
\begin{equation}\label{eqn:continuumL18:370}
\PD{t}{\Bu} +
\spacegrad \cross (\spacegrad \cross \Bu) + \spacegrad \left( \inv{2} \Bu^2 \right)
= -\spacegrad \left( \frac{p}{\rho} + \chi \right) + \nu \spacegrad^2 \Bu .
\end{equation}
%
If we consider the non-viscous region of the flow (far from the boundary layer), we can kill the Laplacian term.  Again, considering only the steady state, and assuming that we have irrotational flow (\(\spacegrad \cross \Bu = 0\)) in the non-viscous region, we have
%
\begin{equation}\label{eqn:continuumL18:390}
\spacegrad \left( \frac{p}{\rho} + \chi + \inv{2} \Bu^2 \right) = 0.
\end{equation}
%
or

\boxedEquation{eqn:continuumL18:410}{
\frac{p}{\rho} + \chi + \inv{2} \Bu^2 = \text{constant}
}

This is \textAndIndex{Bernoulli's equation}.

\paragraph{On streamlines}

The derivation of Bernoulli's equation in \S 5 of \citep{landau1987course} does not mention irrotational flow.
The statement of Bernoulli's theorem in \citep{batchelor1967introduction} is similar, related only to streamlines, with irrotational flow considered later as a special case.
The Landau derivation considers steady state flows along streamlines, and
argues that since the velocites are tangential to any streamline, and because \( \Bu \cross \lr{ \spacegrad \cross \Bu } \) is perpendicular to \( \Bu \), the projection of that curl term on the streamline direction is zero.  That leaves a zero for the projection of the gradient along the streamline direction
%
\begin{equation}\label{eqn:continuumL18:391}
\spacegrad \lr{ \frac{p}{\rho} + \chi + \inv{2} \Bu^2 },
\end{equation}
%
From this \cref{eqn:continuumL18:410} is a statement that the quantity in the gradient operation is constant along any streamline, even for rotational flows.

Observe that the general form of Bernoulli's theorem follows from the fact that
%
\begin{equation}\label{eqn:continuumL18:391}
\Bu \cdot \lr{ \Bu \cross \lr{ \spacegrad \cross \Bu } } = 0,
\end{equation}
%
where the mathematical statement of the theorem is

\boxedEquation{eqn:continuumL18:410}{
\lr{ \Bu \cdot \spacegrad} \lr{ \frac{p}{\rho} + \chi + \inv{2} \Bu^2} = 0.
}

The operation \( \Bu \cdot \spacegrad \) is the projection of the gradient onto the streamline (i.e. the lines formed from the tangents to the velocities along the steady state flow paths).  Along those lines \cref{eqn:continuumL18:410} hold, where the constant can be streamline dependent.