%
% Copyright � 2012 Peeter Joot.  All Rights Reserved.
% Licenced as described in the file LICENSE under the root directory of this GIT repository.
%
% L13
\section{Normals and tangents.}
%
\FIXME{
This did not fit with much of the other lecture material.  Looking at the big picture, it is not clear why it was covered, so have moved it to an appendix.  Does not really seem to be a prerequisite for the surface tension content despite that being when it was covered in class.
}

Consider a surface with some variation as in \cref{fig:continuumL13:continuumL13Fig13}
%
\imageFigure{../figures/phy454-continuumechanics/lec13_variable_surface_geometriesFig13}{Variable surface geometries.}{fig:continuumL13:continuumL13Fig13}{0.2}
%
We can construct an equation for the surface
%
\begin{equation}\label{eqn:continuumL13:160}
z = h(x, t),
\end{equation}
%
or equivalently
%
\begin{equation}\label{eqn:continuumL13:180}
\phi = z - h(x, t) = 0.
\end{equation}
%
If \(d\) is the average height, with the \(\eta(x,t)\) the variation of the height from this average, we can also write
%
\FIXME{had \(z = d + \eta(x, t)\) on board?}
\begin{equation}\label{eqn:continuumL13:200}
h = d + \eta(x, t)
\end{equation}
%
and for the surface
%
\FIXME{had: \(z - \eta(x, t) = \text{constant}\) on the board?}
\begin{equation}\label{eqn:continuumL13:220}
\phi = d - \eta(x, t) = 0
\end{equation}
%
We can generalize this and define a surface function as one that satisfies
%
\begin{equation}\label{eqn:continuumL13:220b}
\phi = d - \eta(x, t) = \text{constant}.
\end{equation}
%
Consider a small section of a 2D surface as in \cref{fig:continuumL13:continuumL13Fig14}
%
\imageFigure{../figures/phy454-continuumechanics/lec13_a_vector_differential_elementFig14}{A vector differential element.}{fig:continuumL13:continuumL13Fig14}{0.2}
%
With \(\phi = \text{constant}\) on the surface, we have for \(\phi = \phi(x, y, z)\)
%
\begin{equation}\label{eqn:continuumL13:240}
d\phi = 0
\end{equation}
%
or, in coordinates
%
\begin{equation}\label{eqn:normalsAndTangents:400}
\begin{aligned}
d\phi &=
\PD{x}{\phi} dx
+\PD{y}{\phi} dy
+\PD{z}{\phi} dz \\
&= \spacegrad \phi \cdot d\Br
\end{aligned}
\end{equation}
%
Pictorially we see that \(d\Br\) is tangential to the surface, but since we also have
%
\begin{equation}\label{eqn:continuumL13:360}
\spacegrad \phi \cdot d\Br = 0,
\end{equation}
%
the implication is that the gradient is normal to the surface
%
\begin{equation}\label{eqn:continuumL13:380}
d\Br \perp \spacegrad \phi.
\end{equation}
%
We can therefore construct the unit normal by scaling the gradient
%
\begin{equation}\label{eqn:continuumL13:260}
\ncap = \frac{\spacegrad \phi}{\Abs{\spacegrad \phi}},
\end{equation}
%
since the direction of \(\ncap\) is \(\spacegrad \phi\).

For example, in our case where \(\phi = y - h(x, t)\), we have
%
\begin{equation}\label{eqn:continuumL13:280}
\spacegrad \phi = \xcap \left( -\PD{x}{h} \right) + \ycap
\end{equation}
%
This has norm
%
\begin{equation}\label{eqn:continuumL13:300}
\Abs{\spacegrad \phi} = \sqrt{ 1 + \left( \PD{x}{h} \right)^2 }
\end{equation}
%
and our unit normal is
%
\begin{equation}\label{eqn:continuumL13:320}
\ncap =
\frac{\xcap \left( -\PD{x}{h} \right) + \ycap}{
\sqrt{ 1 + \left( \PD{x}{h} \right)^2 }
}
\end{equation}
%
By inspection we can also express the unit tangent, and we have for both
%
\begin{equation}\label{eqn:continuumL13:340}
\begin{aligned}
\ncap &=
\inv{
\sqrt{ 1 + \left( \PD{x}{h} \right)^2 }
}
\left( -\PD{x}{h}, 1 \right)  \\
\taucap &=
\inv{
\sqrt{ 1 + \left( \PD{x}{h} \right)^2 }
}
\left( 1, \PD{x}{h} \right)
\end{aligned}
\end{equation}
