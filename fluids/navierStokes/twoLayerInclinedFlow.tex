%
% Copyright � 2012 Peeter Joot.  All Rights Reserved.
% Licenced as described in the file LICENSE under the root directory of this GIT repository.
%

%s/\\br\(.\)/{(\1)}/g
\makeoproblem
%{Two layer inclined viscous flow with equal densities}
{Layered inclined viscous flow.}
{problem:fluids:twoLayerInclinedFlow}
{From \S 2 \citep{acheson1990elementary}}
{
This is a slight variation on what we did in class.

Our problem is illustrated in \cref{fig:twoLayerInclinedFlow:twoLayerInclinedFlowFig1} with a plane set at angle \(\alpha\), fluid depths of \(h^{(1)}\) and \(h^{(2)}\) respectively, and viscosities \(\mu^{(1)}\) and \(\mu^{(2)}\).
%
\imageFigure{../figures/phy454-continuumechanics/twoLayerInclinedFlowFig1}{Two fluids layers in inclined flow.}{fig:twoLayerInclinedFlow:twoLayerInclinedFlowFig1}{0.2}
} % makeoproblem
%
\makeanswer{problem:fluids:twoLayerInclinedFlow}{
%
%\unnumberedSubsection{Setup of the equations of motion for this system}
We have to setup of the equations of motion for this system.  We will write \(H = h^{(1)} + h^{(2)}\).  We have a pair of Navier-Stokes equations to solve
%
\begin{dmath}\label{eqn:twoLayerInclinedFlow:10}
\rho \ddt{\Bu^{(i)}} = \rho \PD{t}{\Bu^{(i)}} + \rho (\Bu^{(i)} \cdot \spacegrad) \Bu^{(i)}
= - \spacegrad p^{(i)} + \mu^{(i)} \spacegrad^2 \Bu^{(i)} + \mu^{(i)} \spacegrad (\spacegrad \cdot \Bu^{(i)}) + \rho \Bg.
\end{dmath}
%
Our steady state and incompressibility constraints break this into a few independent equations
%
\begin{equation}\label{eqn:twoLayerInclinedFlow:30}
\begin{aligned}
\rho \PD{t}{\Bu^{(i)}} &= 0 \\
\spacegrad \cdot \Bu^{(i)} &= 0 \\
\rho (\Bu^{(i)} \cdot \spacegrad) \Bu^{(i)} &= - \spacegrad p^{(i)} + \mu^{(i)} \spacegrad^2 \Bu^{(i)} + \rho \Bg.
\end{aligned}
\end{equation}
%
Let us require no components of the flows in the \(y\), or \(z\) directions initially.  As the equivalent of Newton's law for fluid flows, conservation of linear momentum requires that for our steady state problem we have \(u_y = u_z = 0\) for the flows in both fluid layers.

\FIXME{I am now so used to thinking about this as a symmetry issue from a Lagrangian context.  With no Lagrangian here, what would be a robust way treat this in fluids?  What is the Lagrangian for a fluid mechanics system?  I had imagine that it would be possible to set up a field Lagrangian with velocity fields}

Our problem is now reduced to a problem in four quantities (two velocities and two pressures).  With \(\Bu^{(i)} = (u^{(i)}, 0, 0)\) we can restate Navier-Stokes in coordinate form as
%
\begin{subequations}
\label{eqn:twoLayerInclinedFlow:50}
\begin{equation}\label{eqn:twoLayerInclinedFlow:110}
\partial_x u^{(i)}(x,y,z) = 0,
\end{equation}
\begin{equation}\label{eqn:twoLayerInclinedFlow:130}
\rho (u^{(i)} \partial_x u^{(i)} = - \partial_x p^{(i)} + \mu^{(i)} (\partial_{xx} + \partial_{yy} + \partial_{zz}) u^{(i)} + \rho g \sin\alpha,
\end{equation}
\begin{equation}\label{eqn:twoLayerInclinedFlow:150}
0 = - \partial_y p^{(i)} - \rho g \cos\alpha,
\end{equation}
\begin{equation}\label{eqn:twoLayerInclinedFlow:155}
0 = - \partial_z p^{(i)}.
\end{equation}
\end{subequations}
%
In order to solve this, we have eight simultaneous non-linear PDEs, four unknown functions, plus boundary conditions!

What are the boundary conditions?  One is the ``no-slip'' condition, the experimental observation that velocities match at the interfaces.  So we should have zero velocity for the fluid lying against the plane, and velocity matching between the fluids.  The air above the fluid will also be flowing along at the rate of the uppermost portion of the top layer, but we will neglect that effect (i.e. considering two layers of equal density and not three, with one having a separate density).  We also have matching of the traction vectors at the interfaces.

Writing this, it occurred to me that I did not fully understand what motivated the traction vector matching boundary value condition.  Talking to our Prof about this, the matching of the traction vectors at any point can be thought of as an observational issue, but this is also a force balance issue.  There is an induced velocity in the direction of the traction vector at any given point.  For example, when we have unidirectional flow, we must have no normal component of the traction vector, and only a tangential component, because we have only the tangential flow.  It is probably reasonable to think about this roughly as the equivalent of matching both acceleration and velocity at the boundary, but because densities and viscosities vary, we have to match the traction vectors and not the acceleration itself.

Before continuing to solve our Navier-Stokes equations let us express the condition that the tangential component of the traction vectors match algebraically.

Dropping indices temporarily, for the normal to the surface \(\ncap = (n_1, n_2, n_3) = (0, 1, 0)\) we want to compute
%
\begin{equation}\label{eqn:twoLayerInclinedFlow:595}
\begin{aligned}
\tau_1
&= \sigma_{1 k} n_k \\
&= \sigma_{1 k} \delta_{2 k} \\
&= \sigma_{1 2} \\
&= -p \cancel{\delta_{1 2}} + 2 \mu e^{1 2} \\
&= \mu \left( \PD{y}{u_1} + \cancel{\PD{x}{u_2}} \right).
\end{aligned}
\end{equation}
%
So the tangential component of the traction vector is
%
\begin{equation}\label{eqn:twoLayerInclinedFlow:90}
\BT^{(i)} = \mu^{(i)} \PD{y}{u^{(i)}} \xcap.
\end{equation}
%
As noted above, this is in fact, the only component of the traction vector, since we do not have any non-horizontal flow.

Our boundary value conditions, what we need in addition to the Navier-Stokes equations of \eqnref{eqn:twoLayerInclinedFlow:50}, to solve our problem, are the matching at any interface of the following conditions
%
\begin{subequations}
\label{eqn:twoLayerInclinedFlow:70}
\begin{equation}\label{eqn:twoLayerInclinedFlow:175}
u^{(i)} = u^{(j)},
\end{equation}
\begin{equation}\label{eqn:twoLayerInclinedFlow:195}
p^{(i)} = p^{(j)},
\end{equation}
\begin{equation}\label{eqn:twoLayerInclinedFlow:215}
\mu^{(i)} \PD{y}{u^{(i)}} = \mu^{(i)} \PD{y}{u^{(j)}}.
\end{equation}
\end{subequations}
%
There are actually three interfaces to consider, that of the lower layer liquid with the inclined plane, the interface between the two fluid layers, and the interface between the upper layer fluid and the air above it.

%\unnumberedSubsection{Solving our equations of motion}
Starting with the simplest, the z-coordinate equation, of Navier-Stokes \eqnref{eqn:twoLayerInclinedFlow:155}, we can conclude that each of the pressures is not a function of z, so that we have
%
\begin{equation}\label{eqn:twoLayerInclinedFlow:235}
p^{(i)} = p^{(i)}(x, y).
\end{equation}
%
Using this, we can integrate our y-coordinate Navier-Stokes equation \eqnref{eqn:twoLayerInclinedFlow:150}, to find
%
\begin{equation}\label{eqn:twoLayerInclinedFlow:255}
p^{(i)} = - \rho g y \cos\alpha + f^{(i)}(x).
\end{equation}
%
At this point we can introduce the first boundary value constraint, that the pressures must match at the interfaces.  In particular, on the upper surface, where we have atmospheric pressure \(p_A\) our pressure is
%
\begin{equation}\label{eqn:twoLayerInclinedFlow:275}
p^{(2)}(H) = - \rho g H \cos\alpha + f^{(2)}(x) = p_A,
\end{equation}
%
so \(f^{(2)}\) is constant with value
%
\begin{equation}\label{eqn:twoLayerInclinedFlow:295}
f^{(2)}(x) = p_A + \rho g H \cos\alpha,
\end{equation}
%
which fully determines the density of the upper surface
%
\begin{equation}\label{eqn:twoLayerInclinedFlow:315}
p^{(2)}(y) = \rho g \cos\alpha (H - y) + p_A.
\end{equation}
%
Matching the pressure between the two layers of fluids we have
%
\begin{equation}\label{eqn:twoLayerInclinedFlow:615}
\begin{aligned}
p^{(1)}(h_1) &= - \rho g h_1 \cos\alpha + f^{(1)}(x) \\
             &= p^{(2)}(h_1) \\
             &= \rho g \cos\alpha (H - h_1) + p_A,
\end{aligned}
\end{equation}
%
so that our undetermined function \(f^{(1)}(x)\)
%
\begin{equation}\label{eqn:twoLayerInclinedFlow:335}
f^{(1)}(x) = \rho g H \cos\alpha  + p_A.
\end{equation}
%
This is an intuitively satisfying result.  With the densities equal, it seems sensible that the pressure would have a single functional form throughout both layers, dependent only on the total height \(y\), independent of the velocities and viscosities.  That is precisely what we find
%
\begin{equation}\label{eqn:twoLayerInclinedFlow:355}
p(y) = \rho g \cos\alpha (H - y) + p_A.
\end{equation}
%
Having solved for the pressure, we are now set to return to the remaining Navier-Stokes equations \eqnref{eqn:twoLayerInclinedFlow:110}, and \eqnref{eqn:twoLayerInclinedFlow:130} for this system.  From \eqnref{eqn:twoLayerInclinedFlow:110} we see that the non-linear term on the LHS of \eqnref{eqn:twoLayerInclinedFlow:130} is killed and also see that our velocities can only be functions of \(y\) and \(z\)
%
\begin{equation}\label{eqn:twoLayerInclinedFlow:375}
u^{(i)} = u^{(i)}(y, z).
\end{equation}
%
While more general solutions can likely be found, we will limit ourselves to looking only for solutions that are functions of \(y\).  From our solution to the pressure part of the problem \(p^{(i)} = p^{(i)}(y)\), we also see that the pressure term \(\partial_x p^{(i)}\) of \eqnref{eqn:twoLayerInclinedFlow:130} is killed.  We are left with just
%
\begin{equation}\label{eqn:twoLayerInclinedFlow:635}
\begin{aligned}
0 &= \mu^{(i)} (\partial_{xx} + \partial_{yy} + \partial_{zz}) u^{(i)} + \rho g \sin\alpha  \\
  &= \mu^{(i)} \partial_{yy} u^{(i)} + \rho g \sin\alpha \\
  &= \mu^{(i)} \frac{d^2}{dy^2} u^{(i)} + \rho g \sin\alpha.
\end{aligned}
\end{equation}
%
This is directly integrable, and we find for the velocities and traction vectors respectively
%
\begin{subequations}
\begin{equation}\label{eqn:twoLayerInclinedFlow:395}
u^{(i)} = -\frac{\rho g \sin\alpha }{2 \mu^{(i)}} y^2 + A^{(i)} y + B^{(i)},
\end{equation}
\begin{equation}\label{eqn:twoLayerInclinedFlow:415}
\tau_x^{(i)} = \mu^{(i)} \PD{y}{u^{(i)}} = - \rho g y \sin\alpha + \mu^{(i)} A^{(i)}.
\end{equation}
\end{subequations}
%
The boundary conditions left to exploit are
%
\begin{equation}\label{eqn:twoLayerInclinedFlow:435}
\begin{aligned}
u^{(1)}(0) &= 0 \\
u^{(1)}(h_1) &= u^{(2)}(h_1) \\
\tau_x^{(1)}(h_1) &= \tau_x^{(2)}(h_1) \\
\tau_x^{(2)}(H) &= 0.
\end{aligned}
\end{equation}
%
The first is the no-slip condition with the plane.  The last is an approximation that assumes the liquid is not producing a measurable force on the air above it.  The other two are for the interfaces between the two fluids.

From \(u^{(1)}(0) = 0\) we see immediately that we have \(B^{(1)} = 0\).  From the traction vector equality in the atmosphere, we have
%
\begin{equation}\label{eqn:twoLayerInclinedFlow:455}
0 = - \rho g H \sin\alpha + \mu^{(2)} A^{(2)},
\end{equation}
%
or
%
\begin{equation}\label{eqn:twoLayerInclinedFlow:475}
A^{(2)} = \frac{\rho g H \sin\alpha }{ \mu^{(2)} }.
\end{equation}
%
These reduce the problem to solving for two last integration constants, where our velocities are
%
\begin{equation}\label{eqn:twoLayerInclinedFlow:495}
\begin{aligned}
u^{(1)} &= -\frac{\rho g \sin\alpha }{2 \mu^{(1)}} y^2 + A^{(1)} y \\
u^{(2)} &= \frac{\rho g \sin\alpha }{2 \mu^{(2)}} \left( 2 H y -y^2 \right) + B^{(2)}.
\end{aligned}
\end{equation}
%
and our traction vectors are
%
\begin{equation}\label{eqn:twoLayerInclinedFlow:515}
\begin{aligned}
\tau_x^{(1)} &= - \rho g y \sin\alpha + \mu^{(1)} A^{(1)} \\
\tau_x^{(2)} &= \rho g \sin\alpha \left( H - y \right).
\end{aligned}
\end{equation}
%
Matching both at the interface (\(y = h_1\)) gives us
%
\begin{equation}\label{eqn:twoLayerInclinedFlow:535}
\begin{aligned}
-\frac{\rho g \sin\alpha }{2 \mu^{(1)}} h_1^2 + A^{(1)} h_1 &= \frac{\rho g \sin\alpha }{2 \mu^{(2)}} h_1 \left( 2 h_2 + h_1 \right) + B^{(2)} \\
- \rho g h_1 \sin\alpha + \mu^{(1)} A^{(1)} &= \rho g h_2 \sin\alpha.
\end{aligned}
\end{equation}
%
We find
%
\begin{equation}\label{eqn:twoLayerInclinedFlow:555}
A^{(1)} = \frac{\rho g H \sin\alpha }{ \mu^{(1)} },
\end{equation}
%
and
%
\begin{equation}\label{eqn:twoLayerInclinedFlow:655}
\begin{aligned}
B^{(2)}
&=
-\frac{\rho g \sin\alpha }{2 \mu^{(1)}} h_1^2 +
\frac{\rho g H \sin\alpha }{ \mu^{(1)} }
h_1 - \frac{\rho g \sin\alpha }{2 \mu^{(2)}} h_1 \left( 2 h_2 + h_1 \right) \\
&=
\frac{\rho g h_1 \sin\alpha }{2}
\left(
- \frac{h_1}{ \mu^{(1)}}
+ \frac{2 H}{ \mu^{(1)}}
- \frac{2 h_2 + h_1}{ \mu^{(2)}}
\right) \\
&=
\frac{\rho g h_1 \sin\alpha }{2} (2 h_2 + h_1) \left(
\inv{ \mu^{(1)}}
-\inv{ \mu^{(2)}}
\right).
\end{aligned}
\end{equation}
%
So, finally, we have
%
\begin{equation}\label{eqn:twoLayerInclinedFlow:575}
\begin{aligned}
u^{(1)}(y) &= \frac{\rho g \sin\alpha }{2 \mu^{(1)}} (2 H y -y^2) \\
u^{(2)}(y) &= \frac{\rho g \sin\alpha }{2 \mu^{(2)}} \left( 2 H y -y^2 + h_1 (2 h_2 + h_1)
%\frac{ \mu^{(2)} - \mu^{(1)} }{ \mu^{(1)} }
\left( \frac{\mu^{(2)}}{\mu^{(1)}} - 1 \right)
\right) \\
p(y) &= \rho g \cos\alpha (H - y) + p_A.
\end{aligned}
\end{equation}
%
The final result looks reasonable.  If the viscosities are equal then we have the same velocity profile in both layers.  That makes sense given the equal densities, since there would really be nothing that would then distinguish the two layers.
%As a follow-on to this calculation, I had like to re-do it allowing for different densities.  Specifically, I am curious how much the air in the neighborhood of some flowing water gets dragged by that flow.  It never occurred to me that this would occur, and I had like to plug in some numbers and see what the results are.  This should be something that can be modeled with two layers like this, one of fluid, one of air of a specific thickness (allowing pressure to vary due to the velocity gradient), and one final layer of air at a fixed pressure (atmospheric).  I would not expect that problem to be much harder than this one, although it may end up being worthwhile to let a computer algebra system do some of the grunt work to solve all the resulting equations.
} % end answer
