%
% Copyright � 2012 Peeter Joot.  All Rights Reserved.
% Licenced as described in the file LICENSE under the root directory of this GIT repository.
%

%
%
%
%\section{Newtonian fluids.  Mass conservation.  Constitutive relation.  Incompressible fluids}
\label{chap:continuumL9}
%\section{Review: Relative motion near a point in a fluid}
%
%Referring to \cref{fig:continuumL9:continuumL9fig1}
%\imageFigure{../figures/phy454-continuumechanics/lec9_velocity_displacements_at_a_fluid_pointFig1}{velocity displacements at a fluid point.}{fig:continuumL9:continuumL9fig1}{0.2}
%
%we write
%
%\begin{equation}\label{eqn:continuumL9:10}
%d\Bx' = d\Bx + d\Bu \delta t
%\end{equation}
%
%or in coordinate form
%
%
%\begin{dmath}\label{eqn:continuumL9:30}
%dx_i
%= dx_i + du_i \delta t
%= dx_i + \PD{x_j}{u_i} dx_j \delta t
%\end{dmath}
%
%We can now split the components of the gradient of \(u_i\) into symmetric and antisymmetric parts in the normal way
%
%
%\begin{dmath}\label{eqn:continuumL9:50}
%\PD{x_j}{u_i}
%=
%\inv{2} \left(
%\PD{x_j}{u_i}
%+\PD{x_i}{u_j}
%\right)
%+
%\inv{2} \left(
%\PD{x_j}{u_i}
%-\PD{x_i}{u_j}
%\right)
%\equiv e_{ij} + \omega_{ij}.
%\end{dmath}
%
\section{Antisymmetric term, the vorticity}
%
With
%
\begin{equation}\label{eqn:continuumL9:70}
\Bomega = \spacegrad \cross \Bu,
\end{equation}
%
we introduce the dual vector
%
\begin{equation}\label{eqn:continuumL9:90}
\BOmega = \Omega_k \Be_k = \inv{2} \Bomega
\end{equation}
%
defined according to
%
\begin{equation}\label{eqn:continuumL9:110}
\begin{aligned}
\Omega_1 &= \inv{2} \omega_{32} = \inv{2} \omega_1 \\
\Omega_2 &= \inv{2} \omega_{13} = \inv{2} \omega_2 \\
\Omega_3 &= \inv{2} \omega_{21} = \inv{2} \omega_3
\end{aligned}
\end{equation}
%
With
\begin{equation}\label{eqn:continuumL9:610}
\omega_{k}
= \epsilon_{ijk} \partial_i u_j
\end{equation}
%
we can write
\begin{equation}\label{eqn:continuumL9:590}
\Omega_k = \inv{2} \epsilon_{ijk} \partial_i u_j.
\end{equation}
%
In matrix form this becomes
%
\begin{equation}\label{eqn:continuumL9:130}
\omega_{ij} =
\begin{bmatrix}
0 & -\Omega_3 & \Omega_2 \\
\Omega_3 & 0 & -\Omega_1  \\
-\Omega_2 & \Omega_1 & 0
\end{bmatrix}.
\end{equation}
%
For the special case \(e_{ij} = 0\), our displacement equation in vector form becomes
%
\begin{equation}\label{eqn:continuumL9:150}
d\Bx' = d\Bx + \BOmega \cross d\Bx \delta t.
\end{equation}
%
Let us do a quick verification that this is all kosher.
%
\begin{equation}\label{eqn:newtonianFluidsAndMassConservationAndConstitutiveRelationAndIncompressibleFluids:730}
\begin{aligned}
(\BOmega \cross d\Bx)_i
&=
\Omega_r dx_s \epsilon_{rsi} \\
&=
\left(\inv{2} \epsilon_{abr} \partial_a u_b \right) dx_s \epsilon_{rsi} \\
&=
\inv{2} \partial_a u_b dx_s \delta^{[ab]}_{si} \\
&=
\inv{2} (
\partial_s u_i
-\partial_i u_s
) dx_s  \\
&=
\inv{2} \left(
-\PD{x_i}{u_s}
+\PD{x_s}{u_i}
\right) dx_s  \\
&=
\inv{2} \left(
\PD{x_j}{u_i}
-\PD{x_i}{u_j}
\right) dx_j  \\
&=
\omega_{ij} dx_j.
\end{aligned}
\end{equation}
%
All's good in the world of signs and indices.
%
\section{Symmetric term, the strain tensor}
%
Now let us look at the symmetric term.  With the initial volume
%
\begin{equation}\label{eqn:continuumL9:170}
dV = dx_1 dx_2 dx_3,
\end{equation}
%
and the final volume written assuming that we are working in our principle strain basis, we have (very much like the solids case)
%
\begin{equation}\label{eqn:newtonianFluidsAndMassConservationAndConstitutiveRelationAndIncompressibleFluids:750}
\begin{aligned}
dV'
&= dx_1' dx_2' dx_3' \\
&=
(1 + e_{11} \delta t) dx_1
+(1 + e_{22} \delta t) dx_2
+(1 + e_{33} \delta t) dx_3
\\
&=
(1 + (e_{11} + e_{22} + e_{33}) \delta t) dx_1 dx_2 dx_3 + O((\delta t)^2) \\
&=
\left(1 +
\left(
\PD{x_1}{u_1}
+\PD{x_2}{u_2}
+\PD{x_3}{u_3}
\right)
\delta t \right) dV \\
&=
\left(
1 + (\spacegrad \cdot \Bu)
\delta t
\right) dV \\
\end{aligned}
\end{equation}
%
So much like we expressed the relative change of volume in solids, we now can express the relative change of volume per unit time as
%
\begin{equation}\label{eqn:continuumL9:190}
\frac{dV' - dV}{dV \delta t} = \spacegrad \cdot \Bu,
\end{equation}
%
or
%
\begin{equation}\label{eqn:continuumL9:210}
\frac{\delta(dV)}{dV \delta t} = \spacegrad \cdot \Bu,
\end{equation}
%
We identify the divergence of the displacement as the relative change in volume per unit time.
%
\section{Newtonian Fluids}
%
\makedefinition{Newtonian Fluids}{dfn:continuumL9:230}{A fluid for which the rate of strain tensor is linearly related to stress tensor. \index{Newtonian fluid}}

For such a fluid, the \textAndIndex{constitutive relation} takes the form
%
\boxedEquation{eqn:continuumL9:250}{
\sigma_{ij} = - p \delta_{ij} + 2 \mu e_{ij},
}
%
where \(p\) is called the isotropic pressure, and \(\mu\) is the viscosity of the fluid.
For comparison, in solids we had
%
\begin{equation}\label{eqn:continuumL9:270}
\sigma_{ij} = \lambda e_{kk} \delta_{ij} + 2 \mu e_{ij}
\end{equation}
%
While we are allowing for rotation in the fluids (\(\omega_{ij}\)) that we did not consider for solids, we now impose a requirement that the strain tensor trace is not a function of the fluid displacements, with
%
\begin{equation}\label{eqn:continuumL9:630}
\lambda e_{kk} = \lambda \spacegrad \cdot \Bu = -p.
\end{equation}
%
What is the physical justification for this?  In words this was explained after class as the effect of rotation invariance with an attempt to measure the pressure at a given point in the fluid.  It does not matter what direction we place our pressure measurement device at a given fixed location in the fluid.  Note that this does not mean the pressure itself is constant.  For example with a gravitational body force applied, our pressure will increase with depth in the fluid.  Noting this provides a nice physical interpretation of the trace of the strain tensor.

Can we mathematically justify this explanation?  We see above that we have
%
\begin{equation}\label{eqn:continuumL9:650}
\spacegrad \cdot \Bu = \frac{\delta \ln(dV)}{\delta t},
\end{equation}
%
so we are in effect making the identification
%
\begin{equation}\label{eqn:continuumL9:670}
\ln dV = -p t /\lambda + \ln dV_0
\end{equation}
%
or
%
\begin{equation}\label{eqn:continuumL9:690}
dV = dV_0 e^{-p t/\lambda}.
\end{equation}
%
The relative change in a differential volume element changes exponentially.
%
\section{Dimensions of viscosity}
%
\begin{equation}\label{eqn:continuumL9:290}
[\mu] = \frac{\text{M}}{\text{L}\text{T}}.
\end{equation}
%
Some examples

\begin{itemize}
\item \(\mu_{\text{air}} = 1.8 \times 10^{-5} \frac{\text{kg}}{\text{m s}}\)
\item \(\mu_{\text{water}} = 1.1 \times 10^{-3} \frac{\text{kg}}{\text{m s}}\)
\item \(\mu_{\text{glycerin}} = 2.3 \frac{\text{kg}}{\text{m s}}\)
\end{itemize}
%
\section{Conservation of mass in fluid}
%
Referring to \cref{fig:continuumL9:continuumL9fig2}
%
\imageFigure{../figures/phy454-continuumechanics/lec9_Area_projections_for_mass_conservation_argumentFig2}{Area projections for mass conservation argument.}{fig:continuumL9:continuumL9fig2}{0.3}
%
\FIXME{What this figure illustrates was not clear to me in class}

we have a flow rate
%
\begin{equation}\label{eqn:continuumL9:310}
\rho \Bu \delta t ds
\end{equation}
%
or
\begin{equation}\label{eqn:continuumL9:330}
\rho \Bu ds,
\end{equation}
%
per unit time.  Here the velocity of fluid particle is \(\Bu\).
%
\begin{equation}\label{eqn:continuumL9:350}
\oint \rho \Bu \cdot d\Bs,
\end{equation}
%
we must have
%
\begin{equation}\label{eqn:continuumL9:710}
\PD{t}{} \int \rho dV
=
-\oint \rho \Bu \cdot d\Bs.
\end{equation}
%
\begin{equation}\label{eqn:continuumL9:370}
dm = \rho dV
\end{equation}
%
\begin{equation}\label{eqn:continuumL9:390}
\frac{dm}{dt} = \frac{d}{dt} (\rho dV)
\end{equation}
%
\begin{itemize}
\item
positive if fluid is coming in.
\item
negative if fluid is going out.
\end{itemize}

By Green's theorem
%
\begin{equation}\label{eqn:continuumL9:410}
\oint \BA \cdot d\Bs = \int_V (\spacegrad \cdot \BA) dV,
\end{equation}
%
so we have
%
\begin{equation}\label{eqn:continuumL9:430}
-\oint \rho \Bu \cdot d\Bs = -\int \spacegrad \cdot (\rho \Bu ) dV,
\end{equation}
%
and must have
%
\begin{equation}\label{eqn:continuumL9:450}
\int \left( \PD{t}{\rho} + \spacegrad \cdot (\rho \Bu) \right) dV = 0.
\end{equation}
%
The total mass has to be conserved.  The mass that is leaving the volume per unit time must move through the surface of the volume in that time.  In differential form this is \footnote{
Note that this is what I had thought we called the \textunderline{continuity equation} in physics, but I think we were using that for \(\spacegrad \cdot \Bu = 0\) instead.
}
%
\boxedEquation{eqn:continuumL9:470}{
\PD{t}{\rho} + \spacegrad \cdot (\rho \Bu) = 0.
}
%
Operating by chain rule we can write this as
%
\begin{equation}\label{eqn:continuumL9:490}
\PD{t}{\rho} + \Bu \cdot \spacegrad \rho = - \rho \spacegrad \cdot \Bu.
\end{equation}
%
To make sense of this, observe that we have for \(f = f(x, y, z, t)\)
%
\begin{equation}\label{eqn:newtonianFluidsAndMassConservationAndConstitutiveRelationAndIncompressibleFluids:770}
\begin{aligned}
\delta f
&= \lim_{\delta t \rightarrow 0} \frac{\delta f}{\delta t} dt \\
&=
\PD{x}{f} \frac{\delta x}{\delta t}
+\PD{y}{f} \frac{\delta y}{\delta t}
+\PD{z}{f} \frac{\delta z}{\delta t}
+ \PD{t}{f} \\
&=
(\spacegrad f) \cdot \Bu + \PD{t}{f}
\end{aligned}
\end{equation}
%
so we have
%
\begin{equation}\label{eqn:continuumL9:510}
\PD{t}{\rho} + \Bu \cdot \spacegrad \rho = \frac{d\rho}{dt}
\end{equation}
%
or
\begin{equation}\label{eqn:continuumL9:530}
\frac{d\rho}{dt} = - \rho \spacegrad \cdot \Bu.
\end{equation}
%
\section{Incompressible fluid}
%
When the density does not change note that we have
%
\begin{equation}\label{eqn:continuumL9:550}
\frac{d\rho}{dt} = 0
\end{equation}
%
which then implies
%
\boxedEquation{eqn:continuumL9:570}{
\spacegrad \cdot \Bu = 0,
}
%
at all points in the fluid.
