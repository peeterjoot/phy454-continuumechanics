%
% Copyright � 2012 Peeter Joot.  All Rights Reserved.
% Licenced as described in the file LICENSE under the root directory of this GIT repository.
%

%
%
\makeoproblem{Steady rectilinear blood flow}{problem:fluids:blood}
{2012 ps2}
{
Imagine a steady rectilinear blood flow of the form \(\Bu = u(y) \ycap\) through a two dimensional artery.  It is driven by a constant pressure gradient \(G = -dp/dx\) maintained by an external `heart'.  The top and bottom walls of the artery are \(2h\) distance apart and the fluid satisfies no-slip boundary conditions at the walls.  Assuming that the fluid is Newtonian,


\makesubproblem{Show that the Navier-Stokes equation reduces to
%
\begin{equation}\label{eqn:continuumProblemSet2:20}
\frac{d^2 u}{dy^2} = -\frac{G}{\mu}
\end{equation}
%
where \(\mu\) is the viscosity of the blood.

}{problem:fluids:bloodReduceNS}

\makesubproblem{Show that the velocity profile of the fluid inside the artery is a parabolic profile}{problem:fluids:bloodVelocityProfile}
\makesubproblem{What is the maximum speed of the fluid?  Draw the velocity profile to show where the maximum speed occurs inside the artery}{problem:fluids:bloodMaxSpeed}
\makesubproblem{If due to smoking etc., the viscosity of the blood gets doubled, then what should be the new pressure gradient to be maintained by the `heart' to keep the liquid flux through the artery at the same level as the non-smoking one?}{problem:fluids:bloodViscositySmoking}
} % makeoproblem

\makeanswer{problem:fluids:blood}{

\makesubanswer{Navier-Stokes}{problem:fluids:bloodReduceNS}
%\unnumberedSubsection{Navier-Stokes equation for the system}
The Navier-Stokes equation, for an incompressible unidirectional fluid \(\Bu = (u, 0, 0)\), assuming that there is no \(z\) dependence, takes the form
%
\begin{subequations}
\begin{equation}\label{eqn:continuumProblemSet2:40}
\rho \PD{t}{u} + u \PD{x}{u} = - \PD{x}{p} + \mu \left( \PDSq{x}{} + \PDSq{y}{} \right) u
\end{equation}
\begin{equation}\label{eqn:continuumProblemSet2:60}
0 = - \PD{y}{p}
\end{equation}
\begin{equation}\label{eqn:continuumProblemSet2:80}
0 = - \PD{z}{p}
\end{equation}
\begin{equation}\label{eqn:continuumProblemSet2:100}
0 = \PD{x}{u}.
\end{equation}
\end{subequations}
%
With a steady state assumption we kill the \(\PDi{t}{u}\) term, and \eqnref{eqn:continuumProblemSet2:100} kills of the x-component of the Laplacian and our non-linear inertial term on the LHS, leaving just
%
\begin{subequations}
\begin{equation}\label{eqn:continuumProblemSet2:40a}
0 = - \PD{x}{p} + \mu \PDSq{y}{u}
\end{equation}
\begin{equation}\label{eqn:continuumProblemSet2:60b}
0 = - \PD{y}{p}
\end{equation}
\begin{equation}\label{eqn:continuumProblemSet2:80c}
0 = - \PD{z}{p}.
\end{equation}
\end{subequations}
%
With \(\PD{z}{p} = \PD{y}{p} = 0\), we have \(\PD{x}{p} = dp/dx = -G\), so \eqnref{eqn:continuumProblemSet2:40a} is reduced to
%
\begin{equation}\label{eqn:continuumProblemSet2:40b}
0 = G + \mu \PDSq{y}{u}.
\end{equation}
%
Finally, since we have an assumption of no z-dependence (\(\PDi{z}{u} = 0\)) and from the incompressibility assumption \eqnref{eqn:continuumProblemSet2:100} (\(\PDi{x}{u} = 0\)), we have
%
\begin{equation}\label{eqn:continuumProblemSet2:140}
\PDSq{y}{u} = \frac{d^2 u}{dy^2} = -\frac{G}{\mu},
\end{equation}
%
as desired.

\makesubanswer{Velocity profile}{problem:fluids:bloodVelocityProfile}
For the velocity profile, integrating \eqnref{eqn:continuumProblemSet2:140} twice, we have
%
\begin{equation}\label{eqn:continuumProblemSet2:160}
u = -\frac{G}{2 \mu} y^2 + A y + B.
\end{equation}
%
Application of the no-slip boundary value condition \(u(\pm h) = 0\), we have
%
\begin{equation}\label{eqn:continuumProblemSet2:180}
\begin{aligned}
0 &= -\frac{G}{2 \mu} h^2 + A h + B \\
0 &= -\frac{G}{2 \mu} h^2 - A h + B
\end{aligned}
\end{equation}
%
Adding and subtracting these, we find
%
\begin{subequations}
\begin{equation}\label{eqn:continuumProblemSet2:280}
A = 0
\end{equation}
\begin{equation}\label{eqn:continuumProblemSet2:300}
B = \frac{G}{2 \mu} h^2,
\end{equation}
\end{subequations}
%
so the velocity is given by the parabolic function
%
\begin{equation}\label{eqn:continuumProblemSet2:220}
u(y) = \frac{G}{2 \mu} \left( h^2 - y^2 \right).
\end{equation}
%
\makesubanswer{Maximum speed}{problem:fluids:bloodMaxSpeed}
It is clear that the maximum speed of the fluid is found at \(y = 0\)
%
\begin{equation}\label{eqn:continuumProblemSet2:240}
u(0) = \frac{G h^2}{2 \mu}
\end{equation}
%
The velocity profile for this flow is drawn in \cref{fig:continuumProblemSet2:continuumProblemSet2Fig1}.

%\begin{figure}[htp]
%   \centering
%   \includegraphics[totalheight=0.3\textheight]{continuumProblemSet2Fig1}
%   \caption{Velocity profile for 1D constant pressure gradient steady state flow}\label{fig:continuumProblemSet2:continuumProblemSet2Fig1}
%\end{figure}
%
\pdfTexFigure{../figures/phy454-continuumechanics/continuumProblemSet2Fig1r2.pdf_tex}{Velocity profile for 1D constant pressure gradient steady state flow.}{fig:continuumProblemSet2:continuumProblemSet2Fig1}{0.2}
%
\makesubanswer{Effects of viscosity doubling}{problem:fluids:bloodViscositySmoking}
With our velocity being dependent on the \(G/\mu\) ratio, it is clear that to consider the effects of viscosity doubling, even without calculating the flux, that we will need  twice the pressure gradient if the viscosity is doubled to maintain the same flux through the artery and veins.  To demonstrate this more thoroughly we can calculate this mass flux.  For an element of mass leaving a portion of the conduit, bounded by the plane normal to \(\xcap\) we have
%
\begin{equation}\label{eqn:continuumProblemSet2:980}
\begin{aligned}
\frac{dm}{dt}
&= \rho \frac{dV}{dt} \\
&= \rho dz dy \Bu \cdot \xcap
\end{aligned}
\end{equation}
%
Integrating this over a width \(\Delta z\), our flux through the plane is
%
\begin{equation}\label{eqn:continuumProblemSet2:1000}
\begin{aligned}
\text{Flux}
&=
\int_0^{\Delta z} dz
\int_{-h}^h dy \frac{G}{2 \mu} \left( h^2 - y^2 \right) \\
&=
\Delta z
\frac{G}{2 \mu}
\evalrange{ \left( h^2 y - \inv{3} y^3 \right) }{-h}{h} \\
&=
\Delta z
\frac{G h^3}{\mu} \left( 1 - \inv{3} \right)  \\
&=
\Delta z \frac{2 G h^3}{3 \mu}.
\end{aligned}
\end{equation}
%
Doubling the blood viscosity for our smoker, our respective fluxes are
%
\begin{equation}\label{eqn:continuumProblemSet2:320}
\begin{aligned}
\text{Flux}_{\text{smoker}} &= \Delta z \frac{2 G_{\text{smoker}} h^3}{3 (2 \mu)}  \\
\text{Flux}_{\text{non-smoker}} &= \Delta z \frac{2 G h^3}{3 \mu}.
\end{aligned}
\end{equation}
%
Demanding equality before and after smoking we find
%
\begin{equation}\label{eqn:continuumProblemSet2:260}
G_{\text{smoker}} = 2 G.
\end{equation}
%
where \(G\) is the magnitude of the pressure gradient before the bad habits kicked in.  The smoker's poor little heart (soon to be a big overworked and weak heart) has to generate pressure gradients that are twice as big to get the same quantity of blood distributed through the body.
} % end answer

\makeoproblem{Simple shearing flow}{problem:fluids:simpleShearing}
{2012 ps2}
{
Consider steady simple shearing flow with no imposed pressure gradient \((G = 0)\) of a two layer fluid with viscosity
%
\begin{equation}\label{eqn:continuumProblemSet2:120}
\mu =
\left\{
\begin{array}{l l}
\mu^{(1)} & \quad \mbox{\(-h < y < 0,\)} \\
\mu^{(2)} & \quad \mbox{\(0 < y < h.\)}
\end{array}
\right.
\end{equation}
%
The boundary conditions are no-slip at the lower plate \((y = -h)\) and at \(y = 0\).  The top plate is moving with a velocity \(-U\) at \(y = h\) and fluid is sticking to it.  using the continuity of tangential (shear) stress at the interface (\(y = 0\))

\makesubproblem{Derive the velocity profile of the two fluids}{problem:fluids:simpleShearing1}
\makesubproblem{Calculate the maximum speed}{problem:fluids:simpleShearing2}
\makesubproblem{Calculate the mean speed}{problem:fluids:simpleShearing3}
\makesubproblem{Calculate the flux (the volume flow rate.)}{problem:fluids:simpleShearing4}
\makesubproblem{Calculate the tangential force (per unit width) \(F_x\) on the strip \(0 \le x \le L\) of the wall \(y = -h\)}{problem:fluids:simpleShearing5}
\makesubproblem{Calculate the tangential force (per unit width) \(F_x^0\) on the strip \(0 \le x \le L\) at the interface \(y = 0\) by the top fluid on the lower fluid}{problem:fluids:simpleShearing6}
} % makeoproblem

\makeanswer{problem:fluids:simpleShearing}{

\makesubanswer{Velocity profiles}{problem:fluids:simpleShearing1}
Starting with the velocity profile derivation for the two fluids, we set up coordinates as in \cref{fig:continuumProblemSet2:continuumProblemSet2Fig2}.  Our steady flow for layers \(1\) and \(2\) has the form

%\begin{figure}[htp]
%   \centering
%   \includegraphics[totalheight=0.3\textheight]{continuumProblemSet2Fig2}
%   \caption{Two layer flow induced by moving wall}\label{fig:continuumProblemSet2:continuumProblemSet2Fig2}
%\end{figure}
%
\pdfTexFigure{../figures/phy454-continuumechanics/continuumProblemSet2Fig2r2.pdf_tex}{Two layer flow induced by moving wall.}{fig:continuumProblemSet2:continuumProblemSet2Fig2}{0.4}
%
\begin{subequations}
\begin{equation}\label{eqn:continuumProblemSet2:340}
0 = - \PD{x}{p} + \mu^{(i)} \PDSq{y}{u^{(i)}}
\end{equation}
\begin{equation}\label{eqn:continuumProblemSet2:360}
0 = - \PD{y}{p}
\end{equation}
\begin{equation}\label{eqn:continuumProblemSet2:380}
0 = - \PD{z}{p},
\end{equation}
\end{subequations}
%
as we found in Q1.  Only the boundary value conditions and the driving pressure are different here.  In this problem and the next, we have constant pressure gradients \(dp/dx = -G\) to deal with, so we really have just the pair of equations
%
\begin{equation}\label{eqn:continuumProblemSet2:400}
0 = G + \mu^{(i)} \frac{d^2 u^{(i)}(y)}{dy^2},
\end{equation}
%
to solve.  For this Q2 problem we have \(G = 0\), so the algebra to match our boundary value constraints becomes a bit easier.  Our boundary value constraints are
%
\begin{subequations}
\begin{equation}\label{eqn:continuumProblemSet2:420}
u^{(1)}(-h) = 0 \\
\end{equation}
\begin{equation}\label{eqn:continuumProblemSet2:440}
u^{(2)}(h) = -U \\
\end{equation}
\begin{equation}\label{eqn:continuumProblemSet2:460}
u^{(1)}(0) = u^{(2)}(0),
\end{equation}
\end{subequations}
%
plus one more to match the tangential components of the traction vector with respect to the normal \(\ncap = (0, 1, 0)\).  The components of that traction vector are
%
\begin{equation}\label{eqn:continuumProblemSet2:1020}
\begin{aligned}
t_i
&= \left( -p \delta_{ij} + 2 \mu e_{ij} \right) n_j \\
&= \left( -p \delta_{ij} + 2 \mu e_{ij} \right) \delta_{2j} \\
&= -p \delta_{i2} + 2 \mu e_{i2},
\end{aligned}
\end{equation}
%
but we are only interested in the horizontal component \(t_1\) which is
%
\begin{equation}\label{eqn:continuumProblemSet2:1040}
\begin{aligned}
t_1
&=
 -p \delta_{12} + 2 \mu e_{12} \\
&=
2 \mu \inv{2} \left(
\PD{u}{y}
+
\cancel{\PD{v}{x}}
\right).
\end{aligned}
\end{equation}
%
So the matching the tangential components of the traction vector at the interface gives us our last boundary value constraint
%
\begin{equation}\label{eqn:continuumProblemSet2:480}
\evalbar{\mu^{(1)} \PD{u}{y}}{y = 0} = \evalbar{\mu^{(2)} \PD{u}{y}}{y = 0},
\end{equation}
%
and we are ready to do our remaining bits of algebra.  We wish to solve the pair of equations
%
\begin{equation}\label{eqn:continuumProblemSet2:500}
\begin{aligned}
u^{(1)} &= A^{(1)} y + B^{(1)} \\
u^{(2)} &= A^{(2)} y + B^{(2)},
\end{aligned}
\end{equation}
%
for the four integration constants \(A^{(i)}\) and \(B^{(i)}\) using our boundary value constraints.  The linear system to solve is
%
\begin{equation}\label{eqn:continuumProblemSet2:520}
\begin{aligned}
0 &= -A^{(1)} h + B^{(1)} \\
-U &= A^{(2)} h + B^{(2)} \\
B^{(1)} &= B^{(2)} \\
\mu^{(1)} A^{(1)} &= \mu^{(2)} A^{(2)}.
\end{aligned}
\end{equation}
%
With \(B = B^{(i)}\), we have
%
\begin{equation}\label{eqn:continuumProblemSet2:540}
\begin{aligned}
0 &= -A^{(1)} h + B \\
-U &= \frac{\mu^{(1)}}{\mu^{(2)}} A^{(1)} h + B
\end{aligned}
\end{equation}
%
Subtracting these to solve for \(A^{(1)}\) we find
%
\begin{equation}\label{eqn:continuumProblemSet2:560}
-U = h A^{(1)} \left( \frac{\mu^{(1)}}{\mu^{(2)}} + 1 \right).
\end{equation}
%
This gives us everything we need
%
\begin{equation}\label{eqn:continuumProblemSet2:580}
\begin{aligned}
A^{(1)} &= -\frac{U \mu^{(2)}}{h(\mu^{(1)} + \mu^{(2)})} \\
A^{(2)} &= -\frac{U \mu^{(1)}}{h(\mu^{(1)} + \mu^{(2)})} \\
B^{(1)} = B^{(2)} &= -\frac{U \mu^{(2)}}{\mu^{(1)} + \mu^{(2)}},
\end{aligned}
\end{equation}
%
Referring back to \eqnref{eqn:continuumProblemSet2:500} our velocities are
\boxedEquation{eqn:continuumProblemSet2:600}{
\begin{aligned}
%u^{(1)} &=
%-\frac{U }{h(\mu^{(1)} + \mu^{(2)})} \mu^{(2)} \left( y + h \right) \\
%u^{(2)} &=
%-\frac{U }{h(\mu^{(1)} + \mu^{(2)})} \left( \mu^{(1)} y + \mu^{(2)} h \right)
u^{(1)} &=
-\frac{U \mu^{(2)} }{(\mu^{(1)} + \mu^{(2)})} \left( 1 + \frac{y}{h} \right) \\
u^{(2)} &=
-\frac{U \mu^{(2)} }{(\mu^{(1)} + \mu^{(2)})} \left( 1 + \frac{\mu^{(1)}}{\mu^{(2)}} \frac{y}{h} \right)
\end{aligned}
}
%-\frac{U \mu^{(1)}}{h(\mu^{(1)} + \mu^{(2)})} y
%-\frac{U \mu^{(2)}}{h(\mu^{(1)} + \mu^{(2)})} h
Checking, we see at a glance we see that we have \(u^{(2)}(h) = -U\), \(u^{(1)}(-h) = 0\), \(u^{(1)}(0) = u^{(2)}(0)\), and \(\evalbar{\mu^{(1)} du^{(1)}/dy}{y=0} = \mu^{(2)} \evalbar{du^{(2)}/dy}{y=0}\) as desired.

As an example, let us add some numbers.  With mercury and water in layers \({(1)}\) and \({(2)}\) respectively, we have
%
\begin{equation}\label{eqn:continuumL16:620}
\begin{aligned}
\mu^{(1)} &= 0.001526 \quad \text{Pa-s} \\
\mu^{(2)} &= 0.00089 \quad \text{Pa-s}
\end{aligned}
\end{equation}
%
so that our velocity is
%
\begin{equation}\label{eqn:continuumProblemSet2:640}
u(y) =
\left\{
\begin{array}{l l}
-1.37 U \left(1 + \frac{y}{h}\right) & \quad \mbox{\(y \in [-h, 0]\)} \\
-1.37 U \left(1 + 1.7 \frac{y}{h}\right) & \quad \mbox{\(y \in [0, h]\)}
\end{array}
\right.
\end{equation}
%
This is plotted with \(h = U = 1\) in \cref{fig:continuumProblemSet2:continuumProblemSet2Fig3}
\imageFigure{../figures/phy454-continuumechanics/continuumProblemSet2Fig3}{Two layer shearing flow with water over mercury.}{fig:continuumProblemSet2:continuumProblemSet2Fig3}{0.2}
%
\makesubanswer{Maximum speed}{problem:fluids:simpleShearing2}
We are now ready to calculate the maximum speed.

With \(u^{(1)}(-h) = 0\), and \(u^{(1)}\) linearly decreasing, then \(u^{(2)}\) linearly decreasing further from the value at \(y = 0\), it is clear that the maximum speed, no matter the viscosities of the fluids, is on the upper moving interface.  This maximum takes the value \(\Abs{u^{(2)}(h)} = U\).

\makesubanswer{Mean speed}{problem:fluids:simpleShearing3}
As linear functions the average speeds of the respective fluids fall on the midpoints \(y = \pm h/2\).  These are
%
\begin{equation}\label{eqn:continuumProblemSet2:660}
\begin{aligned}
\expectation{u^{(1)}} &= -\frac{U \mu^{(2)} }{ 2 (\mu^{(1)} + \mu^{(2)})} \\
\expectation{u^{(2)}} &= -\frac{U \mu^{(2)} }{(\mu^{(1)} + \mu^{(2)})} \left( 1 + \frac{\mu^{(1)}}{2 \mu^{(2)}} \right)
\end{aligned}
\end{equation}
%
Averaging these two gives us the overall average, so we find
%
\begin{equation}\label{eqn:continuumProblemSet2:680}
\expectation{u(y)} =
-\frac{U }{ 4 (\mu^{(1)} + \mu^{(2)})} \left( 3 \mu^{(2)} + \mu^{(1)} \right)
\end{equation}
%
\makesubanswer{Volume flux}{problem:fluids:simpleShearing4}
We can calculate the volume flux, much like the mass flux (although the mass flux seems a more sensible quantity to calculate).  Looking at the rate of change of an element of fluid passing through the \(y-z\) plane we have
%
\begin{equation}\label{eqn:continuumProblemSet2:700}
\frac{dV}{dt} = dy dz \Bu \cdot \xcap
\end{equation}
%
Integrating over the total height, for a width \(\Delta z\) we have
%
\begin{equation}\label{eqn:continuumProblemSet2:1060}
\begin{aligned}
\text{Volume Flux}
&= \Delta z \int_{-h}^h u(y) dy \\
&= \Delta z 2h \expectation{u} \\
\end{aligned}
\end{equation}
%
So our volume flux through a width \(\Delta z\) is
%
\begin{equation}\label{eqn:continuumProblemSet2:720}
\text{Volume Flux}
=
-\frac{2 h U \Delta z}{ 4 (\mu^{(1)} + \mu^{(2)})} \left( 3 \mu^{(2)} + \mu^{(1)} \right).
\end{equation}
%
\makesubanswer{Tangential force on lower wall}{problem:fluids:simpleShearing5}
We see from \eqnref{eqn:continuumProblemSet2:600} the tangential components of our traction vectors are
%
\begin{equation}\label{eqn:continuumProblemSet2:1080}
\begin{aligned}
t^{(1)}
&= \mu^{(1)} \frac{d u^{(1)}}{dy} \\
&= -\frac{U \mu^{(1)} \mu^{(2)} }{(\mu^{(1)} + \mu^{(2)})} \inv{h}
\end{aligned}
\end{equation}
%
and
%
\begin{equation}\label{eqn:continuumProblemSet2:1100}
\begin{aligned}
t^{(2)}
&= \mu^{(2)} \frac{d u^{(2)}}{dy} \\
&= -\frac{U \mu^{(1)} \mu^{(2)} }{(\mu^{(1)} + \mu^{(2)})} \inv{h}
\end{aligned}
\end{equation}
%
We see that the tangential component of the traction vector is a constant throughout both fluids.  Allowing this force to act on a length \(L\) of the lower wall, our force per unit width over that strip is just
%
\begin{equation}\label{eqn:continuumProblemSet2:740}
F = -\frac{U \mu^{(1)} \mu^{(2)} }{(\mu^{(1)} + \mu^{(2)})} \frac{L}{h}.
\end{equation}
%
The negative value here makes sense since it is acting to push the fluid backwards in the direction of the upper wall motion.

\makesubanswer{Tangential force on upper wall}{problem:fluids:simpleShearing6}
We note that due to constant nature of the tangential component of the traction vector shown above, the force per unit width of the upper fluid acting on the lower fluid, is also given by \eqnref{eqn:continuumProblemSet2:740}.
} % end answer

\makeoproblem{Shearing flow, w/ pressure gradient}{problem:fluids:simpleShearingPressureGrad}
{2012 ps2}
{
Add a constant pressure gradient \(G = -dp/dx\), applied between the boundaries \(y = \pm h\), to the problem above.
Describe qualitatively what type of flow profile you would expect in the steady state.  Draw the velocity profiles for two cases (i) \(\mu^{(1)} > \mu^{(2)}\) (ii) \(\mu^{(1)} < \mu^{(2)}\).  Explain your result.
} % makeoproblem

\makeanswer{problem:fluids:simpleShearingPressureGrad}{
We showed earlier that the Navier-Stokes equations for this Q3 case, where \(G\) is non-zero were given by \eqnref{eqn:continuumProblemSet2:400}, which restated is
\begin{equation}\label{eqn:continuumProblemSet2:400b}
0 = G + \mu^{(i)} \frac{d^2 u^{(i)}(y)}{dy^2}.
\end{equation}
%
Our solutions will now necessarily be parabolic, of the form
\begin{equation}\label{eqn:continuumProblemSet2:760}
u^{(i)}(y) = -\frac{G}{2 \mu^{(i)}} y^2 + A^{(i)} y + B^{(i)},
\end{equation}
with the tangential traction vector components given by
\begin{equation}\label{eqn:continuumProblemSet2:780}
t^{(i)} = -G y + A^{(i)} \mu^{(i)}
\end{equation}
%
The boundary value constants become a bit messier to solve for, and should we wish to do so we would have to solve the system
\begin{equation}\label{eqn:continuumProblemSet2:800}
\begin{aligned}
0 &= -\frac{G}{2 \mu^{(1)}} h^2 - A^{(1)} h + B^{(1)} \\
-U &= -\frac{G}{2 \mu^{(2)}} h^2 + A^{(2)} h + B^{(2)} \\
B^{(1)} &= B^{(2)} \\
A^{(1)} \mu^{(1)} &= A^{(2)} \mu^{(2)}
\end{aligned}
\end{equation}
%
Without actually solving this system we should expect that our solution will have the form of our pure shear flow, with parabolas superimposed on these linear flows.  For a higher viscosity bottom layer \(\mu^{(1)} > \mu^{(2)}\), this should look something like \cref{fig:continuumProblemSet2:continuumProblemSet2Fig4} whereas for the higher viscosity on the top, these would be roughly flipped as in \cref{fig:continuumProblemSet2:continuumProblemSet2Fig5}.
%
\imageFigure{../figures/phy454-continuumechanics/continuumProblemSet2Fig4}{Superposition of constant pressure gradient and shear flow solutions (\(\mu^{(1)} > \mu^{(2)}\)).}{fig:continuumProblemSet2:continuumProblemSet2Fig4}{0.3}
%
\imageFigure{../figures/phy454-continuumechanics/continuumProblemSet2Fig5}{Superposition of constant pressure gradient and shear flow solutions (\(\mu^{(1)} < \mu^{(2)}\)).}{fig:continuumProblemSet2:continuumProblemSet2Fig5}{0.3}
%
This superposition can be justified since we have no \((\Bu \cdot \spacegrad)\Bu\) term in the Navier-Stokes equations for these systems.

%\unnumberedSubsection{Exact solutions}

The figures above are kind of rough.  It is not actually hard to solve the system above.  After some simplification, I find using Mathematica in (\nbref{problemSetIIQ3exactSolution.cdf}) the following solution
%
\begin{equation}\label{eqn:continuumProblemSet2:840}
\begin{aligned}
u^{(1)}(y) &= -\frac{\mu^{(2)} U -G h^2}{\mu^{(1)}+\mu^{(2)}}-\frac{y \left(G h^2 (\mu^{(2)} -\mu^{(1)}) +2 \mu^{(1)} \mu^{(2)} U \right)}{2 h \mu^{(1)} \left(\mu^{(1)}+\mu^{(2)}\right)}-\frac{G y^2}{2 \mu^{(1)}} \\
u^{(2)}(y) &= -\frac{\mu^{(2)} U -G h^2}{\mu^{(1)}+\mu^{(2)}}-\frac{y \left(G h^2 (\mu^{(2)} -\mu^{(1)}) +2 \mu^{(1)} \mu^{(2)} U \right)}{2 h \mu^{(2)} \left(\mu^{(1)}+\mu^{(2)}\right)}-\frac{G y^2}{2 \mu^{(2)}}.
\end{aligned}
\end{equation}
%
Should we wish a more exact plot for any specific values of the viscosities, we could plot exactly with software the vector field described by these velocities.

I suppose it is cheating to use Mathematica and then say that the solution is easy?  To make amends for being lazy with my algebra, let us show that it is easy to do manually too.  I will do the same problem manually, but generalize it slightly.  We can do this easily if we just be a bit smarter with our integration constants.  Let us solve the problem for the upper and lower walls moving with velocity \(V_2\) and \(V_1\) respectively, and let the heights from the interface be \(h_2\) and \(h_1\) respectively.

We have the same set of differential equations to solve, but now let us write our solution with the undetermined coefficients expressed as
%
\begin{equation}\label{eqn:continuumProblemSet2:860}
\begin{aligned}
u^{(2)} &= -\frac{G}{2 \mu^{(2)}}\left( h_2^2 - y^2 \right) + \frac{A_2}{\mu^{(2)}} (y - h_2) + B_2 \\
u^{(1)} &= -\frac{G}{2 \mu^{(1)}}\left( h_1^2 - y^2 \right) + \frac{A_1}{\mu^{(1)}} (y + h_1) + B_1.
\end{aligned}
\end{equation}
%
Now it is super easy to match the boundary conditions at \(y = -h_1\) and \(y = h_2\)(the lower and upper walls respectively).  Clearly the integration constants \(B_1,B_2\) are just the velocities.  Matching the tangential component of the traction vectors at \(y = 0\) we have
%
\begin{equation}\label{eqn:continuumProblemSet2:880}
A_2 = A_1
\end{equation}
%
and matching velocities at \(y = 0\) gives us
%
\begin{equation}\label{eqn:continuumProblemSet2:900}
-\frac{G}{2 \mu^{(2)}}h_2^2 - \frac{A_2}{\mu^{(2)}} h_2 + V_2 = -\frac{G}{2 \mu^{(1)}}h_1^2 + \frac{A_1}{\mu^{(1)}} h_1 + V_1.
\end{equation}
%
This gives us
\begin{equation}\label{eqn:continuumProblemSet2:920}
\begin{aligned}
u^{(2)} &= \frac{G h_2^2}{2 \mu^{(2)}}\left(\frac{y^2}{h_2^2} - 1 \right) + A \frac{ h_2}{\mu^{(2)}} \left( \frac{y}{h_2} - 1 \right) + V_2 \\
u^{(1)} &= \frac{G h_1^2}{2 \mu^{(1)}}\left(\frac{y^2}{h_1^2} - 1 \right) + A \frac{ h_1}{\mu^{(1)}} \left( \frac{y}{h_1} + 1 \right) + V_1 \\
A
&=
\frac{
V_2 - V_1
+
\frac{G h_1^2}{2 \mu^{(1)}}
-\frac{G h_2^2}{2 \mu^{(2)}}
}{
\frac{h_1}{\mu^{(1)}}
+\frac{h_2}{\mu^{(2)}}
}.
\end{aligned}
\end{equation}
%
Plotting this with sliders or animation in Mathematica (\nbref{problemSetIIQ3PlotWithManipulate.cdf}) is a fun way to explore visualizing this.  The results vary widely depending on the various parameters.  Here are animations with variation of the pressure gradient for \(v_1 = 0\), \(h_1 = h_2\), showing the superposition of the shear and channel flow solutions

\begin{itemize}
\item With \(\mu^{(1)} > \mu^{(2)}\).  See \youtubehref{2xVoFAL9XGA}.
%\cref{fig:continuumProblemSet2:continuumProblemSet2Animation1} (or

\item With \(\mu^{(2)} > \mu^{(1)}\).  See \youtubehref{FJekyGf6XJw}.
%\cref{fig:continuumProblemSet2:continuumProblemSet2Animation2} (or
\end{itemize}

%\movieFigure{./animatedTwoLayerFlowMu1GreaterThanMu2Try3.mp4}{Fluids with densities of mercury and water in the lower (1) and upper (2) layers respectively}{fig:continuumProblemSet2:continuumProblemSet2Animation1}{width=320pt,height=240pt}

%\movieFigure{./animatedTwoLayerFlowMu2GreaterThanMu1Try3.mp4}{Fluids with densities of water and mercury in the lower (1) and upper (2) layers respectively}{fig:continuumProblemSet2:continuumProblemSet2Animation2}{width=320pt,height=240pt}

\FIXME{re-draw the figures and adjust above in response to these points}
I lost a couple of marks on this assignment, all on the hand plotting.  The remarks were

\begin{enumerate}
\item Gradients are the wrong way around.
\item \(G\) is constant across the fluid.
\end{enumerate}

I am assuming that the comment about \(G\) being constant across the fluid means that the two humped velocity distribution I drew is not realistic.  Here is two actual plots using the above calculations.

%\cref{fig:continuumProblemSet2:continuumProblemSet2Fig6mu1Gtmu2}
\imageFigure{../figures/phy454-continuumechanics/continuumProblemSet2Fig6mu1Gtmu2}{\(\mu^{(1)} > \mu^{(2)}\).}{fig:continuumProblemSet2:continuumProblemSet2Fig6mu1Gtmu2}{0.3}
%
%\cref{fig:continuumProblemSet2:continuumProblemSet2Fig6mu2Gtmu1}
\imageFigure{../figures/phy454-continuumechanics/continuumProblemSet2Fig6mu2Gtmu1}{\(\mu^{(2)} > \mu^{(1)}\).}{fig:continuumProblemSet2:continuumProblemSet2Fig6mu2Gtmu1}{0.3}
%
It was not clear to me initially what the grader meant by the gradients were the wrong way around, but I see that too looking at the actual plots.  If you check out the Mathematica worksheet itself you will see that I do not have a pressure gradient slider, but a velocity \(V_{\text{pressure}}\) slider.  This was based on the fact that for a channel flow our average speed is proportional to the pressure gradient
%
\begin{equation}\label{eqn:continuumProblemSet2:940}
\expectation{u} = \frac{G h}{3 \mu},
\end{equation}
%
so in order to parameterize the pressure gradient in an intuitive sense I defined it as a weighted average
%
\begin{equation}\label{eqn:continuumProblemSet2:960}
G = 3 V_{\text{pressure}} \inv{2} \left( \frac{\mu^{(1)}}{h_1} + \frac{\mu^{(2)}}{h_2}\right).
\end{equation}
%
Sure enough when I set \(V_{\text{pressure}} > 0\) in the Mathematica slider (so that the pressure gradient is also positive) I get the channel flows pointing in the opposite direction as indicated in the grading comment.  I should have sketched this channel flow more carefully in the very simplest case first before doing the two layer flow.
} % end answer
