%
% Copyright � 2012 Peeter Joot.  All Rights Reserved.
% Licenced as described in the file LICENSE under the root directory of this GIT repository.
%

%
%
\makeoproblem{Non-Newtonian fluid.}{problem:fluids:midterm:1c}
{2012 midterm, p1 c}
{
What is the definition of a Non-Newtonian fluid?
} % makeoproblem
%
\makeanswer{problem:fluids:midterm:1c}{
A non-Newtonian fluid would be one with a more general constitutive relationship.

A Newtonian fluid \citep{wiki:newtonianFluids} is one with a linear stress strain relationship, and a non-Newtonian fluid would be one with a non-linear relationship.  An example of a non-Newtonian material that we are all familiar with is Silly Putty.
\FIXME{This linearity is also how a Newtonian fluid was defined in the notes, but I did not remember that and lost a mark here on the midterm (this is not really something we use since we assume all fluids and materials are Newtonian in any calculations that we do)}

\FIXME{
In class midterm discussion:

We had been asked to solve a 1D flow problem with one fixed surface, and one moving surface, as illustrated in \cref{fig:continuumL13:continuumL13Fig1}
\imageFigure{../figures/phy454-continuumechanics/lec13_Return_flow_configuration__One_fixed_surface_one_moving_surfaceFig1}{Return flow configuration.  One fixed surface, one moving surface.}{fig:continuumL13:continuumL13Fig1}{0.2}
%
we end up showing that our solutions are of the forms

\begin{itemize}
\item
Shear flow \cref{fig:continuumL13:continuumL13Fig2a}.  \(U \ne 0, G = 0\).
\imageFigure{../figures/phy454-continuumechanics/lec13_Shear_FlowFig2a}{Shear Flow.}{fig:continuumL13:continuumL13Fig2a}{0.2}
\item Channel flow \cref{fig:continuumL13:continuumL13Fig2b}.  \(U = 0, G \ne 0\).
\imageFigure{../figures/phy454-continuumechanics/lec13_Channel_flowFig2b}{Channel flow.}{fig:continuumL13:continuumL13Fig2b}{0.2}
\item Return flow \cref{fig:continuumL13:continuumL13Fig2c}.  \(U \ne 0, G \ne 0\).
\imageFigure{../figures/phy454-continuumechanics/lec13_Return_flowFig2c}{Return flow.}{fig:continuumL13:continuumL13Fig2c}{0.2}
\end{itemize}

Note that the last sort of solution, that of return flow, was discussed in the context of surface tension.
}
} % end answer
%
\makeoproblem{No-slip boundary condition.}{problem:fluids:midterm:1d}
{2012 midterm, p1 d}
{
What do you mean by \emph{no-slip} boundary condition at a fluid-fluid interface?
} % makeoproblem
%
\makeanswer{problem:fluids:midterm:1d}{
%\unnumberedSubsection{Exam time management note} Somehow in my misguided attempt to be complete, I missed this question amongst the rest of my verbosity).

%\unnumberedSubsection{Answer}
The no slip boundary condition is just one of velocity matching.  At a non-moving boundary, the no-slip condition means that we will require the fluid to also have no velocity (ie. at that interface the fluid is not slipping over the surface).  Between two fluids, this is a requirement that the velocities of both fluids match at that point (and all the rest of the points along the region of the interaction.)
} % end answer
%
\makeoproblem{Continuity equation.}{problem:fluids:midterm:1e}
{2012 midterm, p1 e}
{
Write down the continuity equation for an incompressible fluid.
} % makeoproblem
%
\makeanswer{problem:fluids:midterm:1e}{
An incompressible fluid has
%
\begin{equation}\label{eqn:continuumMidTermReflection:210}
\frac{d\rho}{dt} = 0,
\end{equation}
%
but since we also have
%
\begin{equation}\label{eqn:continuumMidtermReflection:850}
\begin{aligned}
0
&=
\frac{d\rho}{dt} \\
&= - \rho (\spacegrad \cdot \Bu)  \\
&=
\PD{t}{\rho} + (\Bu \cdot \spacegrad) \rho \\
&= 0.
\end{aligned}
\end{equation}
%
A consequence is that \(\spacegrad \cdot \Bu = 0\) for an incompressible fluid.  Let us recall where this statement comes from.  Looking at mass conservation, the rate that mass leaves a volume can be expressed as
%
\begin{equation}\label{eqn:continuumMidtermReflection:870}
\begin{aligned}
\frac{dm}{dt}
&= \int \frac{d\rho}{dt} dV \\
&= -\int_{\partial V} \rho \Bu \cdot d\BA \\
&= -\int_V \spacegrad \cdot (\rho \Bu) dV.
\end{aligned}
\end{equation}
%
The minus sign here signifying that the mass is leaving the volume through the surface, and that we are using an outwards facing normal on the volume.
If the surface bounding the volume does not change with time (ie. \(\PDi{t}{V} = 0\)) we can write
%
\begin{equation}\label{eqn:continuumMidTermReflection:230}
\PD{t}{} \int \rho dV = -\int \spacegrad \cdot (\rho \Bu) dV,
\end{equation}
%
or
%
\begin{equation}\label{eqn:continuumMidTermReflection:250}
0 = \int \left( \PD{t}{\rho} + \spacegrad \cdot (\rho \Bu) \right) dV,
\end{equation}
%
so that in differential form we have
%
\begin{equation}\label{eqn:continuumMidTermReflection:270}
0 = \PD{t}{\rho} + \spacegrad \cdot (\rho \Bu).
\end{equation}
%
Expanding the divergence by chain rule we have
%
\begin{equation}\label{eqn:continuumMidTermReflection:290}
\PD{t}{\rho} +\Bu \cdot \spacegrad \rho = -\rho \spacegrad \cdot \Bu,
\end{equation}
%
but this is just
%
\begin{equation}\label{eqn:continuumMidTermReflection:310}
\frac{d\rho}{dt} = -\rho \spacegrad \cdot \Bu.
\end{equation}
%
So, for an incompressible fluid (one for which \(d\rho/dt =0\)), we must also have \(\spacegrad \cdot \Bu = 0\).
} % end answer
%
\makeoproblem{Steady simple shearing flow.}{problem:fluids:midterm:p2}
{2012 midterm, p2}
{
Consider steady simple shearing flow \(\Bu = \xcap u(y)\) as shown in \cref{fig:continuumMidtermReflection:continuumMidtermReflectionFigQ1} with imposed constant pressure gradient (\(G = -dp/dx\)), \(G\) being a positive number, of a single layer fluid with viscosity \(\mu\).
%
\imageFigure{../figures/phy454-continuumechanics/continuumMidtermReflectionFigQ1}{Shearing flow with pressure gradient and one moving boundary.}{fig:continuumMidtermReflection:continuumMidtermReflectionFigQ1}{0.2}
%
The boundary conditions are no-slip at the lower plate (\(y = h\)).  The top plate is moving with a velocity \(-U\) at \(y = h\) and fluid is sticking to it, so \(u(h) = -U\), \(U\) being a positive number.  Using the Navier-Stokes equation.
%
\makesubproblem{Derive the velocity profile of the fluid.}{problem:fluids:midterm:p2a}
\makesubproblem{Draw the velocity profile with the direction of the flow of the fluid when \(U = 0\), \(G \ne 0\).}{problem:fluids:midterm:p2b}
\makesubproblem{Draw the velocity profile with the direction of the flow of the fluid when \(G = 0\), \(U \ne 0\).}{problem:fluids:midterm:p2c}
\makesubproblem{Using linear superposition draw the velocity profile of the fluid with the direction of flow qualitatively when \(U \ne 0\), \(G \ne 0\). (i) low \(U\), (ii) large \(U\).}{problem:fluids:midterm:p2d}
\makesubproblem{Calculate the maximum speed when \(U \ne 0\), \(G \ne 0\).}{problem:fluids:midterm:p2e}
\makesubproblem{Calculate the flux (the volume flow rate) when \(U \ne 0\), \(G \ne 0\).}{problem:fluids:midterm:p2f}
\makesubproblem{Calculate the mean speed when \(U \ne 0\), \(G \ne 0\).}{problem:fluids:midterm:p2g}
\makesubproblem{Calculate the tangential force (per unit width) \(F_x\) on the strip \(0 \le x \le L\) of the wall \(y = -h\) when \(U \ne 0\), \(G \ne 0\).}{problem:fluids:midterm:p2h}
} % makeoproblem
%
\makeanswer{problem:fluids:midterm:p2}{
%
\makesubanswer{Velocity profile.}{problem:fluids:midterm:p2a}
Our equations of motion are
%
\begin{subequations}
\begin{equation}\label{eqn:continuumMidTermReflection:330}
0 = \spacegrad \cdot \Bu,
\end{equation}
\begin{equation}\label{eqn:continuumMidTermReflection:350}
\cancel{\rho \PD{t}{\Bu}} + (\Bu \cdot \spacegrad) \Bu = - \spacegrad p + \mu \spacegrad (\cancel{\spacegrad \cdot \Bu}) + \mu \spacegrad^2 \Bu + \cancel{\rho \Bg}.
\end{equation}
\end{subequations}
%
Here, we have used the steady state condition and are neglecting gravity, and kill off our mass compression term with the incompressibility assumption.  In component form, what we have left is
%
\begin{equation}\label{eqn:continuumMidTermReflection:370}
\begin{aligned}
0 &= \partial_x u \\
u \cancel{\partial_x u} &= -\partial_x p + \mu \spacegrad^2 u \\
0 &= -\partial_y p \\
0 &= -\partial_z p,
\end{aligned}
\end{equation}
%
with \(\partial_y p = \partial_z p = 0\), we must have
%
\begin{equation}\label{eqn:continuumMidTermReflection:390}
\PD{x}{p} = \frac{dp}{dx} = -G,
\end{equation}
%
which leaves us with just
%
\begin{equation}\label{eqn:continuumMidtermReflection:890}
\begin{aligned}
0
&= G + \mu \spacegrad^2 u(y)  \\
&= G + \mu \PDSq{y}{u} \\
&= G + \mu \frac{d^2 u}{dy^2}.
\end{aligned}
\end{equation}
%
Having dropped the partials we really just want to integrate our very simple ODE a couple times
%
\begin{equation}\label{eqn:continuumMidTermReflection:430}
u'' = -\frac{G}{\mu}.
\end{equation}
%
Integrate once
%
\begin{equation}\label{eqn:continuumMidTermReflection:450}
u' = -\frac{G}{\mu} y + \frac{A}{h},
\end{equation}
%
and once more to find the velocity
%
\begin{equation}\label{eqn:continuumMidTermReflection:470}
u = -\frac{G}{2 \mu} y^2 + \frac{A}{h} y + B'.
\end{equation}
%
Let us incorporate an additional constant into \(B'\)
%
\begin{equation}\label{eqn:continuumMidTermReflection:490}
B' = \frac{G}{2 \mu} h^2 + B,
\end{equation}
%
so that we have
%
\begin{equation}\label{eqn:continuumMidTermReflection:510}
u = \frac{G}{2 \mu} (h^2 - y^2) + \frac{A}{h} y + B.
\end{equation}
%
(I did not do use \(B'\) this way on the exam, nor did I include the factor of \(1/h\) in the first integration constant, but both of these should simplify the algebra since we will be evaluating the boundary value conditions at \(y = \pm h\).)
%
\begin{equation}\label{eqn:continuumMidTermReflection:530}
u = \frac{G}{2 \mu} (h^2 - y^2) + \frac{A}{h} y + B.
\end{equation}
%
Applying the velocity matching conditions we have for the lower and upper plates respectively
%
\begin{equation}\label{eqn:continuumMidTermReflection:550}
\begin{aligned}
0 &= \frac{A}{h} (-h) + B \\
-U &= \frac{A}{h} (h) + B.
\end{aligned}
\end{equation}
%
Adding these we find
%
\begin{equation}\label{eqn:continuumMidTermReflection:570}
B = -\frac{U}{2},
\end{equation}
%
and subtracting find
%
\begin{equation}\label{eqn:continuumMidTermReflection:590}
A = -\frac{U}{2}.
\end{equation}
%
Our velocity is
%
\begin{equation}\label{eqn:continuumMidTermReflection:610}
u = \frac{G}{2 \mu} (h^2 - y^2) - \frac{U}{2 h} y -\frac{U}{2},
\end{equation}
%
or rearranged a bit
%
\boxedEquation{eqn:continuumMidTermReflection:630}{
u(y) = \frac{G}{2 \mu} (h^2 - y^2) - \frac{U}{2} \left( 1 + \frac{y}{h} \right).
}
%
\makesubanswer{Zero shear.}{problem:fluids:midterm:p2b}
With \(U = 0\) our velocity has a simple parabolic profile with a max of \(\frac{G}{2 \mu} (h^2 - y^2)\) at \(y = 0\)
%
\begin{equation}\label{eqn:continuumMidTermReflection:650}
u(y) = \frac{G}{2 \mu} (h^2 - y^2).
\end{equation}
%
This is plotted in \cref{fig:continuumMidtermReflection:continuumMidtermReflectionFig3}
\imageFigure{../figures/phy454-continuumechanics/continuumMidtermReflectionFig3}{Parabolic velocity profile.}{fig:continuumMidtermReflection:continuumMidtermReflectionFig3}{0.2}
%
\makesubanswer{Zero pressure gradient.}{problem:fluids:midterm:p2c}
%
With \(G = 0\), we have a plain old shear flow
%
\begin{equation}\label{eqn:continuumMidTermReflection:670}
u(y) = - \frac{U}{2} \left( 1 + \frac{y}{h} \right).
\end{equation}
%
This is linear with minimum velocity \(u = 0\) at \(y = -h\), and a maximum of \(-U\) at \(y = h\).  This is plotted in \cref{fig:continuumMidtermReflection:continuumMidtermReflectionFig4}
\imageFigure{../figures/phy454-continuumechanics/continuumMidtermReflectionFig4}{Shear flow.}{fig:continuumMidtermReflection:continuumMidtermReflectionFig4}{0.2}
%
\makesubanswer{Qualitative sketches.}{problem:fluids:midterm:p2d}
%\unnumberedSubsection{Exam time management note} Somehow I missed this question when I wrote the exam ... I figured this out right at the end when I had run out of time by being too verbose elsewhere.  I am really not very good at writing exams in tight time constraints anymore.

For low \(U\) we will let the parabolic dominate, and can graphically add these two as in \cref{fig:continuumMidtermReflection:continuumMidtermReflectionFig5}
\imageFigure{../figures/phy454-continuumechanics/continuumMidtermReflectionFig5}{Superposition of shear and parabolic flow (low \(U\)).}{fig:continuumMidtermReflection:continuumMidtermReflectionFig5}{0.2}
For high \(U\), we will let the shear flow dominate, and have plotted this in \cref{fig:continuumMidtermReflection:continuumMidtermReflectionFig6}
\imageFigure{../figures/phy454-continuumechanics/continuumMidtermReflectionFig6}{Superposition of shear and parabolic flow (high \(U\)).}{fig:continuumMidtermReflection:continuumMidtermReflectionFig6}{0.2}
%
\makesubanswer{Maximum speed.}{problem:fluids:midterm:p2e}
Since our acceleration is
%
\begin{equation}\label{eqn:continuumMidTermReflection:690}
\frac{du}{dy} = -\frac{G}{\mu} y - \frac{U}{2 h},
\end{equation}
%
our extreme values occur at
%
\begin{equation}\label{eqn:continuumMidTermReflection:710}
y_m = -\frac{U \mu}{2 h G}.
\end{equation}
%
At this point, our velocity is
%
\begin{equation}\label{eqn:continuumMidtermReflection:910}
\begin{aligned}
u(y_m)
&=
\frac{G}{2 \mu} \left(h^2 -
\left( \frac{U \mu}{2 h G} \right)^2
\right) - \frac{U}{2} \left( 1
-\frac{U \mu}{2 h^2 G}
\right) \\
&=
\frac{G h^2}{2 \mu} -\frac{U}{2}
+ \frac{U^2 \mu}{4 h^2 G} \left(
1 -\inv{2}
\right),
\end{aligned}
\end{equation}
%
or just
%
\begin{equation}\label{eqn:continuumMidTermReflection:730}
u_{\text{max}} = \frac{G h^2}{2 \mu} -\frac{U}{2} + \frac{U^2 \mu}{8 h^2 G}.
\end{equation}
%
\makesubanswer{Volume flow rate.}{problem:fluids:midterm:p2f}
An element of our volume flux is
%
\begin{equation}\label{eqn:continuumMidTermReflection:750}
\frac{dV}{dt} = dy dz \Bu \cdot \xcap.
\end{equation}
%
Looking at the volume flux through the width \(\Delta z\) is then
%
\begin{equation}\label{eqn:continuumMidtermReflection:930}
\begin{aligned}
\text{Flux}
&= \int_0^{\Delta z} dz \int_{-h}^h dy u(y) \\
&= \Delta z
\int_{-h}^h dy
\frac{G}{2 \mu} (h^2 - y^2) - \frac{U}{2} \left( 1 + \frac{y}{h} \right) \\
&= \Delta z
\int_{-h}^h dy
\frac{G}{2 \mu} \left(h^2 y - \inv{3} y^3 \right) - \frac{U}{2} \left( y + \frac{y^2}{2 h} \right) \\
&= \Delta z
\left( \frac{2 G h^3}{3 \mu} - U h \right).
\end{aligned}
\end{equation}
%
\makesubanswer{Mean speed.}{problem:fluids:midterm:p2g}
%\unnumberedSubsection{Exam time management note}  I squandered too much time on other stuff and did not get to this part of the problem (which was unfortunately worth a lot).  This is how I think it should have been answered.
We have done most of the work above, and just have to divide the flux by \(2 h \Delta z\).  That is
%
\begin{equation}\label{eqn:continuumMidTermReflection:770}
\expectation{u} = \frac{G h^2}{3 \mu} - \frac{U}{2}.
\end{equation}
%
\makesubanswer{Tangential force on the strip.}{problem:fluids:midterm:p2h}
Our traction vector is
%
\begin{equation}\label{eqn:continuumMidtermReflection:950}
\begin{aligned}
T_1
&= \sigma_{1j} n_j \\
&= \left( -p \delta_{1j} + 2 \mu e_{1j} \right) \delta_{2j} \\
&= 2 \mu e_{12} \\
&= \mu \left(
\PD{y}{u}
+
\cancel{\PD{x}{v}}
\right).
\end{aligned}
\end{equation}
%
So the \(\xcap\) directed component of the traction vector is just
%
\begin{equation}\label{eqn:continuumMidTermReflection:790}
T_1 = \mu \PD{y}{u}.
\end{equation}
%
We have calculated that derivative above in \cref{eqn:continuumMidTermReflection:690}, so we have
%
\begin{equation}\label{eqn:continuumMidtermReflection:970}
\begin{aligned}
T_1
&= \mu \left( -\frac{G}{\mu} y - \frac{U}{2 h} \right) \\
&= - G y - \frac{U \mu}{2 h}.
\end{aligned}
\end{equation}
%
so at \(y = -h\) we have
%
\begin{equation}\label{eqn:continuumMidTermReflection:810}
T_1(-h) = G h - \frac{U \mu}{2 h}.
\end{equation}
%
To see the contribution of this force on the lower wall over an interval of length \(L\) we integrate, but this amounts to just multiplying by the length of the segment of the wall
%
\begin{equation}\label{eqn:continuumMidTermReflection:830}
\int_0^L T_1(-h) dx = \left( G h - \frac{U \mu}{2 h} \right) L.
\end{equation}
%
} % end answer
