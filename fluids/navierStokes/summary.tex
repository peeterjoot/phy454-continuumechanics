%
% Copyright � 2012 Peeter Joot.  All Rights Reserved.
% Licenced as described in the file LICENSE under the root directory of this GIT repository.
%
\section{Summary.}
\subsection{Vector displacements.}
\index{vector displacement}
%
Those portions of the theory of elasticity that we did cover have the appearance of providing some logical context for the derivation of the Navier-Stokes equation.  Our starting point is almost identical, but we now look at displacements that vary with time, forming
%
\begin{equation}\label{eqn:continuumFluidsReview:830}
d\Bx' = d\Bx + d\Bu \delta t.
\end{equation}
%
We compute a first order Taylor expansion of this differential, defining a symmetric strain and antisymmetric vorticity tensor
%
\begin{subequations}
\begin{equation}\label{eqn:continuumFluidsReview:850}
e_{ij} = \inv{2} \left(
\PD{x_j}{u_i} +
\PD{x_i}{u_j} \right).
\end{equation}
\begin{equation}\label{eqn:continuumFluidsReview:870}
\omega_{ij} = \inv{2} \left(
\PD{x_j}{u_i}
-\PD{x_i}{u_j} \right)
\end{equation}
\end{subequations}
%
Allowing us to write
%
\begin{equation}\label{eqn:continuumFluidsReview:890}
dx_i' = dx_i + e_{ij} dx_j \delta t + \omega_{ij} dx_j \delta t.
\end{equation}
%
We introduced vector and dual vector forms of the vorticity tensor with
%
\begin{subequations}
\begin{equation}\label{eqn:continuumFluidsReview:970}
\Omega_k = \inv{2} \partial_i u_j \epsilon_{i j k}
\end{equation}
\begin{equation}\label{eqn:continuumFluidsReview:990}
\omega_{i j} = -\Omega_k \epsilon_{i j k},
\end{equation}
\end{subequations}
%
or
%
\begin{subequations}
\begin{equation}\label{eqn:continuumFluidsReview:910}
\Bomega = \spacegrad \cross \Bu
\end{equation}
\begin{equation}\label{eqn:continuumFluidsReview:930}
\BOmega = \inv{2} (\Bomega)_a \Be_a.
\end{equation}
\end{subequations}
%
We were then able to put our displacement differential into a partial vector form
%
\begin{equation}\label{eqn:continuumFluidsReview:950}
d\Bx' = d\Bx + \left( \Be_i (e_{ij} \Be_j) \cdot d\Bx + \BOmega \cross d\Bx \right) \delta t.
\end{equation}
%
\subsection{Relative change in volume.}
%
We are able to identity the divergence of the displacement as the relative change in volume per unit time in terms of the strain tensor trace (in the basis for which the strain is diagonal at a given point)
%
\begin{equation}\label{eqn:continuumFluidsReview:1010}
\frac{dV' - dV}{dV \delta t} = \spacegrad \cdot \Bu.
\end{equation}
%
\subsection{Conservation of mass.}
%
Utilizing Green's theorem we argued that
%
\begin{equation}\label{eqn:continuumFluidsReview:1030}
\int \left( \PD{t}{\rho} + \spacegrad \cdot (\rho \Bu) \right) dV = 0.
\end{equation}
%
We were able to relate this to rate of change of density, computing
%
\begin{equation}\label{eqn:continuumFluidsReview:1050}
\frac{d\rho}{dt} = \PD{t}{\rho} + \Bu \cdot \spacegrad \rho =
- \rho \spacegrad \cdot \Bu.
\end{equation}
%
An important consequence of this is that for incompressible fluids (the only types of fluids considered in this course) the divergence of the displacement \(\spacegrad \cdot \Bu = 0\).
%
\subsection{Constitutive relation.}
%
We consider only Newtonian fluids, for which the stress is linearly related to the strain.  We will model fluids as disjoint sets of hydrostatic materials for which the constitutive relation was previously found to be
%
\begin{equation}\label{eqn:continuumFluidsReview:1070}
\sigma_{ij} = - p \delta_{ij} + 2 \mu e_{ij}.
\end{equation}
%
\subsection{Conservation of momentum (Navier-Stokes).}
%
As in elasticity, our momentum conservation equation had the form
%
\begin{equation}\label{eqn:continuumFluidsReview:1090}
\rho \frac{du_i}{dt} = \PD{x_j}{\sigma_{ij}} + f_i,
\end{equation}
%
where \(f_i\) are the components of the external (body) forces per unit volume acting on the fluid.
%
\subsection{Observe the first order time derivative here.}
Note that unlike our momentum conservation equation in elasticity \eqnref{eqn:continuumElasticityReview:570}, we have a first order time derivative on the LHS.  This is because \(u_i\) is taken to be a velocity here, but was a position displacement in the elasticity review.

Utilizing the constitutive relation and explicitly evaluating the stress tensor divergence \(\PDi{x_j}{\sigma_{ij}}\) we find
%
\begin{equation}\label{eqn:continuumFluidsReview:1110}
\rho \frac{d\Bu}{dt}
=
\rho \PD{t}{\Bu} + \rho (\Bu \cdot \spacegrad) \Bu
= -\spacegrad p + \mu \spacegrad^2 \Bu
+ \mu \spacegrad (\spacegrad \cdot \Bu) + \rho \Bf.
\end{equation}
%
Since we treat only incompressible fluids in this course we can decompose this into a pair of equations
%
\begin{subequations}
\begin{equation}\label{eqn:continuumFluidsReview:1130}
\rho \PD{t}{\Bu} + \rho (\Bu \cdot \spacegrad) \Bu
= -\spacegrad p
+ \mu \spacegrad^2 \Bu
+ \rho \Bf.
\end{equation}
\begin{equation}\label{eqn:continuumFluidsReview:1150}
\spacegrad \cdot \Bu = 0
\end{equation}
\end{subequations}
%
%
\subsection{No slip condition.}
%
We will find in general that we have to solve for our boundary value conditions.  One of the important constraints that we have to do so will be a requirement (experimentally motivated) that our velocities match at an interface.  This was illustrated with a rocker tank video in class.

This is the no-slip condition, and includes a requirement that the fluid velocity at the boundary of a non-moving surface is zero, and that the fluid velocity on the boundary of a moving surface matches the rate of the surface itself.

For fluids \(A\) and \(B\) separated at an interface with unit normal \(\ncap\) and unit tangent \(\taucap\) we wrote the no-slip condition as
%
\begin{subequations}
\begin{equation}\label{eqn:continuumFluidsReview:1170}
\Bu_A \cdot \taucap = \Bu_B \cdot \taucap
\end{equation}
\begin{equation}\label{eqn:continuumFluidsReview:1190}
\Bu_A \cdot \ncap = \Bu_B \cdot \ncap.
\end{equation}
\end{subequations}
%
For the problems we attempt, it will often be enough to consider only the tangential component of the velocity.
%
\subsection{Traction vector matching at an interface.}
\index{traction vector}
%
As well as matching velocities, we have a force balance requirement at any interface.  This will be expressed in terms of the traction vector
%
\begin{equation}\label{eqn:continuumFluidsReview:1210}
\Btau = \Be_i \sigma_{ij} n_j = \Bsigma \cdot \ncap
\end{equation}
%
where \(\ncap = n_j \Be_i\) is the normal pointing from the interface into the fluid (so the traction vector represents the force of the interface on the fluid).  When that interface is another fluid, we are able to calculate the force of one fluid on the other.

In addition the the constraints provided by the no-slip condition, we will often have to constrain our solutions according to the equality of the tangential components of the traction vector
%
\begin{equation}\label{eqn:continuumFluidsReview:1230}
\evalbar{\tau_i (\sigma_{ij} n_j)}{A} =
\evalbar{\tau_i (\sigma_{ij} n_j)}{B},
\end{equation}
%
We will sometimes also have to consider, especially when solving for the pressure, the force balance for the normal component of the traction vector at the interface too
%
\begin{equation}\label{eqn:continuumFluidsReview:1250}
\evalbar{n_i (\sigma_{ij} n_j)}{A} =
\evalbar{n_i (\sigma_{ij} n_j)}{B}.
\end{equation}
%
As well as having a messy non-linear PDE to start with, our boundary value constraints can be very complicated, making the subject rich and tricky.
%
\subsection{Flux.}
%
A number of problems we did asked for the flux rate.  A slightly more sensible physical quantity is the mass flux, which adds the density into the mix
%
\begin{equation}\label{eqn:continuumFluidsReview:1450}
\int \frac{dm}{dt} = \rho \int \frac{dV}{dt} = \rho \int (\Bu \cdot \ncap) dA
\end{equation}
%
