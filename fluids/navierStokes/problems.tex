%
% Copyright � 2012 Peeter Joot.  All Rights Reserved.
% Licenced as described in the file LICENSE under the root directory of this GIT repository.
%
\makeproblem{Rectilinear flow with shear and pressure gradients}{problem:fluids:review:q1}
{
Solve for the velocity and discuss.
} % makeproblem

\makeanswer{problem:fluids:review:q1}{
Lets specify that we have fluid flowing between surfaces at \(z = \pm h\), the lower surface moving at velocity \(v\) and pressure gradient \(dp/dx = -G\) we find that Navier-Stokes for an assumed flow of \(\Bu = u(z) \xcap\) takes the form

\begin{align}\label{eqn:continuumFluidsReview:1270}
0 &= \partial_x u + \partial_y (0) + \partial_z (0) \\
u \cancel{\partial_x u} &= - \partial_x p + \mu \partial_{zz} u \\
0 &= -\partial_y p \\
0 &= -\partial_z p
\end{align}

We find that this reduces to

\begin{equation}\label{eqn:continuumFluidsReview:1290}
\frac{d^2 u}{dz^2} = -\frac{G}{\mu}
\end{equation}

with solution

\begin{equation}\label{eqn:continuumFluidsReview:1310}
u(z) = \frac{G}{2\mu}(h^2 - z^2) + A (z + h) + B.
\end{equation}

Application of the no-slip velocity matching constraint gives us in short order

\begin{equation}\label{eqn:continuumFluidsReview:1330}
u(z) = \frac{G}{2\mu}(h^2 - z^2) + v \left( 1 - \inv{2h} (z + h) \right).
\end{equation}

With \(v = 0\) this is the channel flow solution, and with \(G = 0\) this is the shearing flow solution.

Having solved for the velocity at any height, we can also solve for the mass or volume flux through a slice of the channel.  For the mass flux \(\rho Q\) per unit time (given volume flux \(Q\))

\begin{equation}\label{eqn:continuumFluidsReview:1350}
\int \frac{dm}{dt}
=
\int \rho \frac{dV}{dt}
=
\rho (\Delta A) \int \Bu \cdot \taucap,
\end{equation}

we find

\begin{equation}\label{eqn:continuumFluidsReview:1370}
\rho Q =
\rho (\Delta y) \left(
\frac{2 G h^3}{3 \mu} + h v
\right).
\end{equation}

We can also calculate the force of the boundaries on the fluid.  For example, the force per unit volume of the boundary at \(z = \pm h\) on the fluid is found by calculating the tangential component of the traction vector taken with normal \(\ncap = \mp \zcap\).  That tangent vector is found to be

\begin{equation}\label{eqn:continuumFluidsReview:1390}
\Bsigma \cdot (\pm \ncap) = -p \zcap \pm 2 \mu \Be_i e_{ij} \delta_{j 3} = - p \zcap \pm \xcap \mu \PD{z}{u}.
\end{equation}

The tangential component is the \(\xcap\) component evaluated at \(z = \pm h\), so for the lower and upper interfaces we have

\begin{align}\label{eqn:continuumFluidsReview:1410}
\evalbar{(\Bsigma \cdot \ncap) \cdot \xcap}{z = -h} &= -G (-h) - \frac{v \mu}{2 h} \\
\evalbar{(\Bsigma \cdot -\ncap) \cdot \xcap}{z = +h} &= -G (+h) + \frac{v \mu}{2 h},
\end{align}

so the force per unit area that the boundary applies to the fluid is

\begin{align}\label{eqn:continuumFluidsReview:1430}
\text{force per unit length of lower interface on fluid} &= L \left( G h - \frac{v \mu}{2 h} \right) \\
\text{force per unit length of upper interface on fluid}
&= L \left( -G h + \frac{v \mu}{2 h} \right).
\end{align}

Does the sign of the velocity term make sense?  Let us consider the case where we have a zero pressure gradient and look at the lower interface.  This is the force of the interface on the fluid, so the force of the fluid on the interface would have the opposite sign

\begin{equation}\label{eqn:continuumFluidsReview:1470}
\frac{v \mu}{2 h}.
\end{equation}

This does seem reasonable.  Our fluid flowing along with a positive velocity is imparting a force on what it is flowing over in the same direction.
} % end answer

