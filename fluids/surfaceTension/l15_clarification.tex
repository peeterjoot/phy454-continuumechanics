%
% Copyright � 2012 Peeter Joot.  All Rights Reserved.
% Licenced as described in the file LICENSE under the root directory of this GIT repository.
%
\section{Traction vector at the interface} \index{traction vector}
%
For a surface like \cref{fig:continuumL15:continuumL15Fig1}
\imageFigure{../figures/phy454-continuumechanics/lec15_Vapor_liquid_interfaceFig1}{Vapor liquid interface.}{fig:continuumL15:continuumL15Fig1}{0.2}
%
we have a discontinuous jump in density.  We will have to consider three boundary value constraints

\begin{enumerate}
\item Mass balance.  This is the continuity equation.
\item Momentum balance.  This is the Navier-Stokes equation.
\item Energy balance.  This is the heat equation.
\end{enumerate}

We have not yet discussed the heat equation, but this is required for non-isothermal problems.

We will define
%
\begin{equation}\label{eqn:continuumL14:150}
\begin{aligned}
\sigma &= \text{surface tension} \\
R &= \text{radius of curvature} \\
\spacegrad_I &= \text{gradient along the interface}
\end{aligned}
\end{equation}
%
and consider the boundary condition at the interface.  Note that we are switching notations for the stress tensor since we will be using \(\sigma\) for surface tension here.

Performing a stress balance at the interface, we express the difference in the traction vector here by
%
\begin{equation}\label{eqn:continuumL15:130}
[\Bt]_1^2 = \Bt_2 - \Bt_1 = -\frac{\sigma}{2 R} \ncap - \spacegrad_I \sigma.
\end{equation}
%
The suffix \(2\) and prefix \(1\) indicates that we are considering the interface between fluids labeled \(1\) and \(2\) (liquid and air respectively in the diagram).

Here the gradient is in the tangential direction of the surface as in \cref{fig:continuumL15:continuumL15Fig2}.
%
\imageFigure{../figures/phy454-continuumechanics/lec15_normal_and_tangent_vectors_on_a_curveFig2}{Normal and tangent vectors on a curve.}{fig:continuumL15:continuumL15Fig2}{0.2}
%
In the normal direction
%
\begin{equation}\label{eqn:l15_clarification:170}
\begin{aligned}
[\Bt]_1^2 \cdot \ncap
&= (\Bt_2 - \Bt_1) \cdot \ncap \\
&= -\frac{\sigma}{2 R}
\end{aligned}
\end{equation}
%
With the traction vector having the value
%
\begin{equation}\label{eqn:l15_clarification:190}
\begin{aligned}
\Bt
&= \Be_i T_{ij} n_j \\
&=
\Be_i \left(
-p \delta_{ij} + \mu \left(
\PD{x_j}{u_i}
+\PD{x_i}{u_j}
\right)
\right)
n_j
\end{aligned}
\end{equation}
%
We have in the normal direction
%
\begin{equation}\label{eqn:continuumL15:10}
\Bt \cdot \Bn
=
n_i \left(
-p \delta_{ij} + \mu \left(
\PD{x_j}{u_i}
+\PD{x_i}{u_j}
\right)
\right) n_j
\end{equation}
%
With \(\Bu = 0\) on the surface, and \(n_i \delta_{ij} n_j = n_j n_j = 1\) we have
%
\begin{equation}\label{eqn:continuumL15:30}
\Bt \cdot \Bn = -p
\end{equation}
%
Returning to \((\Bt_2 - \Bt_1) \cdot \ncap\) we have
%
\boxedEquation{eqn:continuumL15:50}{
-p_2 + p_1 = -\frac{\sigma}{2 R}.
}
%
This is the Laplace pressure.  Note that the sign of the difference is significant, since it effects the direction of the curvature.  This is depicted pictorially in \cref{fig:continuumL15:continuumL15Fig3}
\imageFigure{../figures/phy454-continuumechanics/lec15_pressure_and_curvature_relationshipsFig3}{Pressure and curvature relationships.}{fig:continuumL15:continuumL15Fig3}{0.1}
%
%\unnumberedSubsection{Question}
Note that in \citep{landau1987course} the curvature term is written
%
\begin{equation}\label{eqn:continuumL15:70}
\inv{2 R} \rightarrow \inv{R_1} + \inv{R_2}.
\end{equation}
%
The second radius of curvature is to account for non-spherical surfaces, where we have curvature in two directions.  Illustrating by example, imagine a surface like as in \cref{fig:continuumL15:continuumL15Figq}
\imageFigure{../figures/phy454-continuumechanics/lec15_Example_of_non-spherical_curvatureFigq}{Example of non-spherical curvature.}{fig:continuumL15:continuumL15Figq}{0.2}
%
%\section{Followup required to truly understand things}
%
%While, this review clarifies things, we still do not really know how the surface tension term \(\sigma\) is defined.  Nor have we been given any sort of derivation of \eqnref{eqn:continuumL15:130}, from which the end result follows.

\FIXME{I am assuming that \(\sigma\) is a property of the two fluids at the interface, so if you have, for example, oil and vinegar in a bottle, we have surface tension and curvature that is probably related to how the two of these interact.  If there is still a mixing or settling process occurring, I had imagine that this could even vary from point to point on the surface (imagine adding soap to a surface where stuff can float until the soap mixes in enough that things start sinking in the radius of influence of the soap.)}

Reading: An treatment of this topic that looks complete enough to understand looks like it can be found in \S 7 of \citep{landau1987course}.
%
\section{Surface tension gradients}
%
Now consider the tangential component of the traction vector
%
\begin{equation}\label{eqn:continuumL15:90}
\Bt_2 \cdot \taucap - \Bt_1 \cdot \taucap = - \cancel{ \frac{\sigma}{2 R} \ncap \cdot \taucap} - \taucap \cdot \spacegrad_I \sigma
\end{equation}
%
So we see that for a static fluid, we must have
%
\begin{equation}\label{eqn:continuumL15:110}
\spacegrad_I \sigma = 0
\end{equation}
%
For a static interface there cannot be any surface tension gradient.  This becomes very important when considering stability issues.  We can have surface tension induced flow called capillary, or mandarin (?) flow.

