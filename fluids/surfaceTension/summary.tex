%
% Copyright � 2012 Peeter Joot.  All Rights Reserved.
% Licenced as described in the file LICENSE under the root directory of this GIT repository.
%
\section{Summary.}
%\subsection{Surfaces, normals and tangents}
%
%We reviewed basic surface theory, noting that we can parameterize a surface as in the following example
%
%\begin{equation}\label{eqn:continuumFluidsReview:2010}
%\phi = z - h(x, t) = 0.
%\end{equation}
%
%Computing the gradient we find
%
%\begin{equation}\label{eqn:continuumFluidsReview:2030}
%\spacegrad \phi = -\xcap \PD{x}{h} \zcap.
%\end{equation}
%
%Recalling that the gradient is normal to the surface we can compute the unit normal and unit tangent vectors
%
%\begin{subequations}
%\begin{equation}\label{eqn:continuumFluidsReview:2050}
%\ncap =
%\inv{
%\sqrt{ 1 + \left( \PD{x}{h} \right)^2 }
%}
%\left( -\PD{x}{h}, 1 \right)
%\end{equation}
%\begin{equation}\label{eqn:continuumFluidsReview:2070}
%\taucap =
%\inv{
%\sqrt{ 1 + \left( \PD{x}{h} \right)^2 }
%}
%\left( 1, \PD{x}{h} \right)
%\end{equation}
%\end{subequations}
%
\subsection{Laplace pressure.}
%
It was argued in class that the traction vector differences at the surfaces between a pair of fluids have the form
%
\begin{equation}\label{eqn:continuumFluidsReview:2090}
\Bt_2 - \Bt_1 = -\frac{\sigma}{2 R} \ncap - \spacegrad_I \sigma
\end{equation}
%
where \(\spacegrad_I = \spacegrad - \ncap (\ncap \cdot \spacegrad)\) is the tangential (interfacial) gradient, \(\sigma\) is the surface tension, a force per unit length value, and \(R\) is the radius of curvature.

In static equilibrium where \(\Bt = -p \ncap\) (since \(\Bsigma = 0\) if \(\Bu = 0\)), then dotting with \(\ncap\) we must then have
%
\begin{equation}\label{eqn:continuumFluidsReview:2110}
p_2 - p_1 = \frac{\sigma}{2 R}
\end{equation}
%
Reading: \citep{landau1987course} covers this topic in typical fairly hard to comprehend detail, but there is lots of valuable info there.  \S 2.4.9-2.4.10 of \citep{granger1995fluid} also has small section that is a bit easier to understand, with less detail.  Recommended in that text is the ``Surface Tension in Fluid Mechanics'' movie 
\citep{surfaceTensionNationalCommitteeforFluidMechanics},
%which can be found on youtube in three parts \youtubehref{DkEhPltiqmo}, \youtubehref{yiixltf\_HKw}, \youtubehref{5d6efCcwkWs}, 
which is very interesting and entertaining to watch.
%
\subsection{Surface tension gradients.}
%
Considering the tangential component of the traction vector difference we find
%
\begin{equation}\label{eqn:continuumFluidsReview:2130}
(\Bt_2 - \Bt_1) \cdot \taucap = - \taucap \cdot \spacegrad_I \sigma
\end{equation}
%
If the fluid is static (for example, has none of the creep that we see in the film) then we must have \(\spacegrad_I \sigma = 0\).  It is these gradients that are responsible for capillary flow and other related surface tension driven motion (lots of great examples of that in the film).
%
\subsection{Surface tension for a spherical bubble.}
%
In the film above it is pointed out that the surface tension equation we were shown in class
%
\begin{equation}\label{eqn:continuumFluidsReview:1490}
\Delta p = \frac{2 \sigma}{R},
\end{equation}
%
is only for spherical objects that have a single \textAndIndex{radius of curvature}.  This formula can in fact be derived with a simple physical argument, stating that the force generated by the \textAndIndex{surface tension} \(\sigma\) along the equator of a bubble (as in \cref{fig:continuumFluidsReview:continuumFluidsReviewFig1}), in a fluid would be balanced by the difference in \textAndIndex{pressure} times the area of that equatorial cross section.  That is
%
\imageFigure{../figures/phy454-continuumechanics/fluidsReviewSpherical_bubble_in_liquidFig1}{Spherical bubble in liquid.}{fig:continuumFluidsReview:continuumFluidsReviewFig1}{0.3}
%
\begin{equation}\label{eqn:continuumFluidsReview:1510}
\sigma 2 \pi R = \Delta p \pi R^2
\end{equation}
%
Observe that we obtain \eqnref{eqn:continuumFluidsReview:1490} after dividing through by the area.

