%
% Copyright � 2012 Peeter Joot.  All Rights Reserved.
% Licenced as described in the file LICENSE under the root directory of this GIT repository.
%
%% L14
%\section{Traction vector at the interface}
%
%Recall that our stress tensor has the form
%
%\begin{equation}\label{eqn:continuumL14:90}
%T_{ij} =
%- p \delta_{ij} + \rho \nu \left(
%\PD{x_j}{u_i}
%+\PD{x_i}{u_j}
%\right)
%\end{equation}
%
%
%The traction vector components are
%
%\begin{equation}\label{eqn:continuumL14:110}
%t_i = T_{ij} n_j =
%- p n_i + \rho \nu \left(
%\PD{x_j}{u_i}
%+\PD{x_i}{u_j}
%\right) n_j
%\end{equation}
%
%Considering a control volume as illustrated in we can arrive at what we call the jump stress balance equation
%
%\cref{fig:continuumL14:continuumL14fig3}
%\imageFigure{../figures/phy454-continuumechanics/lec14_Control_volume_for_liquid_air_interfaceFig3}{Control volume for liquid air interface.}{fig:continuumL14:continuumL14fig3}{0.2}
%
%\begin{equation}\label{eqn:continuumL14:130}
%[\BT \ncap]^2_1 = \frac{2 \sigma}{R} \ncap - \spacegrad_I \sigma
%\end{equation}
%
%Force balance along the normal direction gives
%
%\begin{equation}\label{eqn:continuumL14:170}
%\ncap [\BT \ncap]^2_1 = \ncap \cdot \frac{2 \sigma}{R} \ncap - \cancel{\ncap \cdot (\spacegrad_I \sigma)}
%\end{equation}
%
%If you do this calculation, you will get
%
%\begin{equation}\label{eqn:continuumL14:190}
%[-p]^2_1 = \frac{ 2 \sigma}{R}
%\end{equation}
%
%I think this was called the Laplace equation?
%
%Question: How was \(\sigma\) defined?  A: Energy per unit area.
%
%\FIXME{
%\Cref{fig:continuumL14:continuumL14fig4} was given as part of an explanation of surface tension and curvature, but I missed part of that discussion.  Perhaps this is elaborated on in the class notes?
%}
%
%\imageFigure{../figures/phy454-continuumechanics/lec14_Molecular_gas_and_liquid_interactions_at_a_surfaceFig4}{Molecular gas and liquid interactions at a surface.}{fig:continuumL14:continuumL14fig4}{0.2}
%
