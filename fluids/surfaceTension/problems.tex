%
% Copyright � 2012 Peeter Joot.  All Rights Reserved.
% Licenced as described in the file LICENSE under the root directory of this GIT repository.
%
\makeproblem{Meniscus curve against one wall.}{problem:fluids:review:q3}
{
As an application of our surface tension results, solve for the shape of a meniscus of water against a wall.  Work from the brief solution found in \citep{landau1987course} and add sufficient details that the solution can be understood more easily.
} % makeproblem
\makeanswer{problem:fluids:review:q3}{
As in the text we will work with \(z\) axis up, and the fluid up against a wall at \(x = 0\) as illustrated in \cref{fig:continuumFluidsReview:continuumFluidsReviewFig2}.
To get some idea a better feeling for , let us look to a worked problem.
%
\imageFigure{../figures/phy454-continuumechanics/fluidsReviewCurvature_of_fluid_against_a_wallFig2}{Curvature of fluid against a wall.}{fig:continuumFluidsReview:continuumFluidsReviewFig2}{0.4}
%
The starting point is a variation of what we have in class
%
\begin{equation}\label{eqn:continuumFluidsReview:1530}
p_1 - p_2 = \sigma \left( \inv{R_1} + \inv{R_2} \right),
\end{equation}
%
where \(p_2\) is the atmospheric pressure, \(p_1\) is the fluid pressure, and the (signed!) radius of curvatures positive if pointing into medium 1 (the fluid).

For fluid at rest, Navier-Stokes takes the form
%
\begin{equation}\label{eqn:continuumFluidsReview:1630}
0 = -\spacegrad p_1 + \rho \Bg.
\end{equation}
%
With \(\Bg = -g \zcap\) we have
%
\begin{equation}\label{eqn:continuumFluidsReview:1650}
0 = -\PD{z}{p_1} - \rho g,
\end{equation}
%
or
%
\begin{equation}\label{eqn:continuumFluidsReview:1670}
p_1 = \text{constant} - \rho g z.
\end{equation}
%
%At a height \(z\) from the base of the surface (i.e. the bottom of the meniscous), our pressure is
%
%\begin{equation}\label{eqn:continuumFluidsReview:1550}
%p_1 = p_a + \rho g z.
%\end{equation}
%
We have \(p_2 = p_a\), the atmospheric pressure, so our pressure difference is
%
\begin{equation}\label{eqn:continuumFluidsReview:1570}
p_1 - p_2 = \text{constant} - \rho g z.
\end{equation}
%
We have then
%
\begin{equation}\label{eqn:continuumFluidsReview:1590}
\text{constant} -\frac{\rho g z}{\sigma} = \inv{R_1} + \inv{R_2}.
\end{equation}
%
One of our axis of curvature directions is directly along the \(y\) axis so that curvature is zero \(1/R_1 = 0\).  We can fix the constant by noting that at \(x = \infty\), \(z = 0\), we have no curvature \(1/R_2 = 0\).  This gives
%
\begin{equation}\label{eqn:continuumFluidsReview:1690}
\text{constant} -0 = 0 + 0.
\end{equation}
%
That leaves just the second curvature to determine.  For a curve \(z = z(x)\) our absolute curvature, according to \citep{wiki:curvature} is
%
\begin{equation}\label{eqn:continuumFluidsReview:1610}
\Abs{\inv{R_2}} = \frac{\Abs{z''}}{(1 + (z')^2)^{3/2}}.
\end{equation}
%
Now we have to fix the sign.  I did not recall any sort of notion of a signed radius of curvature, but there is a blurb about it on the curvature article above, including a nice illustration of signed radius of curvatures can be found in this \href{http://goo.gl/Wqzz2}{wikipedia radius of curvature figure for a Lemniscate}.  Following that definition for a curve such as \(z(x) = (1-x)^2\) we would have a positive curvature, but the text explicitly points out that the curvatures are will be set positive if pointing into the medium.  For us to point the normal into the medium as in the figure, we have to invert the sign, so our equation to solve for \(z\) is given by
% shorten to get rid of _ that latex does not like.
%http://upload.wikimedia.org/wikipedia/commons/b/b2/Lemniscate_nebeneinander_animated.gif
%
\begin{equation}\label{eqn:continuumFluidsReview:1710}
-\frac{\rho g z}{\sigma} = -\frac{z''}{(1 + (z')^2)^{3/2}}.
\end{equation}
%
The text introduces the capillary constant
%
\begin{equation}\label{eqn:continuumFluidsReview:1730}
a = \sqrt{2 \sigma/ g \rho}.
\end{equation}
%
Using that capillary constant \(a\) to tidy up a bit and multiplying by a \(z'\) integrating factor we have
%
\begin{equation}\label{eqn:continuumFluidsReview:1750}
-\frac{2 z z'}{a^2} = -\frac{z'' z'}{(1 + (z')^2)^{3/2}},
\end{equation}
%
we can integrate to find
%
\begin{equation}\label{eqn:continuumFluidsReview:1770}
A - \frac{z^2}{a^2} = \frac{1}{(1 + (z')^2)^{1/2}}.
\end{equation}
%
Again for \(x = \infty\) we have \(z = 0\), \(z' = 0\), so \(A = 1\).  Rearranging we have
%
\begin{equation}\label{eqn:continuumFluidsReview:1790}
\int dx = \int dz \left( \inv{(1 - z^2/a^2)^2} - 1 \right)^{-1/2}.
\end{equation}
%
Integrating this with Mathematica I get
%
\begin{equation}\label{eqn:continuumFluidsReview:1810}
\begin{aligned}
x - x_0 &=
\sqrt{2 a^2-z^2} \sgn(a-z) \\
&\quad+ \frac{a}{\sqrt{2}} \ln \left(\frac{a \left(2 a-\sqrt{4 a^2-2 z^2} \sgn(a-z)\right)}{z}\right).
\end{aligned}
\end{equation}
%
It looks like the constant would have to be fixed numerically.  We require at \(x = 0\)
%
\begin{equation}\label{eqn:continuumFluidsReview:1830}
z'(0) = \frac{-\cos\theta}{\sin\theta} = -\cot \theta,
\end{equation}
%
but we do not have an explicit function for \(z\).
} % end answer

