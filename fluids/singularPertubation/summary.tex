%
% Copyright � 2012 Peeter Joot.  All Rights Reserved.
% Licenced as described in the file LICENSE under the root directory of this GIT repository.
%
\section{Summary}
\subsection{Singular perturbation theory}

The non-dimensional form of Navier-Stokes had the form

\begin{equation}\label{eqn:continuumFluidsReview:2670}
(\Bu \cdot \spacegrad) \Bu = -\spacegrad p + \inv{\text{Re}} {\spacegrad}^2 \Bu
\end{equation}

where the inverse of Reynold's number

\begin{equation}\label{eqn:continuumFluidsReview:2610}
\text{Re} = \frac{U L}{\nu}
\end{equation}

can potentially get very small.  That introduces an ill-conditioning into the problems that can make life more interesting.

We looked at a couple of simple LDE systems that had this sort of ill conditioning.  One of them was

\begin{equation}\label{eqn:continuumFluidsReview:2630}
\epsilon \frac{du}{dx} + u = x,
\end{equation}

for which the exact solution was found to be

\begin{equation}\label{eqn:continuumFluidsReview:2650}
u = (1 + \epsilon) e^{-x/\epsilon} + x - \epsilon
\end{equation}

The rough idea is that we can look in the domain where \(x \sim \epsilon\) and far from that.  In this example, with \(x\) far from the origin we have roughly

\begin{equation}\label{eqn:continuumFluidsReview:2690}
\epsilon \times 1 + u = x \approx 0 + u
\end{equation}

so we have an asymptotic solution close to \(u = x\).  Closer to the origin where \(x \sim O(\epsilon)\) we can introduce a rescaling \(x = \epsilon y\) to find

\begin{equation}\label{eqn:continuumFluidsReview:2710}
\epsilon \inv{\epsilon} \frac{du}{dy} + u = \epsilon y.
\end{equation}

This gives us

\begin{equation}\label{eqn:continuumFluidsReview:2730}
\frac{du}{dy} + u \approx 0,
\end{equation}

for which we find

\begin{equation}\label{eqn:continuumFluidsReview:2750}
u \propto e^{-y} = e^{-x/\epsilon}.
\end{equation}

