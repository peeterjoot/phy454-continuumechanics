%
% Copyright � 2012 Peeter Joot.  All Rights Reserved.
% Licenced as described in the file LICENSE under the root directory of this GIT repository.
%
% L19
\section{Magnitude of the viscosity and inertial terms}

In the boundary layer analysis we have assumed that our inertial term and viscous terms were of the same order of magnitude.  Lets examine the validity of this assumption

\begin{equation}\label{eqn:continuumL19:410}
u \PD{x}{u} \sim \nu \PDSq{y}{u}
\end{equation}

or

\begin{equation}\label{eqn:continuumL19:430}
\frac{U}{L} \sim \frac{\nu U}{\delta^2}
\end{equation}

or
\begin{equation}\label{eqn:continuumL19:450}
\delta  \sim \frac{\nu L}{U}
\end{equation}

finally
\begin{equation}\label{eqn:continuumL19:470}
\frac{\delta}{L} \sim \frac{\nu}{U L} \sim (\text{Re})^{-1/2}.
\end{equation}

If we have

\begin{equation}\label{eqn:continuumL19:490}
\frac{\delta}{L} \ll 1,
\end{equation}

and
\begin{equation}\label{eqn:continuumL19:510}
\frac{\delta}{L} \ll \inv{\sqrt{\text{Re}}}
\end{equation}

then

\begin{equation}\label{eqn:continuumL19:530}
\frac{\delta}{L} \ll 1
\end{equation}

when \(\text{Re} >> 1\).

(this is the whole reason that we were able to do the previous analysis).

Our EOM is

\begin{equation}\label{eqn:continuumL19:550}
(\Bu \cdot \spacegrad) \Bu = -\frac{\spacegrad p}{\rho } + \nu \spacegrad^2 \Bu
\end{equation}

with

\begin{align}\label{eqn:continuumL19:570}
\Bu &\rightarrow U \\
p &\rightarrow U^2 \rho
\end{align}

as \(x, y \rightarrow L\)

performing a non-dimensionalization we have

\begin{equation}\label{eqn:continuumL19:590}
(\Bu' \cdot \spacegrad') \Bu' = -\spacegrad' p' + \frac{\nu}{U L} {\spacegrad'}^2 \Bu'
\end{equation}

or

\begin{equation}\label{eqn:continuumL19:610}
(\Bu \cdot \spacegrad) \Bu = -\spacegrad p + \inv{\text{Re}} {\spacegrad}^2 \Bu
\end{equation}

to force \(\text{Re} \rightarrow \infty\), we can write

\begin{equation}\label{eqn:continuumL19:630}
\inv{\text{Re}} = \epsilon.
\end{equation}

so that as \(\epsilon \rightarrow 0\) we have \(\text{Re} \rightarrow \infty\).

With a very small number modifying the highest degree partial term, we have a class of differential equations that does not end up converging should we attempt a standard perturbation treatment.  An example that is analogous is the differential equation

\begin{equation}\label{eqn:continuumL19:650}
\epsilon \frac{du}{dx} + u = x,
\end{equation}

where \(\epsilon \ll 1\) and \(u(0) = 1\).  The exact solution of this ill conditioned differential equation is

\begin{equation}\label{eqn:continuumL19:670}
u = (1 + \epsilon) e^{-x/\epsilon} + x - \epsilon
\end{equation}

This is illustrated in \cref{fig:continuumL19:continuumL19Fig3}.

\imageFigure{../../figures/phy454/lec19_Solution_to_the_ill_conditioned_first_order_differential_equationFig3}{Solution to the ill conditioned first order differential equation}{fig:continuumL19:continuumL19Fig3}{0.3}

Study of this class of problems is called \textit{Singular perturbation theory}.

When \(x \gg \epsilon\) we have approximately

\begin{equation}\label{eqn:continuumL19:690}
u \sim x,
\end{equation}

but when \(x \sim O(\epsilon)\), \(x/\epsilon \sim O(1)\) we have approximately

\begin{equation}\label{eqn:continuumL19:710}
u \sim e^{-x/\epsilon}.
\end{equation}
