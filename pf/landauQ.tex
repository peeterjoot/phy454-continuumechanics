%
% Copyright � 2014 Peeter Joot.  All Rights Reserved.
% Licenced as described in the file LICENSE under the root directory of this GIT repository.
%
\input{../latex/blogpost.tex}
\renewcommand{\basename}{landauQ}
\renewcommand{\dirname}{notes/FIXMEwheretodirname/}
%\newcommand{\dateintitle}{}
%\newcommand{\keywords}{}

\input{../latex/peeter_prologue_print2.tex}

\beginArtNoToc

\generatetitle{XXX}
%\chapter{XXX}
%\label{chap:landauQ}
%\section{Motivation}
%\section{Guts}

In \citep{landau1987course} Bernoulli's equation is derived from a consideration of
streamlines, starting with Euler's equation written as

\begin{equation}\label{eqn:landauQ:n}
\inv{2} \spacegrad \Bv^2 - \Bv \cross \lr{ \spacegrad \cross \Bv } = - \spacegrad w,
\end{equation}

where \( w \) is the enthalpy.

They introduce the concept of streamlines defined by

\begin{equation}\label{eqn:landauQ:n}
\frac{dx}{v_x}
=
\frac{dy}{v_y}
=
\frac{dz}{v_z},
\end{equation}

and argue that the \( \Bv \cross \lr{ \spacegrad \cross \Bv } \) projected onto the direction of a streamline is zero.

That direction vector would be proportional to the vector
\begin{equation}\label{eqn:landauQ:n}
\Bone \propto \lr{
\frac{v_x}{dx},
\frac{v_y}{dy},
\frac{v_z}{dz}
}.
\end{equation}

The argument that the streamline direction dotted with the cross term is zero states ``The vector \( \Bv \cross \lr{ \spacegrad \cross \Bv } \) is perpendicular to \( \Bv \), and its projection on the direction of \( \Bone \) is therefore zero.''

I don't understand this argument.  With the streamline components scaled by the factors \( dx, dy, dz \) at each point, I don't understand how the streamline direction is tangent to

\EndArticle
